% ========================
% = TODO: Document
% ========================

% Marc's font stack
% \usepackage{cmbright}       % Sans serif
\usepackage{sourcecodepro}  % Monospace

% Change default font
\usepackage{helvet}
\renewcommand{\familydefault}{\sfdefault}

%\usepackage[francais]{babel}
%\usepackage[french]{babel}

% increase the reference separator size
\setlength\bibitemsep{\baselineskip}

\usepackage{tikz}
\usetikzlibrary{positioning}

\setlength{\parindent}{36pt}

\usepackage{setspace}
\onehalfspacing

\usepackage{tabularx}

\usepackage{tcolorbox}
\tcbset{
  myBox/.style={
    colback=gray!5,
    colframe=gray!60,
    rounded corners,
    sharp corners=south,
    leftrule=1pt,
    rightrule=1pt,
    toprule=1pt,
    bottomrule=1pt,
    boxrule=0.5pt,
    arc=2pt,
    enlarge left by=0mm,
    enlarge right by=0mm,
    drop shadow=black!20
  }

  
}


\captionsetup{justification=centering}

\usepackage{enumitem}
\usepackage{fontawesome5}

% https://blog.chapagain.com.np/latex-numbering-subsubsection-and-showing-it-in-table-of-contents/
\setcounter{tocdepth}{3}
\setcounter{secnumdepth}{3}

\usepackage{float}

% Bullet for items in table
\newcommand{\tabitem}{{\textbullet}\hspace*{0.2cm}}

\usepackage{url}
\usepackage{caption}
\usepackage{listings}


\DeclareUnicodeCharacter{2264}{\ensuremath{\leq}}  % ≤
\DeclareUnicodeCharacter{2265}{\ensuremath{\geq}}  % ≥
\DeclareUnicodeCharacter{2009}{\,}                 % fine-space U+2009
\DeclareUnicodeCharacter{202F}{\,}                 % NNBSP   U+202F
\DeclareUnicodeCharacter{2248}{\ensuremath{\approx}}   % ≈

