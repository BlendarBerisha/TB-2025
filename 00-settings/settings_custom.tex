% ========================
% = TODO: Document
% ========================

% Marc's font stack
% \usepackage{cmbright}       % Sans serif
\usepackage{sourcecodepro}  % Monospace

\usepackage{amssymb}
\usepackage{pifont}
\usepackage{flafter}
% Change default font
\usepackage{helvet}
\renewcommand{\familydefault}{\sfdefault}
\AtEveryCite{\normalfont}

\usepackage{tikz}
\usetikzlibrary{arrows.meta,positioning,calc,fit,shapes.geometric,shapes.misc,backgrounds}
\usepackage[table,xcdraw,dvipsnames]{xcolor}
\usepackage{listings}
\lstset{
  inputencoding=utf8,
  extendedchars=true,
  literate=
    {é}{{\'e}}1 {è}{{\`e}}1 {ê}{{\^e}}1
    {à}{{\`a}}1 {ù}{{\`u}}1
    {É}{{\'E}}1 {À}{{\`A}}1 {È}{{\`E}}1
    {ç}{{\c{c}}}1 {Ç}{{\c{C}}}1
}
\usepackage{ragged2e} % Fix for \RaggedRight undefined control sequence
\usepackage[utf8]{inputenc}
\usepackage[T1]{fontenc}
\usepackage[T1]{fontenc}
 
\usepackage{enumitem}
\usepackage{fontawesome5}
\usepackage{tabularx}
\usepackage{tabularx,booktabs,makecell}
\newcolumntype{Y}{>{\RaggedRight\arraybackslash}X}

\usepackage{listings}
\lstset{
  basicstyle=\ttfamily\footnotesize,
  breaklines=true,
  breakatwhitespace=true,
  columns=fullflexible,
  keepspaces=true,
  showstringspaces=false
}




%\usepackage[francais]{babel}
%\usepackage[french]{babel}

% increase the reference separator size
\setlength\bibitemsep{\baselineskip}

\usepackage{tikz}
\usetikzlibrary{positioning}

\setlength{\parindent}{36pt}

\usepackage{setspace}
\onehalfspacing

\usepackage{tabularx}

\usepackage{tcolorbox}
\tcbset{
  myBox/.style={
    colback=gray!5,
    colframe=gray!60,
    rounded corners,
    sharp corners=south,
    leftrule=1pt,
    rightrule=1pt,
    toprule=1pt,
    bottomrule=1pt,
    boxrule=0.5pt,
    arc=2pt,
    enlarge left by=0mm,
    enlarge right by=0mm,
    drop shadow=black!20
  }

  
}

\usepackage{listings}
\usepackage{xcolor}   

% couleurs facultatives

\usepackage{listings} % --- définition minimale JSON -----------------------
\lstdefinelanguage{json}{
  basicstyle=\ttfamily\small,
  showstringspaces=false,
  morestring=[b]",
  stringstyle=\color{orange},
  comment=[l]{:\ },           % pour colorer les deux-points
  commentstyle=\color{black},
  moredelim=[s][\color{gray}]{\{}{\}},
  moredelim=[s][\color{gray}]{[}{]},
}
\lstdefinelanguage{TypeScript}{
  keywords={class, export, implements, import, interface, let, new, null, public, private, protected, static, super, this, typeof, var, void, while, with, yield, async, await, constructor, enum, extends, false, true, from, as},
  keywordstyle=\color{blue}\bfseries,
  ndkeywords={@Component, @Injectable, @NgModule},
  ndkeywordstyle=\color{orange}\bfseries,
  identifierstyle=\color{black},
  sensitive=true,
  comment=[l]{//},
  morecomment=[s]{/*}{*/},
  commentstyle=\color{gray}\ttfamily,
  stringstyle=\color{red}\ttfamily,
  morestring=[b]',
  morestring=[b]"
}

\lstdefinelanguage{JavaScript}{
  keywords={break, case, catch, class, const, continue, debugger, default, delete, do, else, export, extends, finally, for, function, if, import, in, instanceof, let, new, return, super, switch, this, throw, try, typeof, var, void, while, with},
  keywordstyle=\color{blue}\bfseries,
  ndkeywords={async, await},
  ndkeywordstyle=\color{orange}\bfseries,
  identifierstyle=\color{black},
  sensitive=true,
  comment=[l]{//},
  morecomment=[s]{/*}{*/},
  commentstyle=\color{gray}\ttfamily,
  stringstyle=\color{red}\ttfamily,
  morestring=[b]',
  morestring=[b]"
}

\lstdefinelanguage{bash}{
  keywords={if, then, else, elif, fi, for, in, do, done, while, until, case, esac, function, select, break, continue, return},
  keywordstyle=\color{blue}\bfseries,
  identifierstyle=\color{black},
  sensitive=true,
  comment=[l]{\#},
  morecomment=[s]{/*}{*/},
  commentstyle=\color{gray}\ttfamily,
  stringstyle=\color{red}\ttfamily,
  morestring=[b]',
  morestring=[b]"
} 

% définition de CSS
\lstdefinelanguage{CSS}{
  keywords={@charset, @import, @media, @namespace, @page, @supports, @font-face, @keyframes, @viewport},
  keywordstyle=\color{blue}\bfseries,
  identifierstyle=\color{black},
  sensitive=true,
  comment=[l]{//},
  morecomment=[s]{/*}{*/},
  commentstyle=\color{gray}\ttfamily,
  stringstyle=\color{red}\ttfamily,
  morestring=[b]',
  morestring=[b]"
}

\lstdefinelanguage{PowerShell}{
  morekeywords={New-SelfSignedCertificate, Export-PfxCertificate, ConvertTo-SecureString},
  sensitive=true,
  morecomment=[l]{\#}, % Properly escaped hash symbol
  morestring=[b]"
}

\lstdefinelanguage{yaml}{
  keywords={true,false,null,y,n},
  sensitive=false,
  comment=[l]{\#},
  morecomment=[s]{/*}{*/},
  morestring=[b]",
  morestring=[b]',
  stringstyle=\color{red},
  commentstyle=\color{gray}\ttfamily,
  keywordstyle=\color{blue}\bfseries,
}

\captionsetup{justification=centering}

\usepackage{enumitem}
\usepackage{fontawesome5}
\usepackage{rotating}   % sidewaystable
\usepackage{tabularx}   % colonne X
\usepackage{booktabs}   % top/mid/bottomrule
\usepackage{siunitx} % For \SI command
\usetikzlibrary{calc}
\usepackage[utf8]{inputenc} % Ensure UTF-8 encoding

% https://blog.chapagain.com.np/latex-numbering-subsubsection-and-showing-it-in-table-of-contents/
\setcounter{tocdepth}{3}
\setcounter{secnumdepth}{3}

\usepackage{float}

% Bullet for items in table
\newcommand{\tabitem}{{\textbullet}\hspace*{0.2cm}}

\usepackage{url}
\usepackage{caption}
\usepackage{listings}


\DeclareUnicodeCharacter{2264}{\ensuremath{\leq}}  % ≤
\DeclareUnicodeCharacter{2265}{\ensuremath{\geq}}  % ≥
\DeclareUnicodeCharacter{2009}{\,}                 % fine-space U+2009
\DeclareUnicodeCharacter{202F}{\,}                 % NNBSP   U+202F
\DeclareUnicodeCharacter{2248}{\ensuremath{\approx}}   % ≈

