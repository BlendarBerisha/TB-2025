% -----------------------------------------------------------------
% Résumé – Création de composants BI custom dans Power BI
% -----------------------------------------------------------------
\chapter*{Résumé}
\setlength{\parindent}{0pt}
\markboth{Résumé}{Résumé}
\addcontentsline{toc}{chapter}{Résumé}

Les visuels natifs de Power BI ne couvrant pas certains besoins analytiques d’ECRINS SA, l’entreprise souhaite internaliser la conception de visuels personnalisés. Ce travail poursuit deux objectifs : (1) établir un cadre méthodologique complet — de l’analyse à la mise en production — et (2) démontrer la faisabilité à travers deux prototypes : une Passenger-Flow Map (marketing aéroportuaire) et un Radial Sunburst Decomposition Tree  (analyse budgétaire).

La démarche combine revue de littérature, cadrage avec le métier et développement structuré en phases, appuyé par un outillage de build et d’automatisation. Les visuels sont réalisés en TypeScript et D3, intégrés à Power BI via pbiviz, puis évalués sur des jeux de données de démonstration selon des critères de performance, d’ergonomie et de maintenabilité. Une chaîne de livraison outillée (tests, packaging, distribution interne) soutient l’industrialisation.

L’évaluation technique et fonctionnelle confirme la validité des deux prototypes et la reproductibilité du processus proposé. Les livrables — code, pipeline d’automatisation, guides et annexes — fournissent à ECRINS SA une base réutilisable pour futurs développements de visuels personnalisés et pour renforcer la gouvernance BI.

\vspace{0.8em}
Mots-clés : Business Intelligence ; Power BI ; visuels personnalisés ; TypeScript/D3 ; performance ; ergonomie ; industrialisation.
