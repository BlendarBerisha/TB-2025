% -----------------------------------------------------------------
% Résumé – Création de composants BI custom dans Power BI
% -----------------------------------------------------------------
\chapter*{Résumé}
\setlength{\parindent}{0pt}
\markboth{Résumé}{Résumé}
\addcontentsline{toc}{chapter}{Résumé}

\noindent
Les visuels standard de Power BI ne couvrant plus l’ensemble des besoins analytiques des clients d’ECRINS SA, l’entreprise souhaite internaliser la création de \textbf{visuels personnalisés} (\emph{custom visuals}).  
Ce travail de Bachelor poursuit deux finalités : \textit{(1)} élaborer un \textbf{cadre méthodologique complet} — de l’analyse fonctionnelle jusqu’au déploiement automatisé — pour développer ces composants, et \textit{(2)} en démontrer la faisabilité au moyen d’un \textbf{prototype de visuel pilote}.

La démarche s’appuie sur une étude critique des visuels natifs, un benchmark d’autres plateformes BI et une collaboration étroite avec la responsable produit afin de cibler un besoin prioritaire.  
Le développement adopte un processus itératif inspiré des méthodes Agiles : configuration de l’environnement \texttt{pbiviz}, implémentation en TypeScript / D3, tests et intégration continue via GitHub Actions, puis documentation et bonnes pratiques de gouvernance.

L’évaluation combine des tests techniques, une revue de code et une validation fonctionnelle auprès du client interne.  
Les livrables — code source, pipeline CI/CD opérationnel et guide d’intégration — constituent une base réutilisable pour accélérer la livraison future de visuels sur mesure tout en renforçant la gouvernance BI d’ECRINS SA.

\vspace{0.8em}
\noindent\textbf{Mots clés :} Business Intelligence ; Power BI ; visuels personnalisés ; CI/CD ; gouvernance

