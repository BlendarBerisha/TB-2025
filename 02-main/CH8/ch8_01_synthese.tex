% =============================================================
% ch8_01_synthese.tex
% -------------------------------------------------------------
% Sous-section 8.1 – Synthèse factuelle et bilan du projet
% =============================================================

\section{Synthèse}
\label{sec:chap8-synthese}

Le projet répond point par point à la problématique énoncée au chapitre~1.2 et aux objectifs du chapitre~1.3 : définir des normes de développement et de mise en production pour des visuels Power BI personnalisés, puis démontrer leur applicabilité au travers de deux composants représentatifs diffusables en interne. Les visuels \textit{Passenger-Flow Map} (marketing aéroportuaire) et \textit{Sunburst hiérarchique / Decomposition Tree} (analyse budgétaire) ont été conçus, implémentés, testés et empaquetés en \texttt{.pbiviz}, avec une chaîne CI/CD et une procédure de diffusion interne abouties, sans recours à AppSource.

Sur le plan fonctionnel, \textit{Passenger-Flow Map} illustre la capacité à traiter un besoin spatial interactif (cheminements, focus/zoom, superposition type carte de chaleur) en respectant des contraintes de fluidité propres aux scénarios de navigation. \textit{Sunburst} confirme, de son côté, la pertinence d’un composant hiérarchique léger pour l’exploration budgétaire (drill, fil d’Ariane, KPI central), tout en restant réutilisable et sobre en ressources. Dans les deux cas, l’architecture et l’ergonomie sont documentées de manière à faciliter la maintenance et l’extension (chap.~4).

La démarche technique s’appuie sur un playbook reproductible (environnement, structure de projet, scripts \texttt{npm}, débogage) et sur un packaging orchestré par l’outillage officiel avec audit activé. La chaîne GitHub Actions proposée sépare nettement la construction continue et la release, produit systématiquement une empreinte d’intégrité et conserve les artefacts utiles à la revue. La signature interne est traitée comme une option de gouvernance : elle n’est pas requise pour le magasin organisationnel, mais peut être activée via une procédure simple détaillée en annexe~\ref{ann:signature-procedure}, afin d’ajouter l’attribution cryptographique au-delà de l’empreinte.

Au regard des critères de réussite définis au chapitre~3.2, les résultats sont conformes. Les mesures de performance montrent un temps de rendu P95 inférieur ou égal à 300~ms pour chacun des deux visuels, sur le périmètre de test retenu (chap.~7 et annexe~\ref{ann:a3-tests}). La taille des paquets demeure maîtrisée grâce aux contrôles de seuil intégrés à la CI ; l’accessibilité et l’internationalisation sont couvertes selon le périmètre du projet, avec un point d’attention explicite sur la vérification des contrastes lors de l’application de thèmes personnalisés. Côté sécurité et packaging, aucune dépendance réseau ni évaluation dynamique n’a été introduite, et l’audit de certification accompagne chaque empaquetage. La traçabilité est assurée par la publication des artefacts de release (paquet, empreinte, rapport d’audit, journal des modifications), la signature interne venant s’y ajouter lorsque la politique l’exige.

Enfin, les livrables sont immédiatement exploitables : code source des deux visuels avec leurs paquets \texttt{.pbiviz} issus de la CI, modèles de workflows génériques (build et release) réutilisables par dépôt, guide de publication dans le magasin organisationnel (annexe~\ref{ann:org-store-procedure}) et procédure d’activation de la signature interne (annexe~\ref{ann:signature-procedure}). L’ensemble constitue un cadre méthodologique et technique complet pour recourir à des visuels sur mesure lorsque les visuels natifs ne suffisent pas, en garantissant la reproductibilité du build, des contrôles automatiques de qualité et une diffusion interne maîtrisée.
