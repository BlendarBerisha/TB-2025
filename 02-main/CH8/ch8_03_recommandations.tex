\section{Recommandations et perspectives}
\label{sec:chap8-recommandations}

La question à traiter n’est pas de produire davantage de visuels personnalisés par principe, mais d’établir à quelles conditions cette option crée réellement de la valeur pour l’organisation. Au regard des résultats obtenus et du cadre d’industrialisation défini au chapitre~6, la recommandation centrale est la suivante : recourir à un visuel custom lorsque le besoin n’est pas couvert de manière satisfaisante par les visuels natifs ou par un composant du magasin organisationnel, que l’usage pressenti dépasse un seul rapport ponctuel, et que l’on dispose d’un minimum de capacité interne pour l’entretenir. Dans ce cas, l’investissement est justifié par le gain fonctionnel et par la réutilisabilité à l’échelle de l’entreprise ; dans le cas contraire, il est préférable de s’abstenir et d’orienter les efforts vers la modélisation des données ou la composition de visuels existants.

Pour décider en amont, il est utile de poser un triple filtre. Le premier concerne l’adéquation fonctionnelle : si le besoin implique une interaction ou une représentation indisponible dans l’offre standard — par exemple, une cartographie de flux sur plan interne ou une exploration hiérarchique spécifique — et si cette différence améliore significativement l’analyse métier, alors un développement ciblé est pertinent. Le deuxième filtre porte sur la portée et la durée d’usage : un composant envisagé pour plusieurs équipes ou pour des rapports stratégiques sur au moins une année justifie un investissement, tandis qu’un cas isolé à faible audience ne le justifie pas. Le troisième filtre porte sur la soutenabilité : un propriétaire clair du composant, un dépôt dédié, des scripts reproductibles et la chaîne CI/CD réduisent le coût de possession ; à défaut, la dette d’entretien excède rapidement le bénéfice initial.

Sur le plan de la gouvernance, la diffusion interne via le magasin organisationnel demeure la voie privilégiée. Elle évite les contraintes de la place de marché publique, tout en apportant un point de contrôle homogène. La signature cryptographique doit rester une option de confiance interne plutôt qu’une obligation systématique : l’empreinte SHA-256 et l’audit de packaging suffisent dans la majorité des cas ; la signature s’active lorsque la politique l’exige, selon la procédure simple décrite en annexe~\ref{ann:signature-procedure}. Ce positionnement ménage un bon équilibre entre sécurité, traçabilité et charge opérationnelle, en particulier pour une PME.

En termes de capacités, l’organisation gagne à reconnaître formellement le statut de ces composants comme actifs logiciels. Cela implique une propriété applicative identifiée, un cycle de version explicite, un changelog tenu à jour et des seuils de qualité stables (couverture minimale, taille de paquet, audit sans blocant). Une faible capacité récurrente suffit dans la plupart des cas — par exemple une fraction de profil TypeScript front-end en charge de la maintenance et des mises à jour du SDK — à condition que le dispositif d’intégration continue soit effectivement appliqué et que la publication suive la procédure définie. À l’inverse, un développement sans propriétaire, sans seuils ni pipeline reproductible, accroît le risque d’obsolescence et doit être évité.

Du point de vue économique, la décision doit reposer sur un bénéfice observable et non sur l’attrait technique. Un visuel custom devient rentable lorsqu’il réduit un temps d’analyse récurrent, évite des contournements coûteux ou ouvre un mode d’exploration impossible autrement, et lorsqu’il est réutilisé au-delà d’un seul contexte. La trajectoire proposée dans ce mémoire facilite cette rentabilité : la construction déterministe, l’audit systématique, l’empreinte d’intégrité et la publication contrôlée compressent les coûts de maintenance et sécurisent la diffusion. Le choix de ne pas viser AppSource recentre l’effort sur l’usage interne, là où le retour sur investissement est le plus immédiat.

Enfin, les perspectives d’évolution doivent rester pragmatiques. La généralisation de la Passenger-Flow Map passe par une cartographie paramétrable et par une vérification de lisibilité lorsque les flux se densifient ; celle du Sunburst exige une confirmation d’ergonomie sur des hiérarchies plus profondes. Dans les deux cas, il est préférable de n’engager des extensions qu’à partir de retours d’usage documentés, plutôt que sur hypothèses. Si la demande interne se confirme, la création d’un petit noyau réutilisable de composants — styles, utilitaires d’accessibilité, mécanismes de dimensionnement — permettra d’accélérer les futurs développements tout en maîtrisant la taille des paquets. À défaut de tels signaux, la meilleure recommandation est de capitaliser sur l’existant, de conserver les visuels livrés comme références, et d’appliquer avec rigueur la chaîne de publication interne afin de garantir la qualité dans la durée.
