% =============================================================
% ch8_04_pistes_evolution.tex
% -------------------------------------------------------------
% Sous-section 8.4 – Pistes d’évolution
% =============================================================

\section{Pistes d’évolution}
\label{sec:chap8-pistes-evolution}

Les évolutions les plus utiles sont celles qui augmentent la valeur d’usage sans accroître disproportionnellement la charge d’entretien. Deux axes se dégagent : élargir les cas d’usage des visuels existants et durabiliser l’ingénierie qui permet de les maintenir.

Sur le plan fonctionnel, la \textit{Passenger-Flow Map} gagnera à être généralisée au-delà d’un plan unique. L’objectif n’est pas d’en faire une cartographie universelle, mais de permettre, de manière contrôlée, l’import d’un fond vectoriel (par exemple un plan au format SVG validé) et le mappage de coordonnées normalisées. Ce paramétrage rend possible la réutilisation sur d’autres sites tout en conservant les garanties de performance et de lisibilité (gestion des densités, simplification des tracés, règles d’empilement). Du côté du \textit{Sunburst}, la pertinence tient à la profondeur exploitable : il s’agit de confirmer une navigation hiérarchique au-delà de deux ou trois niveaux par un dévoilement progressif (focus + context, gestion des libellés et des troncatures) plutôt que d’augmenter la complexité visuelle. Dans les deux cas, les options d’affichage doivent rester explicites et limitées, pour éviter la dérive vers un composant « fourre-tout ».

Sur le plan de l’ingénierie, l’effort le mieux rentabilisé consiste à factoriser ce qui est commun en une petite bibliothèque interne. Il ne s’agit pas d’un framework, mais d’un ensemble sobre de briques réutilisables — gabarits de rendu, utilitaires D3 (échelles, formats, palettes), helpers d’accessibilité (gestion du focus, titres et descriptions), conventions TypeScript et styles — publiées dans un registre privé et versionnées de manière sémantique. Cette bibliothèque réduit la taille des paquets, accélère les développements ultérieurs et uniformise l’expérience. Elle s’articule naturellement avec la chaîne décrite au chapitre~6 : un dépôt par visuel, une construction déterministe, un audit de packaging systématique et une diffusion via le magasin organisationnel.

La robustesse passe aussi par une qualité mesurée. Plutôt que d’ajouter des contrôles lourds, il est raisonnable d’instituer un budget de performance (temps de rendu médian et P95) et de le faire respecter par des tests simples exécutés dans la chaîne d’intégration, sur un jeu de données de référence. Ce jeu doit couvrir quelques scénarios réalistes (cardinalités plus élevées, hiérarchies profondes, thèmes personnalisés) et servir autant à la recette qu’à la formation. À ce socle peuvent s’ajouter des vérifications ciblées et peu intrusives : un contrôle de licences des dépendances, une analyse statique JavaScript/TypeScript et, lorsque c’est pertinent, une comparaison visuelle par capture d’écran pour prévenir les régressions de rendu. L’objectif est de conserver un pipeline qui protège la qualité sans complexifier l’outillage.

Enfin, la diffusion doit rester maîtrisée et examinée au prisme du besoin. Le magasin organisationnel demeure le canal privilégié ; la signature interne, lorsqu’elle est utile, s’active selon la procédure simple décrite en annexe, mais n’est pas une obligation générale. Une ouverture vers des utilisateurs pilotes peut être envisagée pour recueillir des retours variés avant un élargissement, à condition que les attentes soient cadrées et que les limites connues (notamment sur mobile ou sur l’export) soient clairement documentées. La publication publique n’a pas vocation à être un objectif par défaut ; elle ne se justifie que si un cas d’usage externe durable et un modèle d’entretien sont établis, compte tenu des exigences supplémentaires qu’elle implique.

Ces orientations privilégient la valeur concrète pour les équipes : des visuels sobres, paramétrables à la marge, appuyés par un outillage léger mais fiable, et une gouvernance claire. Elles permettent d’étendre progressivement le périmètre sans compromettre la maintenabilité ni la qualité de service.
