% -----------------------------------------------------------------
% 4.9 – Paragraphe de transition vers les visuels A & B
% Fichier : ch4_09_transition.tex
% -----------------------------------------------------------------

\section{Mise en pratique du playbook : introduction aux prototypes Passenger‑Flow Map et Sunburst budgétaire}

Le présent chapitre a posé le socle méthodologique nécessaire à la création de tout visuel Power BI personnalisé : installation de l’environnement, génération du squelette, cycle de débogage et gestion des erreurs. Cette démarche générique constitue dorénavant la référence interne d’ECRINS SA pour l’industrialisation des composants d’analytique visuelle.

Les sections suivantes (5.A et 5.B) appliquent concrètement ce playbook à deux prototypes distincts :

\begin{enumerate}
  \item le Passenger‑Flow Map, destiné aux équipes marketing aéroportuaires pour cartographier les parcours passagers ;
  \item le Sunburst budgétaire, orienté analyse, qui combine arbre de décomposition et diagramme en anneau pour une exploration plus riche des données hiérarchiques.
\end{enumerate}

Chaque prototype suit la même structure : architecture, pipeline de rendu, définition du DataView, interactions utilisateur, tests et conformité aux exigences d’accessibilité. Ce découpage illustre la réutilisabilité du cadre défini ci‑dessus, tout en mettant en évidence les choix technique.

En nous appuyant sur ce fil conducteur, nous démontrerons dans le chapitre 6 comment automatiser la chaîne CI/CD, signer numériquement les packages et déployer les visuels dans un environnement gouverné. Les enseignements tirés des prototypes serviront alors de levier d’amélioration continue pour la bibliothèque interne de composants visuels d’ECRINS SA.
