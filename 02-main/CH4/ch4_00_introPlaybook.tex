% -----------------------------------------------------------------
% 4.0 – Introduction générale du playbook
% Fichier : ch4_00_introPlaybook.tex
% -----------------------------------------------------------------

\section{Objet et portée du playbook}
\label{sec:ch4_playbook_intro}

Avant de détailler l’architecture des deux visuels choisis, il est impératif de poser le cadre de référence. Cette section formalise un playbook générique pour le développement de visuels personnalisés dans Power~BI, destiné à servir de référentiel interne à ECRINS~SA. Le chapitre poursuit un double objectif : d’une part, fournir une marche à suivre claire, étape par étape, afin que tout développeur puisse reproduire le processus de création d’un composant custom au moyen du SDK officiel de Microsoft ; d’autre part, encadrer ce processus par des bonnes pratiques et par la signalisation des pièges les plus fréquents, de façon à garantir l’industrialisation et la standardisation conformes aux exigences professionnelles de l’entreprise.

Le déroulé côté poste développeur est synthétisé en Figure~\ref{fig:dev-pipeline} : initialisation du projet, exécution locale en mode développeur, packaging audité puis mise à disposition via le magasin organisationnel.

\begin{figure}[h]
  \centering
  \resizebox{\linewidth}{!}{%
  \begin{tikzpicture}[node distance=1.7cm,>=Latex]
    % Styles sûrs (pas de conflit avec des clés TikZ existantes)
    \tikzset{
      stage/.style={draw,rounded corners,fill=gray!10,align=center,inner sep=6pt},
      flowarrow/.style={->,line width=1pt},
      sidenote/.style={font=\scriptsize,align=center}
    }

    % Étapes
    \node[stage] (new) {\textbf{pbiviz new}\\Init projet};
    \node[stage, right=1.8cm of new] (deps) {\textbf{npm ci}\\Dépendances figées};
    \node[stage, right=1.8cm of deps] (start) {\textbf{pbiviz start}\\Dev local (HTTPS)};
    \node[stage, right=1.8cm of start] (pbi) {\textbf{Power~BI Service}\\Mode développeur};
    \node[stage, right=2.0cm of pbi] (pkg) {\textbf{pbiviz package}\\Audit \& artefact};
    \node[stage, right=1.8cm of pkg] (dist) {\textbf{Import / Store orga}\\Mise à dispo interne};

    % Flux
    \draw[flowarrow] (new) -- (deps);
    \draw[flowarrow] (deps) -- (start);
    \draw[flowarrow] (start) -- (pbi);
    \draw[flowarrow] (pbi) -- (pkg);
    \draw[flowarrow] (pkg) -- (dist);

    % Notes
    \node[sidenote,below=.55cm of start] {Certificat dev\\HTTPS installé};
    \node[sidenote,below=.55cm of pkg] {Version 2.0.0.0\\Rapport d’audit archivé};
  \end{tikzpicture}}
  \caption[Cycle développeur pour un visuel Power BI]
  {Cycle développeur pour un visuel Power~BI : initialisation, exécution locale, packaging audité et distribution interne.}
  \label{fig:dev-pipeline}
\end{figure}

Le style adopté est volontairement rigoureux et orienté vers la reproductibilité. Chaque phase clé — qu’il s’agisse de l’installation de l’environnement, de la génération du code initial, du débogage dans Power~BI Service ou du conditionnement du fichier .pbiviz final — est explicitée et illustrée. Cette approche répond à trois besoins identifiés : capitaliser le savoir-faire existant, réduire le temps de montée en compétence des nouveaux arrivants, et diminuer les risques opérationnels lors du déploiement en production.

La progression s’appuie sur neuf sous-sections logiquement enchaînées. Après la présentation du SDK et le rappel des prérequis, le texte détaille la mise en place de l’environnement puis la création d’un composant minimal. Viennent ensuite la description de l’architecture générée et la procédure de débogage avec hot-reload dans Power~BI Service. Un catalogue d’erreurs récurrentes et de correctifs éprouvés occupe la sous-section~4.8, tandis que la sous-section~4.9 propose une checklist de démarrage rapide. Enfin, la sous-section~4.10 assure la transition vers les deux prototypes développés — la Passenger-Flow Map à vocation marketing et le Sunburst budgétaire à visée financière.

En s’appuyant sur cette articulation progressive, le playbook fournit un fil conducteur unique qui évite les allers-retours inutiles et sécurise chaque livraison intermédiaire. Il doit être appliqué systématiquement aux futurs développements afin d’instaurer, au sein de l’équipe BI d’ECRINS~SA, une culture de qualité logicielle commune.
