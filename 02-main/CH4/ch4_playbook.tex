% =============================================================
% Chapter 4 – Playbook générique de développement des visuels Power BI
% -------------------------------------------------------------
% Ce fichier est le "chapitre maître" : il déclare le chapitre et
% inclut chaque sous‑section majeure dans un fichier séparé.
% Le découpage reprend la structure validée avec l’étudiant :
%   4.0 Intro générale + rôle du playbook
%   4.1 Présentation du SDK / pbiviz
%   4.2 Pré‑requis techniques
%   4.3 Mise en place de l’environnement
%   4.4 Création d’un composant minimal
%   4.5 Structure des fichiers générés
%   4.6 Debug et hot‑reload dans Power BI Service
%   4.7 Erreurs courantes & correctifs
%   4.8 Checklist de démarrage rapide
%   4.9 Paragraphe de transition vers les visuels A & B
% Les fichiers .tex correspondants sont à placer dans
% 02-main/CH4/   et nommés ch4_XX_<sujet>.tex.
% =============================================================

\chapter{Playbook générique de développement des visuels Power~BI}
\label{chap:playbook}
\selectlanguage{french}
\setlength{\parindent}{0pt}

% -----------------------------------------------------------------
% 4.0 Intro – place le contexte et le rôle du playbook
% -----------------------------------------------------------------
% 4.0 – Introduction générale du playbook
% Fichier : ch4_00_introPlaybook.tex
% -----------------------------------------------------------------

\section{Objet et portée du playbook}
\label{sec:ch4_playbook_intro}

Avant de détailler l’architecture des deux visuels choisis, il est impératif de poser le cadre de référence. Cette section formalise un playbook générique pour le développement de visuels personnalisés dans Power~BI, destiné à servir de référentiel interne à ECRINS~SA. Le chapitre poursuit un double objectif : d’une part, fournir une marche à suivre claire, étape par étape, afin que tout développeur puisse reproduire le processus de création d’un composant custom au moyen du SDK officiel de Microsoft ; d’autre part, encadrer ce processus par des bonnes pratiques et par la signalisation des pièges les plus fréquents, de façon à garantir l’industrialisation et la standardisation conformes aux exigences professionnelles de l’entreprise.

Le déroulé côté poste développeur est synthétisé en Figure~\ref{fig:dev-pipeline} : initialisation du projet, exécution locale en mode développeur, packaging audité puis mise à disposition via le magasin organisationnel.

\begin{figure}[h]
  \centering
  \resizebox{\linewidth}{!}{%
  \begin{tikzpicture}[node distance=1.7cm,>=Latex]
    % Styles sûrs (pas de conflit avec des clés TikZ existantes)
    \tikzset{
      stage/.style={draw,rounded corners,fill=gray!10,align=center,inner sep=6pt},
      flowarrow/.style={->,line width=1pt},
      sidenote/.style={font=\scriptsize,align=center}
    }

    % Étapes
    \node[stage] (new) {\textbf{pbiviz new}\\Init projet};
    \node[stage, right=1.8cm of new] (deps) {\textbf{npm ci}\\Dépendances figées};
    \node[stage, right=1.8cm of deps] (start) {\textbf{pbiviz start}\\Dev local (HTTPS)};
    \node[stage, right=1.8cm of start] (pbi) {\textbf{Power~BI Service}\\Mode développeur};
    \node[stage, right=2.0cm of pbi] (pkg) {\textbf{pbiviz package}\\Audit \& artefact};
    \node[stage, right=1.8cm of pkg] (dist) {\textbf{Import / Store orga}\\Mise à dispo interne};

    % Flux
    \draw[flowarrow] (new) -- (deps);
    \draw[flowarrow] (deps) -- (start);
    \draw[flowarrow] (start) -- (pbi);
    \draw[flowarrow] (pbi) -- (pkg);
    \draw[flowarrow] (pkg) -- (dist);

    % Notes
    \node[sidenote,below=.55cm of start] {Certificat dev\\HTTPS installé};
    \node[sidenote,below=.55cm of pkg] {Version 2.0.0.0\\Rapport d’audit archivé};
  \end{tikzpicture}}
  \caption[Cycle développeur pour un visuel Power BI]
  {Cycle développeur pour un visuel Power~BI : initialisation, exécution locale, packaging audité et distribution interne.}
  \label{fig:dev-pipeline}
\end{figure}

Le style adopté est volontairement rigoureux et orienté vers la reproductibilité. Chaque phase clé — qu’il s’agisse de l’installation de l’environnement, de la génération du code initial, du débogage dans Power~BI Service ou du conditionnement du fichier .pbiviz final — est explicitée et illustrée. Cette approche répond à trois besoins identifiés : capitaliser le savoir-faire existant, réduire le temps de montée en compétence des nouveaux arrivants, et diminuer les risques opérationnels lors du déploiement en production.

La progression s’appuie sur neuf sous-sections logiquement enchaînées. Après la présentation du SDK et le rappel des prérequis, le texte détaille la mise en place de l’environnement puis la création d’un composant minimal. Viennent ensuite la description de l’architecture générée et la procédure de débogage avec hot-reload dans Power~BI Service. Un catalogue d’erreurs récurrentes et de correctifs éprouvés occupe la sous-section~4.8, tandis que la sous-section~4.9 propose une checklist de démarrage rapide. Enfin, la sous-section~4.10 assure la transition vers les deux prototypes développés — la Passenger-Flow Map à vocation marketing et le Sunburst budgétaire à visée financière.

En s’appuyant sur cette articulation progressive, le playbook fournit un fil conducteur unique qui évite les allers-retours inutiles et sécurise chaque livraison intermédiaire. Il doit être appliqué systématiquement aux futurs développements afin d’instaurer, au sein de l’équipe BI d’ECRINS~SA, une culture de qualité logicielle commune.


% 4.1 Présentation du SDK Power BI Custom Visuals et de pbiviz
% -----------------------------------------------------------------
% 4.1 – Présentation du SDK Power BI Custom Visuals et de pbiviz
% Fichier : ch4_01_presentationSDK.tex
% -----------------------------------------------------------------

\section{Présentation du SDK Power~BI Custom Visuals et de pbiviz}\label{sec:ch4_sdk_pbiviz}

Le Software Development Kit (SDK) Power BI Custom Visuals publié par Microsoft fournit l’infrastructure nécessaire à la conception, à l’empaquetage et à la distribution de visuels additionnels au format .pbiviz. Au cœur de cet écosystème se trouve l’outil en ligne de commande pbiviz, lequel agit comme point d’entrée unique pour l’ensemble des opérations de scaffolding, de compilation et de débogage.

Concrètement, le SDK s’appuie sur l’écosystème Web standard : le code métier est écrit en TypeScript et manipulé à l’exécution comme du JavaScript transpilé ; la couche de présentation repose sur HTML, CSS / LESS et SVG, avec un recours fréquent à \textsc{d3.js} pour la construction de scènes vectorielles. Cette architecture garantit une intégration transparente avec les navigateurs modernes tout en respectant le modèle de sandboxing imposé par Power BI : chaque visuel s’exécute dans un contexte isolé, sans accès direct aux données au-delà de celles explicitement fournies par le host.

L’outil pbiviz simplifie plusieurs étapes clés. D’abord, la génération d’un squelette de projet complet grâce à la commande pbiviz new évite la configuration manuelle d’un environnement Web ; le développeur obtient immédiatement une arborescence contenant le manifeste pbiviz.json, le fichier de capacités capabilities.json, les sources TypeScript ainsi que les ressources statiques. Ensuite, la phase de compilation est orchestrée par un serveur local déclenché via pbiviz start, lequel assure à la fois la transpilation du code, l’injection à chaud des modifications (hot-reload) et l’exposition du visuel sur https://localhost:8080. Enfin, la commande pbiviz package produit le fichier .pbiviz final, prêt à être importé dans un rapport ou diffusé sur AppSource.

En 2025, l’évolution du service Power BI a déplacé la totalité du cycle de débogage vers la plateforme en ligne ; la prise en charge native du mode développeur dans Power BI Desktop a été abandonnée. Cette décision renforce la cohérence de l’expérience développeur, mais impose l’utilisation d’un compte Pro ou Premium Per User ainsi que l’installation d’un certificat SSL local pour sécuriser la communication entre le navigateur et le serveur pbiviz. Ces contraintes techniques seront examinées en détail dans les sous-sections suivantes.

Au-delà de la simplification opérationnelle, le SDK apporte un cadre de gouvernance : la présence d’un fichier manifeste prescrit la version de l’API, centralise les métadonnées auteur, référence les ressources externes et établit la signature numérique requise pour le déploiement en environnement contrôlé. En conséquence, l’adoption de ce kit constitue un premier levier d’industrialisation, puisqu’elle aligne tous les développements internes sur un standard unique, compatible avec les processus CI/CD abordés au chapitre 5.

À la lumière de ces éléments, la section 4.3 précisera les prérequis matériels et logiciels indispensables à l’installation de pbiviz, tandis que la section 4.4 décrira pas à pas la mise en place de l’environnement de développement.


% 4.2 Pré‑requis techniques
% -----------------------------------------------------------------
% 4.2 – Pré-requis techniques
% Fichier : ch4_02_prerequis.tex
% -----------------------------------------------------------------

\section{Pré-requis techniques du développeur}\label{sec:ch4_prerequis}

Avant toute écriture de code, le poste du développeur doit satisfaire un ensemble de conditions incontournables. Leur conformité garantit la compatibilité avec la version 2025 du SDK Power BI Custom Visuals et élimine les blocages techniques susceptibles d’interrompre le cycle de développement.

Le premier composant à installer est Node.js, en version Long Term Support 18.x ou supérieure. Le runtime JavaScript est indispensable : l’outil pbiviz s’appuie sur l’écosystème Node pour exécuter ses commandes, télécharger les dépendances \textsc{npm} et lancer le serveur local. Une version obsolète entraînerait des erreurs de compilation ou d’incompatibilité avec les bibliothèques récentes.

Une fois Node opérationnel, l’utilitaire en ligne de commande pbiviz doit être installé globalement. L’opération s’effectue par la commande : npm install -g powerbi-visuals-tools@latest. L’exécutable est alors ajouté au \textsc{PATH} ; son bon fonctionnement se vérifie en saisissant simplement pbiviz, ce qui affiche l’écran d’aide. Si la commande n’est pas reconnue, il convient de contrôler la variable d’environnement ou de réinstaller le paquet avec des droits élevés.

Le développement et surtout le débogage d’un visuel exigent un compte Power BI Pro ou Premium Per User. Seuls ces niveaux de licence ouvrent l’accès au mode développeur du service en ligne et autorisent le chargement d’un visuel non publié. Il est par ailleurs recommandé de créer un espace de travail dédié aux tests, de manière à isoler les prototypes des rapports de production.

Côté environnement intégré de développement, Microsoft préconise Visual Studio Code, qui conjugue légèreté, intellisense TypeScript, gestion Git intégrée et terminal embarqué. Les exemples et captures d’écran du présent playbook s’appuient sur VS Code, mais tout IDE disposant d’un support TypeScript (WebStorm, Visual Studio, etc.) demeure acceptable.

Durant la phase de débogage, Power BI charge le visuel via HTTPS, même lorsqu’il s’exécute en local. Il est donc nécessaire de générer un certificat SSL auto-signé et de l’approuver sur la machine. La commande pbiviz --install-cert crée et installe ce certificat ; sous Windows, elle ajoute l’entrée correspondante dans le magasin « Autorités de certification racines de confiance ». En environnement verrouillé, l’exécution de scripts PowerShell peut être bloquée ; il faudra alors solliciter l’administrateur ou importer manuellement le certificat généré.

Enfin, depuis 2024, Microsoft a centralisé le mode développeur sur la plateforme en ligne : la fonction de débogage local auparavant disponible dans Power BI Desktop a été supprimée. L’utilisateur doit activer le mode développeur dans les paramètres du service (<app.powerbi.com>) avant de pouvoir insérer le conteneur spécial « Developer Visual ». Cette activation demeure personnelle et persistante, mais reste tributaire de la présence du certificat SSL et du serveur pbiviz en écoute sur localhost:8080. Une inobservation de l’un de ces prérequis se traduit immanquablement par un message d’erreur « Can’t contact visual server ».

La satisfaction de ces exigences constitue la première étape vers un processus d’industrialisation fiable ; les sections suivantes décrivent la mise en place détaillée de l’environnement et la création du composant minimal.


% 4.3 Mise en place de l’environnement de développement
% -----------------------------------------------------------------
% 4.3 – Mise en place de l’environnement de développement
% Fichier : ch4_03_setupEnv.tex
% -----------------------------------------------------------------

\section{Mise en place de l’environnement de développement}\label{sec:ch4_setup_env}

Une fois les prérequis matériels et logiciels réunis, la préparation effective du poste de travail s’effectue en quatre étapes séquentielles : l’installation de Node.js, l’ajout de l’outil pbiviz, la création du certificat SSL local et l’activation du mode développeur sur Power BI Service. Bien que la procédure ne doive être exécutée qu’une seule fois par machine, son respect intégral conditionne la réussite des itérations ultérieures.

La première étape consiste à installer Node.js en version Long Term Support. Le programme d’installation se télécharge depuis le site officiel de Node (« LTS 18.x » au moment de la rédaction). Sous Windows, l’assistant ajoute également npm et met à jour la variable d’environnement \textsc{path}. Un redémarrage peut être nécessaire pour que la commande
\begin{lstlisting}[language=bash]
node -v
\end{lstlisting}
affiche la version installée.

Dès que Node est opérationnel, l’utilitaire Power BI Visuals Tools s’ajoute globalement au moyen de :
\begin{lstlisting}[language=bash]
npm install -g powerbi-visuals-tools@latest
\end{lstlisting}
La présence de l’exécutable se vérifie par
\begin{lstlisting}[language=bash]
pbiviz --help
\end{lstlisting}
qui doit retourner la liste des sous‑commandes disponibles. L’absence de résultat indique soit un problème de droits d’installation, soit une configuration incorrecte du \textsc{path}. Dans un contexte d’entreprise, l’exécution de la commande avec des privilèges élevés – ou son inscription dans le registre système – peut s’avérer nécessaire.

La troisième opération vise à autoriser la communication sécurisée entre le service Power BI et le serveur local pbiviz. L’outil fournit son propre script :
\begin{lstlisting}[language=bash]
pbiviz --install-cert
\end{lstlisting}
Sous Windows 10/11, la commande exploite l’API New‑SelfSignedCertificate pour générer un certificat intitulé PowerBIVisualTest et l’importer dans le magasin « Autorités de certification racines de confiance ». Un dialogue peut apparaître pour demander la confirmation de l’ajout. Sur macOS et Linux, pbiviz s’appuie sur openssl; il est donc impératif que cette dépendance figure dans le \textsc{path}. Un échec silencieux demeure le symptôme le plus courant en environnement verrouillé ; la solution implique généralement la modification temporaire de la politique d’exécution PowerShell ou l’installation manuelle du certificat généré.

La dernière étape se déroule dans le portail en ligne <app.powerbi.com>. Après authentification avec une licence Pro ou Premium Per User, l’utilisateur ouvre Settings \(\rightarrow\) User settings \(\rightarrow\) Developer et active l’option Power BI mode développeur. Dès validation, un pictogramme \textless{}\textgreater{} supplémentaire apparaît dans le volet Visualizations de tout rapport ouvert en mode Édition. Cette icône constitue l’unique porte d’entrée pour charger un visuel non publié.

À ce stade, le poste de développement est pleinement configuré : un certificat SSL approuvé sécurise la connexion sur https://localhost:8080, pbiviz est disponible globalement et le mode développeur est actif côté service. Les sections suivantes décrivent la génération du squelette de visuel minimal et les premières vérifications fonctionnelles.


% 4.4 Création d’un composant minimal avec pbiviz new
% -----------------------------------------------------------------
% 4.4 – Création d’un composant minimal avec pbiviz new
% Fichier : ch4_04_newVisual.tex
% -----------------------------------------------------------------

\section{Création d’un composant minimal avec pbiviz new}\label{sec:ch4_new_visual}

La première activité concrète du développeur consiste à générer la base du projet en recourant à la commande pbiviz new. Cette opération automatise la création de l’arborescence standard, du manifeste et des fichiers sources, évitant ainsi toute configuration manuelle d’un environnement Web complété par Webpack.

\subsection{Génération du squelette de projet}

Depuis un terminal positionné dans le répertoire de travail, on invoque :
\begin{lstlisting}[language=bash]
pbiviz new SampleVisual
\end{lstlisting}
Le paramètre SampleVisual définit le nom interne du composant ; il doit respecter la casse Pascal et rester unique au sein du tenant Power BI afin d’éviter les collisions de GUID. L’outil crée alors un dossier homonyme contenant l’arborescence complète ; le terminal affiche la réussite de chaque étape (création de pbiviz.json, capabilities.json, visual.ts, etc.). Le développeur ouvre aussitôt ce dossier dans Visual Studio Code pour explorer la structure générée.

\subsection{Installation des dépendances \textsc{npm}}

À l’intérieur du répertoire du projet, l’exécution de :
\begin{lstlisting}[language=bash]
npm install
\end{lstlisting}
télécharge et installe l’ensemble des bibliothèques requises : l’API Power BI Visuals, TypeScript, \textsc{d3.js} le cas échéant, ainsi que les définitions de types. La présence d’un dossier node\_modules confirme le succès de l’opération. Il est recommandé de contrôler la version du paquet powerbi-visuals-api dans package.json pour s’assurer qu’elle correspond à la version cible de l’environnement Power BI.

\subsection{Premier lancement en mode développeur}

Le projet minimal peut désormais être compilé et servi localement. La commande suivante :
\begin{lstlisting}[language=bash]
pbiviz start
\end{lstlisting}
construit le bundle JavaScript, démarre un serveur HTTPS sur https://localhost:8080 et active le rechargement dynamique. La console affiche le port d’écoute et le chemin du certificat utilisé ; tant que le processus demeure actif, il publie les mises à jour à chaque sauvegarde de fichier.

\subsection{Vérification dans Power BI Service}

Le développeur se rend ensuite sur le service Power BI, ouvre un rapport de test en mode Édition, puis insère le conteneur «~Developer Visual~» à l’aide de l’icône \textless{}\textgreater{}. Power BI établit une connexion sécurisée vers localhost:8080, télécharge le visuel et instancie la classe Visual. Le composant s’affiche immédiatement avec son contenu par défaut (un compteur d’appels à update() dans le squelette). L’absence de message d’erreur confirme la bonne configuration de l’environnement.

\subsection{Cycle itératif de modification}

À partir de ce point, toute modification du code visual.ts ou des styles visual.less suivie d’une sauvegarde déclenche une recompilation automatique ; le visuel se rafraîchit alors dans le rapport sans qu’il soit nécessaire de recharger la page. Cette boucle courte – écrire, sauvegarder, observer – constitue la pierre angulaire de la productivité et du débogage. Il est toutefois conseillé de surveiller la console du navigateur (F12) afin de détecter rapidement les exceptions JavaScript ou les avertissements de performance.

\subsection{Arrêt propre du serveur}

Le serveur local se termine par la combinaison Ctrl + C. L’outil demande une confirmation éventuelle ; accepter libère le port 8080 et clôt la session SSL. Le développeur veillera à interrompre proprement le processus avant de changer de branche Git ou de mettre à jour les dépendances, faute de quoi un conflit de port ou de certificat pourrait survenir lors du redémarrage.



Une fois cette étape validée, l’équipe dispose d’un composant minimal fonctionnel, prêt à accueillir la logique métier, le mapping des données et les options de formatage décrits dans les sous-sections suivantes.


% 4.5 Structure des fichiers générés (capabilities.json, visual.ts, ...)
% -----------------------------------------------------------------
% 4.5 – Structure des fichiers générés par pbiviz
% Fichier : ch4_05_filesStructure.tex
% -----------------------------------------------------------------

\section{Structure des fichiers générés par pbiviz}\label{sec:ch4_files_structure}

L’invocation de pbiviz new crée automatiquement une arborescence normalisée, conçue pour séparer les métadonnées du visuel, son code source et ses ressources statiques. Cette organisation reflète le cycle de vie imposé par le host Power BI et facilite la maintenance future.

\subsection{Arborescence racine}

Le projet se présente immédiatement sous la forme suivante :
\begin{lstlisting}[language=bash]
SampleVisual/
|-- src/                # code TypeScript du visuel
|   |-- visual.ts
|   `-- settings.ts
|-- assets/             # icones et images integrees
|   `-- icon.png
|-- style/              # feuilles de style LESS ou CSS
|   `-- visual.less
|-- capabilities.json   # contrat de donnees et format
|-- pbiviz.json         # manifeste et metadonnees
|-- package.json        # dependances npm
|-- tsconfig.json       # compilation TypeScript
`-- ...
\end{lstlisting}
Cette disposition garantit qu’un clean build -- lors d’un déploiement automatisé, par exemple -- peut s’effectuer en isolant sans ambiguïté le source code des artefacts binaires produits.

\subsection{capabilities.json}

Le fichier capabilities.json formalise le contrat qu’énonce le visuel à Power BI. D’une part, il décrit les rôles de données attendus : dimensions catégorielles, mesures numériques, hiérarchies ou tables. D’autre part, il définit l’ensemble des objets de formatage qui apparaîtront dans le volet « Format » de l’interface. Chaque propriété est typée, accompagnée d’un identifiant interne et d’une étiquette traduisible. Le DataView remis au code du visuel se conforme strictement à cette déclaration ; toute incohérence se traduit par un DataView vide ou partiel, d’où l’importance de conserver ce fichier comme source de vérité.

\subsection{visual.ts}

visual.ts contient la classe principale Visual qui implémente l’interface IVisual. La méthode update(\dots) orchestre le rendu : récupération des données, transformation, puis construction ou mise à jour du DOM -- généralement via \textsc{d3.js}. Le fichier accueille aussi les méthodes de cycle de vie (enumerateObjectInstances, destroy) et les gestionnaires d’événements.

\subsection{settings.ts}

Le rôle de settings.ts est de typer les options exposées dans capabilities.json. En héritant de DataViewObjectsParser, le développeur déclare des classes qui regroupent les propriétés de formatage : couleurs, booléens, tailles, limites, etc. Toute modification du schéma de format requiert donc un ajustement symétrique dans ce fichier, faute de quoi des valeurs undefined risquent de perturber le rendu.

\subsection{pbiviz.json}

pbiviz.json joue le rôle de manifeste. Il regroupe : le nom interne du visuel, son display name, le GUID unique, la version de l’API ciblée, les informations d’auteur, l’icône, la référence au fichier capabilities.json et à la feuille de style, ainsi que les ressources additionnelles éventuelles. Lors du package build, ce fichier pilote l’inclusion des fichiers et la signature numérique.

\subsection{Interdépendance et bonnes pratiques}

La configuration tripartite formée par capabilities.json, settings.ts et visual.ts impose un principe fondamental : toute modification dans l’un doit se refléter dans les autres. Il est établi une règle de revue de code qui exige que tout pull request modifiant le contrat de données ou le panneau de format inclue systématiquement l’adaptation correspondante dans les classes TypeScript et les tests unitaires. Ce couplage strict constitue le premier rempart contre les régressions fonctionnelles.



% 4.6 Débogage et hot‑reload dans Power BI Service
% -----------------------------------------------------------------
% 4.6 – Débogage et hot‑reload dans Power BI Service
% Fichier : ch4_06_debug.tex
% -----------------------------------------------------------------

\section{Débogage et hot‑reload dans Power BI Service}\label{sec:ch4_debug}

Depuis 2024, Microsoft a entièrement transféré le mode développeur vers la plateforme en ligne ; Power BI Desktop ne prend plus en charge le chargement dynamique d’un visuel en cours de développement. La présente sous‑section décrit la procédure opérationnelle à suivre pour tester un composant en conditions quasi réelles, détecter les anomalies et réduire le temps d’itération entre deux modifications de code.

\subsection{Insertion du visuel développeur}

Une fois le serveur local lancé via :\vspace{-1ex}
\begin{lstlisting}[language=bash]
pbiviz start
\end{lstlisting}
le développeur se connecte à l’adresse <app.powerbi.com> avec un compte disposant de la licence requise et ouvre un rapport de test en mode « Édition ». Dans le volet Visualizations, l’icône <> devient visible à condition d’avoir activé le Developer mode dans les paramètres utilisateur. Un simple clic insère sur la page un conteneur blanc relié à https://localhost:8080. Si la connexion SSL s’établit correctement, Power BI télécharge le bundle JavaScript du visuel et exécute immédiatement la méthode update().

\subsection{Premier affichage et état d’attente}

Le squelette généré par pbiviz new affiche, par défaut, un compteur du nombre d’appels à update(). Tant qu’aucun champ de données n’est associé, le visuel fonctionne en « mode attente » : il se contente d’occuper son rectangle et d’incrémenter le compteur lors des évènements resize ou refresh. Cette phase est un marqueur précieux : elle confirme que le pipeline de chargement fonctionne et que l’environnement a reconnu le certificat.

\subsection{Association de données et contrôles de base}

Le développeur peut ensuite glisser‑déposer un ou plusieurs champs depuis le panneau Fields vers le visuel. Chaque interaction déclenche un nouvel appel à update(), dont les paramètres contiennent désormais un objet DataView. L’exploration de cette structure dans la console du navigateur permet de vérifier :
\begin{enumerate}
  \item la présence des rôles de données attendus ;
  \item l’exactitude des types (catégorie, valeur, hiérarchie) ;
  \item la cohérence des index, notamment lorsque plusieurs tables sont en jeu.
\end{enumerate}
Une pratique courante consiste à loguer la taille des tableaux categorical.values et categorical.categories afin d’anticiper les surcharges mémoire dans le cas de jeux de données volumineux.

\subsection{Hot‑reload automatique}

Tant que le processus pbiviz start reste actif, toute modification du fichier visual.ts, de la feuille de style ou d’un template HTML déclenche une recompilation suivie d’un rechargement silencieux du visuel dans Power BI. Aucun refresh manuel n’est requis, à condition que le développeur demeure sur l’onglet où le rapport est ouvert. Si la modification touche pbiviz.json ou capabilities.json, Power BI exige toutefois un rechargement complet de la page, car ces métadonnées sont chargées au démarrage du conteneur.

\subsection{Diagnostic des erreurs fréquentes}

Deux catégories de messages d’erreur apparaissent régulièrement :
\begin{description}
  \item[« Can’t contact visual server »] indique l’échec de la connexion HTTPS. Les causes probables sont : certificat non approuvé, serveur stoppé ou port 8080 occupé par un autre processus.
  \item[Exceptions JavaScript] relèvent d’un bug dans le code du visuel. Le stack trace est affiché dans la console et se complète d’une mention « onUpdate » si l’erreur survient durant le rendu.
\end{description}
La meilleure approche consiste à garder la console (F12) ouverte, à activer le filtre « Errors » et à insérer temporairement des console.time() / console.timeEnd() autour des blocs de calcul gourmand afin de détecter les goulots d’étranglement.

\subsection{Nettoyage de la session et reprise}

Lorsque les tests sont terminés, l’arrêt du serveur local par Ctrl + C libère le port et met fin à la session SSL. Il est recommandé de supprimer le conteneur développeur du rapport de test afin d’éviter toute tentative de connexion ultérieure à un serveur inexistant. Un nouveau cycle peut commencer en relançant pbiviz start puis en réinsérant l’icône <>. Cette discipline pre‑prod limite les confusions, notamment lorsque plusieurs développeurs travaillent sur des visuels distincts.



À l’issue de cette phase de débogage interactif, le visuel peut être considéré comme stabilisé. L’étape suivante consiste à répertorier les erreurs courantes observées et à documenter systématiquement leurs correctifs, démarche présentée dans la section 4.7.


% 4.7 Erreurs courantes et contournements
% -----------------------------------------------------------------
% 4.7 – Erreurs courantes et contournements
% Fichier : ch4_07_troubleshoot.tex
% -----------------------------------------------------------------

\section{Erreurs courantes et contournements}\label{sec:ch4_troubleshoot}

Malgré l’application rigoureuse du playbook, plusieurs incidents récurrents peuvent survenir au cours du développement d’un visuel Power BI. Les paragraphes suivants recensent les plus fréquents, accompagnés de leur diagnostic et du correctif validé chez ECRINS SA. Cette capitalisation d’expérience vise à réduire le temps de résolution et à homogénéiser les pratiques de support interne.

\subsection{pbiviz introuvable après installation.} La commande renvoie « not recognised ». Dans la quasi‑totalité des cas, le paquet n’a pas été ajouté au \textsc{PATH} global ou l’installation a été lancée sans privilège suffisant. La vérification commence par node -v, puis par une réinstallation en mode administrateur : npm install -g powerbi-visuals-tools@latest. En environnement verrouillé, il est toléré d’utiliser npx pbiviz localement, mais la résolution définitive passe par la correction du \textsc{PATH} système.

\subsection{Échec de connexion HTTPS : « Can’t contact visual server ».} Power BI ne parvient pas à joindre https://localhost:8080. Les motifs les plus fréquents sont : certificat non installé ou non approuvé, serveur pbiviz start arrêté, port 8080 occupé, pare‑feu bloquant le trafic. La procédure standard consiste à : i) relancer pbiviz --install-cert, ii) ouvrir https://localhost:8080/assets dans le navigateur pour accepter le certificat, iii) vérifier la disponibilité du port via netstat ou Get-NetTCPConnection.

\subsection{Icône développeur absente dans le volet des visuels.} L’option mode développeur n’est pas activée ou l’utilisateur n’a pas de licence adéquate. La solution est d’ouvrir Settings → User settings → Developer et de basculer l’interrupteur, puis de rafraîchir la page du rapport. Si l’icône demeure invisible, il faut confirmer la présence d’une licence Pro ou PPU et l’absence de restrictions administratives au niveau tenant.

\subsection{DataView vide ou partiel malgré la sélection de champs.} Lorsque options.dataViews est undefined, le plus souvent les champs glissés ne correspondent pas aux rôles déclarés dans capabilities.json. Le correctif réside dans la vérification du mapping et, au besoin, l’ajout de l’attribut "required": false pour les rôles optionnels. Côté code, la méthode update() doit toujours vérifier la présence de dataViews[0] et afficher un message d’invite si les données sont insuffisantes.

\subsection{Erreur PowerShell lors de pbiviz --install-cert.} Sur certains postes verrouillés, la politique d’exécution bloque le script interne. Le message typique mentionne « execution of scripts is disabled ». Deux solutions : modifier temporairement la politique pour l’utilisateur courant (Set-ExecutionPolicy RemoteSigned -Scope CurrentUser) ou générer manuellement un certificat avec New-SelfSignedCertificate puis placer les fichiers dans le répertoire attendu par pbiviz.

\subsection{Visuel fonctionnel en mode développeur mais défectueux après empaquetage.} Le fichier .pbiviz importé n’affiche rien ou déclenche une erreur générique. Dans la majorité des cas, le problème provient d’une incohérence de version API ou d’un GUID en doublon. Avant la commande pbiviz package, il est recommandé de mettre à jour pbiviz.json ("apiVersion") et d’exécuter un test d’import dans un nouveau rapport vierge.

\subsection{Avertissements TypeScript et incompatibilités de types.} Lors du lancement de pbiviz start, des erreurs de compilation bloquent la chaîne. La démarche consiste à : i) identifier l’origine (généralement une mise à jour de powerbi-visuals-api), ii) aligner la version cible dans package.json et dans pbiviz.json, iii) corriger les signatures des méthodes concernées. Un npm ci suivi d’un pbiviz start garantit la cohérence du lockfile.

\subsection{Taille de package anormalement faible.} Un .pbiviz de quelques kilo‑octets révèle souvent l’exclusion involontaire d’une bibliothèque externe. Depuis la version 3 du SDK, l’attribut externalJS n’est plus supporté ; les dépendances doivent être intégrées via Webpack. La vérification s’effectue en inspectant le dossier dist/ après empaquetage et en contrôlant la présence d’un fichier JavaScript consolidé de plusieurs centaines de kilo‑octets.


La consolidation de cette liste dans un référentiel interne — baptisé cookbook des erreurs — permet aux développeurs de gagner en autonomie et d’écourter les échanges avec le support. Chaque nouvelle occurrence doit être consignée selon le modèle « symptôme – diagnostic – résolution », afin d’alimenter en continu la base de connaissances d’ECRINS SA.


% 4.8 Checklist de démarrage rapide
% -----------------------------------------------------------------
% 4.8 – Checklist de démarrage rapide
% Fichier : ch4_08_checklist.tex
% -----------------------------------------------------------------

\section{Checklist de démarrage rapide}\label{sec:ch4_checklist}

La checklist ci-dessous offre un aide-mémoire synthétique pour préparer un poste de travail et lancer un premier visuel Power BI en moins d’une heure. Chaque point renvoie à la sous-section détaillée correspondante ; son respect chronologique diminue significativement le temps de mise en route et prévient les blocages les plus courants.

\begin{enumerate}[label=\textbf{Étape \arabic*:}, wide=0pt, itemsep=.8em]
  \item \textbf{Installer Node.js (≥ v18 LTS)} : télécharger l’installateur officiel, exécuter avec les options par défaut, puis vérifier la version (\S\,\ref{sec:ch4_prerequis}).

  \item \textbf{Ajouter l’outil pbiviz} : exécuter npm install -g powerbi-visuals-tools@latest et confirmer la disponibilité de l’exécutable via pbiviz --help (\S\,\ref{sec:ch4_prerequis}).

  \item \textbf{Générer le certificat SSL local} : lancer pbiviz --install-cert, accepter l’importation dans le magasin racine et tester l’URL https://localhost:8080/assets (\S\,\ref{sec:ch4_setup_env}).

  \item \textbf{Activer le mode développeur} : sur <app.powerbi.com>, ouvrir Settings → User settings → Developer et basculer l’interrupteur (\S\,\ref{sec:ch4_setup_env}).

  \item \textbf{Créer le squelette du visuel} : dans un dossier de travail, exécuter pbiviz new <NomDuVisuel> puis npm install pour récupérer les dépendances (\S\,\ref{sec:ch4_new_visual}).

  \item \textbf{Démarrer le serveur local} : à la racine du projet, exécuter pbiviz start et laisser la console ouverte pour compiler à chaud (\S\,\ref{sec:ch4_new_visual}).

  \item \textbf{Insérer le conteneur développeur} : dans un rapport de test ouvert en mode Édition, cliquer sur l’icône <> et vérifier que le visuel se charge sans erreur (\S\,\ref{sec:ch4_debug}).

  \item \textbf{Associer des champs de données} : glisser-déposer au moins un rôle requis, observer la mise à jour du compteur ou du rendu et contrôler la structure DataView via la console JavaScript (\S\,\ref{sec:ch4_debug}).

  \item \textbf{Itérer sur le code} : modifier visual.ts ou visual.less, sauvegarder et valider le hot-reload. Si nécessaire, recharger la page après changement de capabilities.json (\S\,\ref{sec:ch4_debug}).

  \item \textbf{Consigner les anomalies} : toute erreur rencontrée doit être ajoutée au cookbook interne avec son correctif, conformément au modèle « symptôme – diagnostic – résolution » (\S\,\ref{sec:ch4_troubleshoot}).

  \item \textbf{Empaqueter pour validation hors mode dev} : lorsque le prototype est stable, exécuter pbiviz package, importer le fichier .pbiviz dans un rapport vierge et répéter les tests de données.
\end{enumerate}

Le respect de cette séquence permet de sécuriser la livraison d’une première version fonctionnelle tout en capitalisant immédiatement les bonnes pratiques et les leçons apprises. 

% 4.9 Paragraphe de transition vers les visuels A & B
% -----------------------------------------------------------------
% 4.9 – Paragraphe de transition vers les visuels A & B
% Fichier : ch4_09_transition.tex
% -----------------------------------------------------------------

\section{Mise en pratique du playbook : introduction aux prototypes Passenger‑Flow Map et Sunburst budgétaire}

Le présent chapitre a posé le socle méthodologique nécessaire à la création de tout visuel Power BI personnalisé : installation de l’environnement, génération du squelette, cycle de débogage et gestion des erreurs. Cette démarche générique constitue dorénavant la référence interne d’ECRINS SA pour l’industrialisation des composants d’analytique visuelle.

Les sections suivantes (5.A et 5.B) appliquent concrètement ce playbook à deux prototypes distincts :

\begin{enumerate}
  \item le Passenger‑Flow Map, destiné aux équipes marketing aéroportuaires pour cartographier les parcours passagers ;
  \item le Sunburst budgétaire, orienté analyse, qui combine arbre de décomposition et diagramme en anneau pour une exploration plus riche des données hiérarchiques.
\end{enumerate}

Chaque prototype suit la même structure : architecture, pipeline de rendu, définition du DataView, interactions utilisateur, tests et conformité aux exigences d’accessibilité. Ce découpage illustre la réutilisabilité du cadre défini ci‑dessus, tout en mettant en évidence les choix technique.

En nous appuyant sur ce fil conducteur, nous démontrerons dans le chapitre 6 comment automatiser la chaîne CI/CD, signer numériquement les packages et déployer les visuels dans un environnement gouverné. Les enseignements tirés des prototypes serviront alors de levier d’amélioration continue pour la bibliothèque interne de composants visuels d’ECRINS SA.


% =============================================================
