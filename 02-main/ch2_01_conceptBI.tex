\section{Concepts BI et datavisualisation}
\label{sec:concepts-bi-dataviz}

La Business Intelligence peut se définir comme l’ensemble des méthodes et technologies visant à transformer des données brutes en connaissances utiles pour la prise de décision organisationnelle. Chen, Chiang et Storey~\parencite{Chen2012} rappellent que la valeur de la BI réside moins dans l’acquisition massive de données que dans la capacité à les modéliser, les analyser et les représenter de manière intelligible pour l’humain. Un rapport plus récent de Gartner~\parencite{Gartner2024} confirme cette évolution vers une BI self-service, dans laquelle la souplesse des visuels et leur rapidité de création constituent des facteurs différenciants.

L’étape de visualisation constitue ainsi le dernier maillon du pipeline \og ingestion, modélisation, analyse, présentation\fg{}, mais elle s’avère décisive pour convertir des métriques abstraites en informations actionnables. C’est précisément à ce niveau que se situent les visuels Power BI, natifs ou personnalisés, objets d’étude du présent travail.

Dans le domaine de la datavisualisation, les sciences cognitives ont montré que l’œil humain perçoit rapidement certains attributs dits préattentifs (position, longueur, orientation, couleur, etc.). Les travaux fondateurs de Cleveland et McGill établissent une hiérarchie de précision des encodages (la position sur une échelle commune étant la plus fiable)~\parencite{ClevelandMcGill1984}, complétés par une synthèse de référence sur les principes et processus perceptifs~\parencite{Munzner2014}. Ware~\parencite{Ware2019} démontre que l’exploitation adéquate de ces attributs maximise la vitesse et la justesse de la lecture visuelle. Tufte~\parencite{Tufte1983} a, pour sa part, popularisé l’idée de data–ink ratio, soulignant que le graphisme ne doit conserver que l’encre strictement indispensable au message ; tout élément décoratif superflu — le chart-junk — nuit à la clarté. Des travaux empiriques plus récents nuancent toutefois cette position, en montrant que la mémorabilité d’un visuel dépend aussi de facteurs perceptifs et sémantiques~\parencite{BorkinEtAl2013}. Few~\parencite{Few2009} prolonge cette perspective en montrant que la cohérence des encodages (axes, couleurs, échelles) constitue une condition essentielle pour comparer de façon fiable plusieurs séries de données.

L’unification théorique de ces principes a été proposée par Wilkinson~\parencite{Wilkinson2005} puis formalisée par Wickham sous le nom de Grammaire des graphiques. Le modèle décrit chaque graphique comme la combinaison déclarative de couches : données, transformations, géométries, échelles, systèmes de coordonnées et facettage. Ce cadre a influencé la plupart des bibliothèques modernes — notamment D3.js, Vega ou ggplot2 — et se retrouve implicitement dans l’API de Power BI ; chaque visuel y spécifie ses champs (data roles), ses encodages (capabilities) et son canevas de rendu. L’adoption de D3 pour les visuels personnalisés s’appuie par ailleurs sur une base scientifique et industrielle reconnue~\parencite{BostockOgievetskyHeer2011}.

Au-delà des principes, la BI professionnelle ajoute des impératifs tels que la performance d’affichage et l’accessibilité numérique, auxquels s’ajoute la qualité des interactions comme levier d’analyse~\parencite{HeerShneiderman2012}. Ces impératifs serviront de grille d’évaluation au chapitre~6. Les visuels standards de Power BI satisfont ces exigences pour des cas courants, mais ils se heurtent aux demandes spécifiques de certains métiers ; c’est autour de ces limites qu’émerge le besoin de visuels personnalisés. Comprendre les fondements de la BI et de la datavisualisation éclaire ainsi la double problématique du mémoire : confirmer la pertinence d’enrichir Power~BI par de nouveaux composants, puis garantir que ces composants respectent les bonnes pratiques cognitives tout en s’intégrant dans un environnement d’entreprise contrôlé.
