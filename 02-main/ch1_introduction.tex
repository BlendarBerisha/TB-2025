% =============================================================
\chapter{Introduction}

\section{Contexte}

Microsoft Power BI s’est imposé ces dernières années comme l’un des outils phares de la Business Intelligence (BI) en entreprise. Il a réussi à reléguer nombre de concurrents au second plan en offrant une solution de visualisation et d’analyse de données intégrée à un coût d’entrée très attractif. De fait, Power BI est devenu extrêmement populaire auprès des organisations et des utilisateurs, au point que près de 30\,\% des entreprises le plébiscitent déjà et que 60\,\% envisagent de l’adopter, selon la \emph{BI Survey 23} de BARC \parencite{Spies2023,Tirupati2023}.  

L’un des atouts majeurs de Power BI réside dans la richesse de ses fonctionnalités natives, en particulier sa large gamme de visuels prédéfinis qui s’enrichit continuellement. Chaque mise à jour de l’outil apporte de nouveaux graphiques et tableaux de bord standards, couvrant un éventail important de besoins analytiques courants. Au-delà de ces visuels par défaut, Power BI offre également la possibilité d’étendre ses capacités en intégrant des composants personnalisés développés à l’aide de langages scripts tels que Python ou R. Cette ouverture permet aux analystes de réaliser des analyses avancées et de créer des visualisations sur mesure allant au-delà de l’offre standard de l’outil \parencite{Dossier2024}.  

En d’autres termes, les utilisateurs bénéficient d’une liberté supplémentaire pour représenter leurs données de façon plus adaptée et innovante qu’avec les seuls graphiques fournis en standard. Cette tendance vers la personnalisation des tableaux de bord est largement documentée dans la littérature académique, qui y voit un facteur clé d’appropriation et de satisfaction utilisateur \parencite{Amyrotos2024}.  

C’est dans ce contexte technologique et métier qu’intervient la société \textbf{ECRINS SA}, une entreprise de conseil en informatique de gestion basée en Valais. ECRINS compte parmi ses clients divers acteurs de premier plan — tels que le Service Industriel de Genève, Rolex, Genève Aéroport ou Tag Heuer — aux besoins décisionnels pointus. Ces clients sollicitent régulièrement des analyses et indicateurs « sur mesure » que les outils standards ne peuvent pas toujours fournir tels quels. Désireuse de maintenir un haut niveau de satisfaction client, l’entreprise cherche à exploiter la flexibilité de Power BI pour concevoir des composants BI \emph{custom}, c’est-à-dire des visuels personnalisés intégrés à des tableaux de bord Power BI. À travers ce travail de Bachelor, elle ambitionne de développer quelques exemples probants de tels visuels et de définir une méthodologie reproductible pour leur conception. Ce projet servira ainsi de base de référence pour implémenter à l’avenir de nouveaux composants BI \emph{custom} répondant aux demandes spécifiques de la clientèle, tout en assurant un standard de qualité et de gouvernance.  

\section{Problématique}

Bien que la perspective de visuels personnalisés soit attrayante, leur mise en place n’a encore jamais été testée par ECRINS. En l’absence d’expérience préalable et de cadres méthodologiques définis, l’entreprise ne dispose pas des repères nécessaires pour évaluer la faisabilité technique ni l’effort de développement qu’implique la création d’un visuel sur mesure dans Power BI. Les clients formulent parfois des demandes qualifiées en interne d’« impossibles », car visant des représentations graphiques ou des fonctionnalités inexistantes dans les visuels standards de Power BI. Jusqu’à présent, ces attentes spécifiques restent soit insatisfaites, soit très difficilement comblées par des solutions de contournement peu élégantes.  

La question qui se pose est donc la suivante : \textit{comment ECRINS peut-elle concevoir et intégrer de nouveaux composants BI \emph{custom} dans Power BI afin de répondre à des besoins métier non couverts par les visuels natifs, tout en définissant un cadre de développement fiable pour y parvenir ?}  

Il s’agit d’une problématique à la fois technique et organisationnelle. D’un point de vue technique, il faut déterminer les outils et approches de réalisation les plus appropriés (scripts R/Python intégrés, développement d’un \emph{visual custom} via le SDK Power BI, etc.), en tenant compte des avantages et limitations de chaque solution. D’un point de vue organisationnel, la littérature montre que la réussite d’un projet BI dépend d’un alignement entre facteurs technologiques, organisationnels et environnementaux \parencite{AlKharusi2023}. Il est donc nécessaire d’établir des normes de développement et de déploiement pour garantir que les composants créés soient pérennes, maintenables et aisément déployables dans l’environnement Power BI de l’entreprise.  

En somme, ECRINS cherche à élargir le champ des possibilités de Power BI de manière maîtrisée, afin de répondre favorablement, à l’avenir, aux demandes de visuels spécifiques de ses clients. Ce défi s’inscrit plus largement dans la problématique de pousser les limites des visuels BI par défaut — enjeu rencontré par de nombreuses organisations — en recourant à la personnalisation \parencite{Uttam2025}. La résolution de cette problématique passera par l’exploration du processus de développement d’un composant BI \emph{custom} de bout en bout, depuis l’identification du besoin jusqu’à la validation du composant final en situation réelle.  

\section{Objectifs}

Ce projet de Bachelor poursuit un objectif principal : démontrer la faisabilité et l’intérêt de la création de visuels personnalisés dans Power BI, puis formaliser une méthode de développement reproductible. Il s’agit d’analyser l’existant afin de recenser les lacunes, de sélectionner un visuel prioritaire à forte valeur ajoutée, de déterminer l’approche technique (scripts ou SDK), d’implémenter le composant dans un tableau de bord de démonstration et de le tester, puis de documenter une procédure standard réutilisable par ECRINS.  

\section{Portée et limites}

Le travail se concentre sur la réalisation d’un prototype unique de composant BI \emph{custom} et sur l’élaboration de recommandations générales, sans prétendre couvrir l’ensemble des possibilités de personnalisation de Power BI ni constituer une bibliothèque exhaustive. Le composant est testé dans un scénario de démonstration reposant sur des données fictives ou publiques et l’environnement Power BI Desktop, éventuellement complété par le service en ligne.  

Certaines limites techniques sont reconnues. Lorsqu’un visuel repose sur un script Python ou R, le rendu se présente sous forme d’image statique, excluant les interactions utilisateur avancées ; cette approche peut également entraîner des rafraîchissements plus lents lors des mises à jour de données \parencite{RealPython2023}. Pour répondre aux exigences d’interactivité de la clientèle, le présent travail retient donc l’approche SDK TypeScript pour le prototype, malgré la complexité supplémentaire. La certification et la publication éventuelle sur AppSource restent toutefois hors périmètre.

\section{Structure du rapport}

Outre la présente introduction, le mémoire s’articule en six chapitres complémentaires. Le chapitre 2 dresse d’abord un état de l’art des visuels Power BI et des solutions concurrentes, afin de situer la problématique dans son contexte technologique. Le chapitre 3 précise ensuite l’origine du besoin, les critères de réussite retenus et l’organisation du travail, ce qui constitue la méthodologie du projet. Le chapitre 4 décrit la conception et la réalisation du visuel personnalisé sélectionné, en détaillant les décisions techniques majeures. Le chapitre 5 formalise la procédure d’industrialisation : pipeline CI/CD, signature numérique et exigences de gouvernance. Le chapitre 6 présente l’évaluation des résultats tant sur le plan technique que fonctionnel, tandis que le chapitre 7 conclut en synthétisant les apports du travail, en discutant ses limites et en ouvrant des perspectives pour ECRINS SA.

% =============================================================
