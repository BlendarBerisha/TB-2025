
\section{SDK Custom Visuals : principes, sécurité et pipeline}
\label{sec:sdk-custom-visuals}

Le Software Development Kit (SDK) pour visuels personnalisés de Power BI permet de créer des composants graphiques entièrement sur mesure en TypeScript, répondant précisément à des besoins analytiques, interactifs ou visuels très spécifiques. Cette section détaille les principes de développement, les aspects sécuritaires à respecter, ainsi que les bonnes pratiques pour gérer un pipeline d’intégration et de déploiement continu (CI/CD).

\subsection{Principes généraux du développement via SDK}

Le développement d’un visuel personnalisé avec le SDK Power BI implique les étapes suivantes :
\begin{enumerate}
  \item Initialisation du projet via l’outil \texttt{pbiviz} (ligne de commande).
  \item Définition des capacités du visuel via le fichier \texttt{capabilities.json}.
  \item Implémentation en TypeScript dans \texttt{src/visual.ts} en suivant l'interface \texttt{IVisual}.
  \item Gestion du rendu graphique à l’aide de bibliothèques comme D3.js, React ou autres frameworks JavaScript.
  \item Compilation et packaging du visuel en fichier \texttt{.pbiviz} prêt à être déployé.
\end{enumerate}

Le SDK impose une structure modulaire claire, facilitant la maintenance et l’évolution du composant au fil des besoins et des versions du rapport.

\subsection{Sécurité et isolation des visuels custom}

Power BI exécute chaque visuel personnalisé dans une sandbox HTML sécurisée (iframe). Ce mécanisme garantit l’isolation complète du code personnalisé par rapport au reste du rapport, protégeant ainsi :
\begin{itemize}
  \item Contre les conflits de bibliothèques JavaScript (versions incompatibles).
  \item Contre les failles potentielles (accès non autorisé au DOM principal).
  \item En limitant strictement les échanges à travers une API contrôlée (sélections, filtres croisés, thèmes).
\end{itemize}

Depuis la version 4.6 du SDK, il est obligatoire de déclarer explicitement les privilèges et autorisations nécessaires (fichiers externes, accès réseau) dans le fichier \texttt{capabilities.json}. Ces contraintes assurent une transparence complète en matière de sécurité et facilitent l'audit des visuels déployés dans un environnement de production.

\subsection{Pipeline CI/CD pour les visuels custom}

Pour industrialiser efficacement le développement et le déploiement des visuels personnalisés, la mise en place d’un pipeline d’intégration et de déploiement continu (CI/CD) est recommandée. Ce pipeline typique inclut les étapes suivantes :
\begin{itemize}
  \item \textbf{Intégration continue (CI)} :
  \begin{itemize}
    \item Validation automatique des changements via des tests unitaires et fonctionnels.
    \item Compilation du projet TypeScript avec vérification des erreurs et des standards de codage.
    \item Génération automatique du package \texttt{.pbiviz} pour chaque commit.
  \end{itemize}
  \item \textbf{Déploiement continu (CD)} :
  \begin{itemize}
    \item Déploiement automatique du visuel sur des environnements de test puis de production.
    \item Automatisation des étapes de vérification de sécurité et d'intégration.
    \item Gestion simplifiée des versions via le contrôle de source (Git, Azure DevOps, GitHub Actions).
  \end{itemize}
\end{itemize}

Ce processus garantit une haute qualité logicielle, une traçabilité complète des modifications, et une réduction drastique des risques liés à des mises en production manuelles ou ponctuelles.

\subsection{Bonnes pratiques et conseils de mise en œuvre}

Pour réussir le développement de visuels personnalisés avec le SDK Power BI, il est essentiel de suivre quelques recommandations clés :
\begin{itemize}
  \item \textbf{Structure claire du code} : Séparer distinctement la logique métier (traitement de la dataView) de la logique d’affichage (D3.js, Canvas).
  \item \textbf{Tests rigoureux} : Couvrir au maximum les méthodes critiques avec des tests unitaires et d’intégration (Jest, Mocha).\
  \item \textbf{Documentation exhaustive} : Documenter clairement le fonctionnement du visuel, les options disponibles et les limites éventuelles pour faciliter l’intégration par les équipes de développement ou les utilisateurs finaux.
  \item \textbf{Optimisation performance} : Privilégier un rendu différentiel (update incrémental), utiliser des méthodes efficaces pour la gestion des grands jeux de données.
\end{itemize}

\subsection*{Conclusion intermédiaire}

Le SDK Custom Visuals de Power BI offre une solution optimale pour répondre à des besoins spécifiques dépassant les capacités natives ou les limites des visuels Python et R. Grâce à son cadre structurant, ses exigences de sécurité élevées, et son potentiel d’automatisation via CI/CD, il représente la meilleure option pour un usage professionnel à grande échelle. Ce choix technologique s’impose particulièrement lorsque la performance, l’interactivité dynamique, et la personnalisation graphique poussée sont des critères essentiels du projet BI.
