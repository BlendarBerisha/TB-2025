\section{Choix technologiques : TypeScript, D3, (option React)}
\label{sec:choix-technologiques}

Créer un visuel Power BI custom revient essentiellement à développer une application web monopage embarquée. À ce titre, le choix des technologies de développement est déterminant pour réussir à la fois l’implémentation technique et la maintenabilité du code.

À l’heure actuelle (2023–2025), un consensus s’est formé autour du trio technologique suivant pour les visuels Power BI : \textbf{TypeScript} comme langage de développement, \textbf{D3.js} comme bibliothèque de visualisation de bas niveau, et éventuellement \textbf{React} (ou un autre framework UI) pour structurer l’interface si nécessaire. Ces choix ne sont pas exclusifs – en principe, le SDK permet l’usage de n’importe quel framework JavaScript du moment qu’il peut s’intégrer dans le bundle en module ES6\footnote{\url{https://www.reddit.com/r/PowerBI/comments/12ggh93/is_it_possible_to_use_react_with_powerbi_custom/}} – mais ils reflètent les recommandations et pratiques courantes dans la communauté.

\subsection{TypeScript comme langage par défaut}

Le SDK Power BI est conçu pour TypeScript, un sur-ensemble typé de JavaScript. Lorsqu’on initialise un projet de visuel, les fichiers générés (ex. \texttt{visual.ts}, \texttt{settings.ts}) sont en TypeScript\footnote{\url{https://learn.microsoft.com/en-us/power-bi/developer/visuals/custom-visual-develop-tutorial}}.

L’adoption de TypeScript présente plusieurs avantages :
\begin{itemize}
  \item (a) Typage statique permettant d’attraper des erreurs à la compilation et de mieux documenter les intentions du code, ce qui est précieux pour des visuels complexes ;
  \item (b) Meilleure productivité grâce à l’autocomplétion et la documentation intégrée dans les IDE (Microsoft fournit les définitions d’interfaces du Power BI Visuals API en TypeScript) ;
  \item (c) Interopérabilité avec JavaScript – TypeScript compile en JavaScript standard, ce qui permet d’utiliser toute librairie JS existante.
\end{itemize}

En somme, TypeScript est devenu le langage de facto pour ce genre de développement front-end structuré. D’après les experts, « pour bien faire les choses, il faut utiliser le SDK TypeScript, qui donne accès à toutes les APIs fournies par Microsoft »\footnote{\url{https://www.reddit.com/r/PowerBI/comments/12ggh93/is_it_possible_to_use_react_with_powerbi_custom/}}.

Un développeur purement JavaScript pourrait en théorie coder un visuel, mais il perdrait les bénéfices du typage et devrait écrire des bindings à la main pour l’API – d’où l’intérêt de se conformer à l’outil TypeScript officiel.

\subsection{D3.js pour le rendu visuel}

D3.js (Data-Driven Documents) est une bibliothèque JavaScript largement utilisée pour manipuler le DOM en fonction des données, en particulier via SVG, pour créer des visualisations personnalisées. Microsoft elle-même a misé sur D3.js dès l’origine des custom visuals : « en s’appuyant sur des librairies open-source comme D3.js, nous avons rendu la création de nouveaux visuels incroyablement simple »\footnote{\url{https://powerbi.microsoft.com/fr-fr/blog/announcing-the-power-bi-custom-visuals-contest/}}.

La majorité des visuels personnalisés (et même plusieurs visuels natifs sous le capot) utilisent D3 pour générer les éléments graphiques (formes, échelles, axes...).

D3 offre un contrôle très fin sur le rendu : on peut construire quasiment n’importe quel type de visualisation en partant de primitives de base (lignes, cercles, chemins SVG) liées aux données par des bindings data$\rightarrow$DOM. Par exemple, pour un graphique en barres custom, D3 facilite la création des \texttt{<rect>} dimensionnés selon les valeurs des données et applique des échelles et axes avec quelques fonctions.

Un avantage clé est que D3 est agnostique : on peut l’utiliser dans le sandbox sans souci, et il n’a pas de dépendances externes. En fait, Power BI intègre une prise en charge particulière de D3 : il a été noté que « Power BI attache D3 au contexte global de la sandbox, ce qui le rend immédiatement disponible »\footnote{\url{https://community.fabric.microsoft.com/t5/Developer/React-js-not-working-properly-in-custom-visual/td-p/3111857}}.

Cela signifie que D3 s’exécute aisément dans le cadre isolé, alors que d’autres librairies nécessitent parfois des ajustements (voir paragraphe sur React). D3 est donc hautement recommandé pour quiconque crée un visuel custom du fait de sa puissance et de sa souplesse.

Bien sûr, son utilisation requiert une certaine expertise (ce n’est pas une librairie de chart “clé en main” mais un outil bas niveau). Alternativement, certains développeurs optent pour des librairies plus haut niveau (par ex. Chart.js, Plotly.js, ou Vega-Lite).

Celles-ci peuvent être utilisées via le SDK, mais elles offrent moins de contrôle et parfois posent des problèmes de taille de bundle ou de compatibilité. D3 reste le standard de facto, au point que même des solutions no-code comme Deneb (un visuel custom permettant de configurer des graphiques via Vega-Lite) utilisent D3 en interne pour le rendu.

\subsection{React (optionnel) pour la structure UI}

L’utilisation de React – ou d’un autre framework comme Angular, Vue – dans un visuel Power BI est possible, bien que non indispensable dans tous les cas. Un visuel Power BI est par nature assez ciblé (il rend une visualisation spécifique dans un espace contraint).

Beaucoup de visuels custom n’emploient donc pas de framework UI sophistiqué : le code TypeScript avec D3 suffit à créer le SVG ou canvas voulu.

Néanmoins, React peut s’avérer utile dans plusieurs scénarios :
\begin{itemize}
  \item (a) Visuels complexes avec état interne – par exemple un visuel qui propose une petite interface utilisateur (boutons, menus) en plus du graphique. React aide à organiser le code en composants et à gérer les interactions utilisateur de manière déclarative.
  \item (b) Réutilisation de composants existants – un développeur peut intégrer des librairies React de composants (par ex. un sélecteur de couleur, une timeline interactive) au lieu de tout coder from scratch.
  \item (c) Faciliter le développement – pour des développeurs déjà familiers avec React, il peut être plus confortable de raisonner en composant React et virtual DOM qu’en manipulations directes du DOM.
\end{itemize}

Microsoft a d’ailleurs publié un tutoriel officiel montrant pas à pas la création d’un visuel custom en React (un “circle card” affichant un KPI formaté)\footnote{\url{https://learn.microsoft.com/en-us/power-bi/developer/visuals/custom-visual-develop-react-circle-card}}.

Cela dit, intégrer React dans un visuel custom introduit de la complexité. D’une part, cela alourdit le package car il faut inclure React (et possiblement ReactDOM) dans le bundle du visuel – sauf si l’on s’appuie sur la version déjà incluse par Power BI (non garantie dans le sandbox).

D’autre part, du fait du sandboxing, le \texttt{window} global utilisé par React n’est pas le même contexte que celui du host Power BI. Des développeurs ont constaté qu’il fallait ruser pour faire fonctionner React, par exemple en sauvegardant une référence du \texttt{window} isolé pour l’utiliser lors du rendu, car « lorsque React est chargé, il est attaché à un objet \texttt{window} qui est ensuite modifié par le sandbox, le rendant indisponible lors de l’appel \texttt{update} »\footnote{\url{https://community.fabric.microsoft.com/t5/Developer/React-js-not-working-properly-in-custom-visual/td-p/3111857}}.

Un ingénieur Microsoft a expliqué à ce sujet que « Power BI crée une copie de l’objet \texttt{window} pour isoler les librairies du visuel de celles de l’application ; D3 est relié correctement au contexte global, mais React ne l’était pas, nécessitant ce contournement »\footnote{\url{https://community.fabric.microsoft.com/t5/Developer/React-js-not-working-properly-in-custom-visual/td-p/3111857}}.

Il a suggéré qu’à terme l’isolation pourrait être assouplie. Il s’agit de détails techniques, mais qui illustrent que toute librairie externe doit être testée pour fonctionner dans l’environnement particulier de Power BI.

\subsection{Autres choix et écosystème}

À noter que TypeScript/D3/React ne sont pas les seules technologies possibles, mais représentent un socle éprouvé. Certains visuels spécifiques intègrent par exemple Mapbox (pour la cartographie), Three.js (pour de la 3D), ou d’autres libs selon le besoin – ce qui est faisable tant que cela respecte les contraintes du sandbox et du bundle.

Par ailleurs, pour les utilisateurs non développeurs qui veulent tout de même des visuels sur mesure, des outils comme Charticulator (de Microsoft Research) ou Deneb (basé sur Vega-Lite) permettent de créer des visuels custom sans écrire du code, en restant dans les limites du sandbox et avec un résultat intégrable comme un visuel custom.

Ces alternatives confirment qu’au cœur, le moteur de rendu reste du JS/TS manipulant le DOM SVG/HTML.

\subsection{Choix technologique pour ECRINS SA}

Dans le contexte du projet de bachelor « Création de composants BI custom dans Power BI » pour ECRINS SA, le choix de TypeScript et D3.js apparaît donc incontournable pour développer de nouveaux visuels performants et maintenables.

TypeScript assurera la robustesse du code et la compatibilité avec l’API Power BI, tandis que D3.js fournira la boîte à outils nécessaire pour représenter visuellement les données de la manière la plus flexible possible.

L’adoption de React dépendra de la complexité des visuels à créer : si les cas d’usage exigent des interactions UI poussées ou si l’équipe de développement maîtrise déjà React, il pourrait être judicieux de l’utiliser pour structurer l’application visuelle. Sinon, on pourra s’en passer afin de minimiser la complexité.

Quoiqu’il en soit, le respect des bonnes pratiques de sécurité (pas de dépendances externes non contrôlées, respect du sandbox) et d’optimisation (volume de données raisonnable, rendu efficace) guidera le développement. En s’appuyant sur ce trio technologique moderne, l’entreprise disposera d’une base solide pour étendre les capacités de Power BI avec des visuels sur mesure adaptés à ses besoins, tout en s’alignant sur l’état de l’art actuel en matière de visualisation de données interactive.
