%-----------------------------------------------------------
\section{Visuels Python / R : usages, atouts, limites}
\label{sec:python-r-visuals}
%-----------------------------------------------------------

Outre les visuels préfabriqués, Power BI autorise l’exécution de scripts
Python ou R pour produire des visuels sur mesure.  
L’utilisateur place un composant \enquote{Python visual} (ou \enquote{R visual})
dans le rapport, saisit son code dans l’éditeur, puis Power BI exécute ce
script en tâche de fond : les données liées sont transmises sous forme de
\textit{dataframe} et le résultat retourné est une image statique (PNG)
affichée dans le canevas.

Cette fonctionnalité exploite l’écosystème analytique des deux langages :
bibliothèques \textit{Matplotlib}, \textit{Seaborn} ou \textit{Plotly} côté
Python ; \textit{ggplot2} ou \textit{plotly R} côté R.  
Un data-scientist peut ainsi tracer dans Power BI un nuage de points avec
régression \textsc{Loess} (R) ou un diagramme de réseau (Python) en quelques
lignes, visuels impossibles à obtenir via les graphes natifs.


\textbf{Atouts.}  
La puissance réside dans la bibliothèque de packages open-source :
statistiques avancées, machine learning, heatmaps, dendrogrammes, etc.
Les scripts peuvent en outre pré-traiter les données (agrégation, calcul
d’indicateurs, entraînement d’un modèle) avant de dessiner le graphique ;
l’opération se relance automatiquement lorsque le visuel est rafraîchi,
offrant au passage des capacités qu’un simple DAX ne couvre pas (analyse de
texte, séries temporelles complexes).


\textbf{Limites techniques et fonctionnelles.}  
Le rendu reste une image statique : \enquote{les visualisations Python dans
Power BI sont des bitmaps 72 DPI, sans interactivité directe}
\parencite{MicrosoftPythonRVisualsDocs2024}.  
L’utilisateur ne peut donc pas cliquer dans le graphique pour filtrer les
autres visuels ; seul un nouveau filtrage externe provoque la ré-exécution
du script, qui demeure plus lente qu’un visuel natif.

Power BI encadre par ailleurs la quantité de données :  
\emph{(i)} 150 000 lignes et 100 colonnes au maximum sont transmises au
script ; au-delà, seules les premières 150 000 lignes sont prises en compte,
un avertissement s’affichant sur l’image ;  
\emph{(ii)} la taille totale ne doit pas excéder 250 MB en mémoire ;  
\emph{(iii)} un \textit{timeout} de cinq minutes interrompt tout script trop
long\parencite{MicrosoftPythonRVisualsDocs2024}\emph{, mais ce délai est ramené à
une minute lorsque l’exécution se fait dans le service Power BI}
\parencite{MicrosoftRPackagesService2025} ;
\emph{(iv)} pour les \textit{R visuals}, la taille du PNG de sortie est
désormais plafonnée à 2 MB\parencite{MicrosoftRVisualsDocs2025}.   
Ces garde-fous rendent les visuels Python/R adaptés à des échantillons ou à
des agrégats, mais inadaptés à un traitement massif ou à un apprentissage
profond.

D’autres restrictions s’appliquent : les chaînes de plus de 32 766 caractères
sont tronquées ; les visuels ne s’affichent ni dans \textit{Publish to Web}
ni dans certains environnements mobiles pour raisons de sécurité ;
\textbf{le runtime R 4.3.3 dans le service limite en outre le payload
compressé à 30 MB}\parencite{MicrosoftRPackagesService2025} ; enfin,
l’interpréteur Python ou R et les packages requis doivent être installés
localement pour Power BI Desktop, tandis que le service en ligne n’exécute
les scripts que sur un workspace Premium ou pour des comptes Pro,  
\textbf;{ce qui signifie qu’un compte Free n’affichera pas ces visuels}
\parencite{MicrosoftPythonRVisualsDocs2024}. % ← précision licence


\textbf{Sécurité et maintenance.}  
Lors de l’insertion d’un premier visuel Python / R, Power BI émet un
avertissement invitant à exécuter uniquement un code de source fiable
\parencite{MicrosoftPythonRVisualsDocs2024}.  
Dans un contexte d’entreprise, la gouvernance du code devient critique :
le script est embarqué dans le fichier \texttt{.pbix}, hors de tout
versionnage Git natif ; il faut donc organiser une validation et un cycle de
mise à jour des dépendances.


\textbf{Bilan.}  
Les visuels Python et R constituent une solution d’appoint puissante pour
des analyses ad hoc ou spécialisées, en comblant certaines lacunes des
visuels natifs grâce à l’arsenal \textit{data-science}.  
Ils doivent toutefois être utilisés en connaissance de leurs limites :
image statique, performance conditionnée par les seuils de lignes et
colonne, exigences Premium dans le service, et effort de maintenance du
code.  
Dès qu’un visuel doit être interactif, largement partagé ou intégré au
catalogue interne, le développement d’un visuel personnalisé via le SDK —
thème de la section \ref{sec:sdk} — devient souvent la voie la plus pérenne.
