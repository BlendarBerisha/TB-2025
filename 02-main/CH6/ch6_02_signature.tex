% -----------------------------------------------------------------
% CH6_02_signature.tex — Section 6.2 : Signature numérique & gouvernance
% -----------------------------------------------------------------

\section{Signature numérique \& gouvernance}
\label{sec:signature}

La mise en production de visuels personnalisés pose des enjeux spécifiques de sécurité et de gouvernance : contrairement aux visuels natifs, ils embarquent du code exécutable qui doit rester maîtrisé. La signature numérique et une validation interne structurée s’imposent donc pour garantir l’authenticité et l’intégrité des paquets .pbiviz.

\subsection{Chaîne de confiance et contrôles}
Chaque visuel destiné à la production est signé avec un certificat X.509 appartenant à ECRINS~SA (ou à une autorité reconnue dans le cas d’AppSource). Toute altération invalide la signature et peut être bloquée côté plateforme si le locataire restreint l’usage aux visuels certifiés/organisationnels. Dans notre dispositif, la signature intervient au packaging, via un certificat stocké de manière sécurisée (coffre de secrets) et orchestrée par la CI. Les contrôles automatisés exécutés en pipeline (\S\ref{sec:ci-cd}) combinent audit de certification, analyses statiques et règles ESLint renforcées afin de détecter appels réseau non autorisés, exécution dynamique de code, dépendances obsolètes ou motifs risqués. Une revue manuelle peut compléter le dispositif pour les composants sensibles.

\subsection{Rôles et responsabilités}

\begin{description}[leftmargin=0pt, labelsep=0.5em, itemsep=.25\baselineskip]
  \item[Développeur du visuel] Garantit la conformité aux règles de développement sécurisé et la qualité du code (tests, conventions, commentaires).
  \item[Responsable CI/CD] Conçoit la pipeline, gère certificats et secrets, et assure la reproductibilité des builds.
  \item[Sécurité / Administrateur Power BI] Examine les rapports d’audit, vérifie la non-exfiltration de données et valide en environnement isolé.
  \item[Administrateur du tenant Power BI] Publie dans le magasin organisationnel et applique les politiques d’usage.
  \item[Responsable produit / métier] Valide l’adéquation fonctionnelle et donne l’aval avant diffusion.
\end{description}

\subsection{Politique de distribution sécurisée}
Chaque package signé est diffusé par un canal maîtrisé (magasin organisationnel ou espace interne sécurisé). Les envois par e-mail ou supports amovibles sont proscrits. Le versionnement suit un schéma X.Y.Z, accompagné d’un changelog et de la date de déploiement afin de tracer précisément quelle version est utilisée et où.
