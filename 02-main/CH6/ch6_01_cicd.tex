% -----------------------------------------------------------------
% CH6_01_cicd.tex — Section 6.1 : Chaîne CI/CD pour les deux visuels
% -----------------------------------------------------------------

\section{Chaîne CI/CD pour les deux visuels}
\label{sec:ci-cd}

Pour assurer une livraison fiable et reproductible des deux visuels, une chaîne d’intégration et de déploiement continus (CI/CD) a été mise en place avec GitHub Actions. Le pipeline automatise l’ensemble du cycle de vie logiciel, depuis l’installation des dépendances jusqu’au packaging .pbiviz, en intégrant linting, tests et audit de conformité. Il se déclenche à chaque push ou pull request sur la branche principale et peut être exécuté à la demande.

L’objectif est double : garantir la reproductibilité des builds et détecter tôt les régressions. Dans le contexte d’ECRINS SA, où l’équipe BI souhaite industrialiser des visuels Power BI personnalisés, cette automatisation fournit un socle de qualité, de traçabilité et de maintenabilité.

\begin{lstlisting}[language=yaml,
  caption={Pipeline GitHub Actions — build, tests, packaging .pbiviz},
  label={lst:ci-cd-pipeline},
  breaklines=true, breakatwhitespace=true, columns=fullflexible, keepspaces=true, basicstyle=\ttfamily\footnotesize]
name: Build-and-Package-Custom-Visuals

on:
  push:
    branches: [ main ]
  pull_request:
    branches: [ main ]
  workflow_dispatch:

jobs:
  build:
    runs-on: ubuntu-latest
    strategy:
      fail-fast: false
      matrix:
        visual: [ sunburst, passenger-flow ]
    defaults:
      run:
        working-directory: visuals/${{ matrix.visual }}
    steps:
      - name: Checkout repository
        uses: actions/checkout@v4
        with:
          fetch-depth: 0

      - name: Setup Node.js
        uses: actions/setup-node@v4
        with:
          node-version: '18'
          cache: 'npm'

      - name: Install dependencies
        run: npm ci

      - name: Lint
        run: npm run lint

      - name: Unit tests + coverage gate (>= 70%)
        shell: bash
        run: |
          npm test -- --ci --coverage --passWithNoTests
          COVER=$(node -e "console.log(require('./coverage/coverage-summary.json').total.statements.pct)")
          echo "Coverage: $COVER%"
          awk 'BEGIN{exit !( '"$COVER"' >= 70 )}'

      - name: Build
        run: npm run build

      - name: Package (.pbiviz) with audit
        run: npx pbiviz package --certification-audit -o dist

      - name: Enforce debug bundle size (<= 1 MiB)
        shell: bash
        run: |
          BYTES=$(stat -c%s dist/*.pbiviz | sort -nr | head -n1)
          echo "Bundle size: $BYTES bytes"
          [ "$BYTES" -le 1048576 ]

      - name: Upload artifact
        uses: actions/upload-artifact@v4
        with:
          name: ${{ matrix.visual }}-pbiviz
          path: dist/*.pbiviz
\end{lstlisting}

Concrètement, le workflow récupère le code du dépôt, prépare un environnement Node.js LTS 18 et installe les dépendances de manière déterministe (npm ci). Les contrôles de qualité (linting et tests) s’exécutent à chaque commit ; un garde-fou de couverture de 70~\% bloque la progression si la qualité baisse. Le packaging déclenche l’audit de certification fourni par l’outillage Custom Visuals pour repérer les usages interdits (réseau non autorisé, évaluation dynamique, etc.), puis applique un contrôle de taille (\(\leq\) 1~MiB en debug). Enfin, les artefacts .pbiviz sont publiés par visuel, ce qui facilite leur diffusion interne et la traçabilité. La signature et la gouvernance de publication (magasin organisationnel) sont traitées aux sections suivantes, afin de conserver une chaîne de build simple et fiable.
