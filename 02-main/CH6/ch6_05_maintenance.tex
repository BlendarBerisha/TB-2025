% -----------------------------------------------------------------
% CH6_05_maintenance.tex — Section 6.5 : Maintenance & transfert de connaissances
% -----------------------------------------------------------------

\section{Maintenance \& transfert de connaissances}
\label{sec:maintenance}

La pérennité des visuels repose d’abord sur un code commenté et structuré : les portions critiques (par exemple l’algorithme A* du Passenger-Flow Map) sont explicitées, les conventions de nommage uniformes et les structures du DataView documentées. Cette clarté réduit l’effet «~boîte noire~» et accélère la reprise.

L’architecture modulaire isole les responsabilités (conversion des données, modèle de vue, rendu, interactions), ce qui rend les composants plus testables et plus extensibles. La documentation technique associe un schéma annoté des ports \& adaptateurs et une description concise des modules.

Le playbook de développement (chapitre~\ref{chap:playbook}) constitue la référence unique pour créer, tester et packager un visuel : mise en place de l’environnement (pbiviz, Node.js, certificats), enchaînement des commandes (pbiviz new \(\rightarrow\) pbiviz package), bonnes pratiques de performance, d’accessibilité et d’UX, ainsi que les pièges courants.

La gestion de version s’appuie sur Git (branche main stable, branches feature/ et bugfix/), avec revue croisée systématique via pull request et tags alignés sur les versions publiées. Cette routine élève la qualité du code et diffuse la connaissance au sein de l’équipe. Des sessions de pair programming sur les sujets sensibles (optimisation D3, ISelectionManager, capabilities.json) complètent l’habilitation croisée.

Enfin, quelques vigilances orientent la maintenance continue : suivre l’évolution du SDK Power~BI et tester chaque version majeure ; valider la performance sur des volumétries représentatives et surveiller l’absence de fuites mémoire ; maintenir à jour les dépendances (D3.js, etc.) avec une veille automatisée (npm audit, Dependabot) et des installations reproductibles (npm ci) ; anticiper la généricité des visuels (fond paramétrable pour d’autres plans, validation de lisibilité au-delà de trois niveaux hiérarchiques pour le Sunburst). À terme, la factorisation de composants communs en bibliothèque interne (légende, panneau de contrôle, utilitaires d’accessibilité) renforcera encore la maintenabilité.
