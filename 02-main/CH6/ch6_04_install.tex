% -----------------------------------------------------------------
% CH6_04_install.tex — Section 6.4 : Guide d’installation & changelog
% -----------------------------------------------------------------
\section{Guide d’installation \& changelog}
\label{sec:install-changelog}

L’installation des visuels s’effectue via le magasin organisationnel ou, à défaut, par import manuel d’un fichier .pbiviz. Afin d’éviter les redites, le pas-à-pas détaillé (chemin de menus, avertissements éventuels, compatibilité) est centralisé en annexe~A5 (\autoref{ann:a6-install}). La présente section en précise seulement les principes et la logique de mise à jour.

Les mises à jour sont gérées de façon transparente via le magasin organisationnel : si le GUID reste identique, les rapports embarquant le visuel adoptent automatiquement la nouvelle version. En dehors du store, l’utilisateur réimporte explicitement le fichier .pbiviz. Lorsque des livraisons externes sont nécessaires (client), le transfert s’effectue via un espace sécurisé (SharePoint/OneDrive d’entreprise) avec vérification d’empreinte ou de signature.

Chaque visuel est suivi par un changelog versionné (dépôt GitHub ou wiki interne) qui mentionne le numéro de version (X.Y.Z), la date et la liste des modifications, ainsi que l’impact éventuel sur la compatibilité. Cette traçabilité facilite la communication auprès des utilisateurs et la planification des mises à jour.
