%-----------------------------------------------------------
\section{Solutions concurrentes (Tableau, Qlik, Looker)}
\label{sec:concurrence}
%-----------------------------------------------------------

Les principales plateformes BI concurrentes — Tableau, Qlik Sense et
Looker — proposent chacune des mécanismes de personnalisation de visuels
différents de ceux de Power BI, tant sur le plan technique que dans leurs
implications métier. Examiner ces modèles permet de situer les « visuels
custom » Power BI dans un paysage concurrentiel plus large.

\subsection{Tableau.}  
Tableau étend ses capacités à l’aide des Tableau Extensions, des
applications Web encapsulées dans le tableau de bord par l’Extension API
\parencite{TableauExtGuide2024}. Le composant, écrit en JavaScript et
logé dans une iframe, peut introduire un graphique inédit, du
write-back ou l’intégration d’un service tiers
\parencite{TableauBlogExt2024}. Depuis la version 2019.4, les extensions
s’exécutent par défaut en sandbox : aucun accès réseau n’est permis
sans liste blanche explicite dans l’interface d’administration
\parencite{TableauAdmin2025}. Tableau délègue le support de ces briques à
leurs éditeurs, si bien que chaque entreprise doit auditer la fiabilité de
la source avant déploiement. Sur le plan financier, le coût d’entrée reste
sensiblement supérieur à celui de Power BI : l’abonnement Creator est passé
à 75 USD par utilisateur et par mois en juillet 2025
\parencite{TableauPricing2025}. La personnalisation n’est donc rentable que
si l’organisation dispose déjà d’une base installée Tableau ou d’un budget
élargi.

\subsection{Qlik Sense.}  
Qlik propose les Visualization Extensions : objets écrits
en HTML/JavaScript/CSS et déclarés par un manifeste .qext
\parencite{QlikDevHub2024}. L’extension, quand elle respecte l’API Qlik,
s’intègre comme un objet natif, profite du moteur associatif et réagit aux
sélections \parencite{QlikExtAPI2024}. Qlik mise sur une dynamique
communautaire : de nombreux composants sont partagés via Qlik Branch
sans validation officielle, la gouvernance incombant à l’administrateur
qui doit installer manuellement l’archive sur le tenant SaaS.
Commercialement, la déclinaison « Qlik Cloud Starter » débute à
200 USD/mois pour dix utilisateurs, la formule Standard à 825 USD et la
Premium à 2 750 USD, chaque palier incluant un volume de données et un
nombre maximal de créateurs \parencite{QlikPricing2025}. Le modèle reste
avantageux quand l’entreprise possède une équipe front-end JavaScript
capable de maintenir ces extensions.

\subsection{Looker.}  
Looker, désormais composante de Google Cloud, autorise des visuels
spécifiques via le Looker Custom Visualization SDK. Le développeur
implémente la fonction updateAsync qui reçoit les données
d’une explore Looker et restitue un rendu SVG ou Canvas
\parencite{LookerVizSDK2025}. Pour diffusion publique, Google impose une
revue Marketplace : le code doit être hébergé sur GitHub et passer un
contrôle de conformité avant publication
\parencite{LookerMarketplace2024}. Il reste toutefois possible de limiter
l’usage à un tenant interne via l’interface Admin. La logique de Looker
demeure plus centralisée : la personnalisation est permise, mais soumise à
un contrôle étroit et, surtout, à une tarification négociée au cas par cas
qui dépasse largement les paliers Tableau ou Qlik ; les clients visent
donc des gains élevés sur un périmètre de visuels custom restreint.

\subsection{Lecture comparative.}  
Tableau et Qlik accordent à l’utilisateur avancé la liberté de charger
localement une extension, tout en transférant la gouvernance de sécurité à
l’entreprise ; Power BI et Looker exigent au contraire une validation
centralisée — certification AppSource ou revue Marketplace — avant tout
déploiement massif. Techniquement, les trois concurrents s’appuient, comme
Power BI, sur le triptyque HTML/JS/CSS, mais Power BI se démarque par un
SDK TypeScript/D3 structuré, une CLI d’audit intégrée et un vivier
AppSource de presque 600 visuels au 1\textsuperscript{er}
août 2025 \parencite{AppSourceCount2025}. Sur le plan économique, Power BI
reste l’option la plus abordable pour industrialiser des visuels custom à
grande échelle, tandis que Tableau, Qlik et surtout Looker réservent la
personnalisation intensive aux environnements dont le budget justifie
l’investissement initial et la maintenance continue.
