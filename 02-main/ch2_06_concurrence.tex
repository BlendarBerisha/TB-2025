%-----------------------------------------------------------
\section{Solutions concurrentes (Tableau, Qlik, Looker)}
\label{sec:concurrence}
%-----------------------------------------------------------

Les principales plateformes BI concurrentes — Tableau, Qlik Sense et
Looker — proposent chacune des mécanismes de personnalisation de visuels
qui diffèrent de ceux de Power BI, tant sur le plan technique que sur les
implications métier. Comprendre ces approches permet de situer les
\textit{visuels custom} Power BI dans un paysage concurrentiel plus large.

\textbf{Tableau.}
Tableau étend ses capacités via les \emph{Tableau Extensions}, des
applications Web encapsulées dans les tableaux de bord grâce à l’Extension
API\parencite{TableauExtGuide2024}. Un développeur crée, en JavaScript,
un composant (souvent dans une \texttt{iframe}) qui ajoute un graphique
personnalisé ou une interaction spécifique, par exemple du \emph{write-back}
ou l’intégration d’un service tiers\parencite{TableauBlogExt2024}. Depuis
la v2019.4, les extensions sont exécutées par défaut en \emph{sandbox},
sans accès réseau, et toute extension nécessitant un accès externe doit
être explicitement inscrite sur liste blanche par l’administrateur serveur
ou site\parencite{TableauAdmin2025}. Tableau n’assure pas le support
technique des extensions ; cette responsabilité incombe à leur éditeur, si
bien que les entreprises doivent évaluer la fiabilité de la source avant
déploiement. Sur le plan financier, la licence Creator avoisine
70 USD utilisateur/mois\parencite{TableauPricing2025}, ce qui fait de la
personnalisation une option coûteuse pour les organisations sensibles au
budget, à la différence d’un environnement Power BI où le coût d’entrée est
plus faible.

\textbf{Qlik Sense.}
L’extensibilité de Qlik repose sur les \emph{Visualization Extensions} :
objets personnalisés écrits en HTML 5 / JavaScript / CSS et déclarés par un
manifeste \texttt{.qext}\parencite{QlikDevHub2024}. Une extension Qlik,
lorsqu’elle respecte l’API Qlik, s’intègre comme un objet natif : on la
fait glisser dans la feuille, on ajuste ses propriétés, et elle réagit
aux sélections grâce au moteur associatif\parencite{QlikExtAPI2024}. Le
développeur peut embarquer n’importe quelle bibliothèque (D3, three.js,
etc.) pour façonner un rendu spécifique. Qlik privilégie ici une approche
communautaire : de nombreuses extensions sont partagées via
\emph{Qlik Branch} sans validation officielle. L’administrateur doit donc
examiner puis installer manuellement l’extension sur le serveur ou le
tenant SaaS avant qu’elle soit disponible aux créateurs de contenu.
Commercialement, Qlik Sense Business se situe autour de 30 USD par
utilisateur/mois, et l’édition Enterprise encore au-dessus
\parencite{QlikPricing2025}. Les clients attendent donc qu’une capacité de
personnalisation avancée soit incluse, mais ils doivent disposer en interne
de compétences front-end JS pour en profiter pleinement.

\textbf{Looker.}
Looker (aujourd’hui composant de Google Cloud) autorise des visuels
personnalisés via le \emph{Looker Custom Visualization SDK}. Le
développeur code en JavaScript un composant qui implémente la fonction
\texttt{updateAsync} afin de recevoir les données d’une
\emph{explore} Looker et de restituer un rendu SVG ou Canvas
\parencite{LookerVizSDK2025}. Pour diffusion publique, Google impose un
processus Marketplace : le code doit être hébergé sur GitHub et passer une
revue de conformité avant publication\parencite{LookerMarketplace2024}. Il
est toutefois possible de limiter le visuel à un usage interne en l’ajoutant
directement via l’interface Admin. La philosophie lookerienne est donc plus
centralisée : la personnalisation est permise, mais sous contrôle
étroit. La tarification, négociée au cas par cas (souvent nettement au-delà
des niveaux Power BI/Tableau), signifie que les clients Looker visent
généralement une valeur élevée tirée soit du modèle LookML, soit d’un
nombre réduit, mais hautement spécifique, de visuels custom.

\textbf{Lecture comparative.}
Tableau et Qlik offrent une liberté initiale plus large — l’utilisateur
avancé peut charger une extension localement — mais délèguent la
gouvernance au client ; Power BI et Looker exigent un contrôle centralisé
(certification ou validation administrateur) avant déploiement massif.
Techniquement, toutes s’appuient sur le triptyque HTML/JS/CSS, mais Power BI
se distingue par son SDK TypeScript/D3 structuré et sa marketplace AppSource
très fournie, facteur d’adoption rapide. Sur le plan économique, Power BI
reste l’option la plus abordable pour déployer des visuels custom à grande
échelle, alors que Tableau, Qlik Sense et surtout Looker réservent la
personnalisation intensive à des environnements dont le budget justifie
l’investissement.
