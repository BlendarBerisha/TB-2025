%-----------------------------------------------------------
\section{Organisation du travail — cycle \emph{Waterfall} en cinq phases}
\label{sec:organisation-waterfall}
%-----------------------------------------------------------

Le projet suit un cycle de développement structuré de type \emph{Waterfall}, organisé en cinq phases successives couvrant la période du 12 mai au 14 août 2025. Ce modèle en cascade, bien que linéaire, offre une clarté dans le déroulement des étapes, permet de fixer des objectifs intermédiaires précis et facilite la validation progressive des livrables, en cohérence avec le cadre académique.

\begin{table}[H]
  \centering
  \caption{Découpage du projet selon un cycle Waterfall}
  \label{tab:planning-waterfall}
  \begin{tabularx}{\textwidth}{@{}>{\bfseries}l c X@{}}
    \toprule
    Phase & Période & Objectif principal \\
    \midrule
    Analyse \& cadrage & 12 mai – 31 mai & Formaliser les besoins, les contraintes Power BI, et produire le cahier des charges fonctionnel. \\
    Conception détaillée & 1 juin – 14 juin & Définir l’architecture des visuels, produire les maquettes, planifier les tests et établir le dossier de conception. \\
    Implémentation & 15 juin – 19 juillet & Développer les deux visuels (Passenger-Flow Map, Sunburst) et les intégrer dans un rapport Power BI. \\
    Vérifications \& optimisation & 20 juillet – 4 août & Tester les composants (fonctionnalité, performance, accessibilité), corriger les anomalies et optimiser les performances. \\
    Livraison finale & 5 août – 14 août & Rédiger la documentation, finaliser le packaging, préparer le rapport et la soutenance. \\
    \bottomrule
  \end{tabularx}
\end{table}

La première phase, prévue entre le 12 et le 31 mai, est consacrée à l’analyse du besoin et au cadrage fonctionnel. Elle a pour but de formaliser les attentes des clients d’ECRINS~SA, d’identifier les limites des visuels existants et de poser les fondations du projet à travers un cahier des charges. La seconde phase, qui s’étend du 1\up{er} au 14 juin, consiste à traduire ces exigences en une conception détaillée : choix techniques, structure des données, maquettes visuelles, et architecture du composant. Ces éléments constitueront le dossier de conception, servant de référence pour le développement.

La phase suivante, qui s’étale du 15 juin au 19 juillet, est dédiée à l’implémentation proprement dite des deux visuels (Passenger-Flow Map et Sunburst hiérarchique). Durant cette période, les premières versions fonctionnelles seront produites et intégrées dans un rapport Power BI de démonstration. Viendra ensuite la phase de tests et d’optimisation, entre le 20 juillet et le 4 août, qui vise à valider la conformité des composants par rapport aux critères de performance, d’accessibilité et d’intégration attendus. Les tests automatisés et les ajustements finaux permettront de consolider la qualité des livrables. Enfin, la période du 5 au 14 août est réservée à la livraison finale : rédaction de la documentation utilisateur et technique, préparation du guide d’intégration et formalisation du rapport académique.

Le pilotage du projet s’appuie sur un mode de gestion individuel allégé. Aucun outil de gestion de projet formel n’est imposé, mais un ensemble cohérent de pratiques permet d’en assurer le suivi rigoureux. Un journal de bord personnel consigne les tâches réalisées, les choix techniques, les obstacles rencontrés et les décisions prises. Ce journal, versionné via Git, constitue également une base factuelle pour le chapitre d’évaluation.

L’avancement est visualisé à l’aide d’un tableau de type \emph{Kanban} dans GitHub Projects, organisé en trois colonnes (\emph{À faire}, \emph{En cours}, \emph{Terminé}). Cette méthode offre une vue claire de la charge de travail et permet de prioriser les actions chaque semaine. Par ailleurs, des réunions de suivi hebdomadaires sont organisées avec le professeur référent. Elles servent à valider l’état d’avancement, à ajuster les priorités si nécessaire et à arbitrer d’éventuelles décisions techniques.

La qualité du livrable est suivie à trois niveaux. D’abord, la couverture des tests unitaires est mesurée tout au long du développement, avec un objectif minimal de 75\,\%. Ensuite, l’accessibilité est vérifiée selon les recommandations WCAG~2.2 à l’aide d’outils automatisés tels que \texttt{axe-core}. Enfin, un audit de conformité est réalisé en amont de chaque livraison majeure à l’aide de la commande \texttt{pbiviz package --certification-audit}, qui garantit le respect des exigences de sécurité et d’isolation imposées par Power BI.

Plusieurs risques ont été identifiés en amont. Une dérive du planning est anticipée par l’ajout de marges de sécurité entre certaines phases, notamment entre l’implémentation et la phase de test. Les blocages techniques liés au SDK Power BI feront l’objet d’un traitement rapide par l’analyse de la documentation officielle et le recours aux communautés techniques ; à défaut, des simplifications fonctionnelles seront envisagées. Enfin, si les jeux de données venaient à évoluer au cours du projet, un format pivot standardisé (fichier CSV structuré + dictionnaire de schéma) a été défini pour garantir la compatibilité avec le code développé.

En résumé, l’organisation du travail repose sur un équilibre entre structure formelle (phases, jalons, livrables) et agilité opérationnelle (journal de bord, révisions hebdomadaires, tableaux de tâches). Cette méthode garantit la qualité du livrable final tout en respectant les contraintes de temps et les exigences pédagogiques du travail de Bachelor.
