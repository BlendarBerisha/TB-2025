%-----------------------------------------------------------
\section{Sources de validation (revue experte)}
\label{sec:validation-sources}
%-----------------------------------------------------------

Dans la mesure où ce travail constitue avant tout une \textit{proof of concept}, 
l’objectif est de démontrer la faisabilité technique et la pertinence métier des visuels, 
plutôt que de produire une évaluation statistiquement généralisable. 
Selon Mohagheghi et Dehlen, un PoC vise essentiellement à réduire l’incertitude d’un projet 
en vérifiant, sur un périmètre réduit, que la solution envisagée répond aux attentes clés 
\parencite{MohagheghiPoC2008}. 
La démarche de validation retenue est donc qualitative et pragmatique, 
reposant sur une unique \textbf{revue experte fonctionnelle}.

\subsection{Revue experte fonctionnelle.}
La professeure référente, également active au sein d’ECRINS\,SA, a évalué 
les visuels sur la base d’un jeu de scénarios représentatifs : filtrage horaire 
et carte de chaleur pour la carte de flux, navigation \textit{drill-down} 
et fil d’Ariane pour le Sunburst, cohérence des libellés et des agrégations. 
Cette revue a donné lieu à des commentaires formalisés, consignés dans le journal de bord, 
et intégrés dans la synthèse des retours du Chapitre~\ref{sec:validation-resultats}. 

\subsection{Écart par rapport au plan initial.}
Le protocole initial prévoyait également une démonstration interne 
par un collaborateur d’ECRINS\,SA ainsi qu’un mini-test utilisateur exploratoire. 
Ces étapes n’ont pas pu être réalisées pour des raisons de disponibilité, 
ce qui limite la diversité des perspectives recueillies. 
La validation se limite donc à un avis expert unique, 
ce qui reste cohérent avec le périmètre \textit{proof of concept} 
et permet d’identifier des pistes d’amélioration 
avant toute évaluation élargie.

\subsection{Limites assumées.}
Cette revue experte ne fournit pas de preuve quantitative et ne permet pas 
de généraliser les résultats à l’ensemble des utilisateurs potentiels. 
Elle répond néanmoins aux objectifs d’un PoC : confirmer la faisabilité, 
vérifier l’utilité métier et recueillir des pistes d’amélioration 
avant toute industrialisation future.
