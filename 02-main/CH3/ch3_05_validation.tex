%-----------------------------------------------------------
\section{Sources de validation (revue experte et mini-test utilisateur)}
\label{sec:validation-sources}
%-----------------------------------------------------------

Dans la mesure où ce travail constitue avant tout une \textit{proof of
concept}, l’objectif est de démontrer la faisabilité technique et la pertinence
métier des visuels, plutôt que de produire une évaluation statistiquement
généralisable. Selon Mohagheghi et Dehlen, un PoC vise essentiellement à
réduire l’incertitude d’un projet en vérifiant, sur un périmètre réduit, que
la solution envisagée répond aux attentes clés\parencite{MohagheghiPoC2008}.
La démarche de validation retenue est donc qualitative et pragmatique : elle
combine une revue experte fonctionnelle et un mini-test utilisateur
exploratoire.

\paragraph{Revue experte fonctionnelle.}
La professeure référente, également active au sein d’ECRINS SA, évaluera les
visuels sur la base d’un jeu de scénarios représentatifs : filtrage
horaire et heat-map pour la carte de flux, navigation drill-down pour le
Sunburst, cohérence des libellés et des agrégations. Cette revue donnera lieu
à des commentaires formalisés, consignés dans le journal de bord, puis, le cas
échéant, à des ajustements fonctionnels avant la démonstration interne.

\paragraph{Démonstration interne.}
En fin de développement, une courte présentation sera organisée devant un
collaborateur d’ECRINS SA (chef de projet ou responsable métier). L’objectif
est de recueillir un retour immédiat sur l’\emph{adhérence métier} :
la carte met-elle bien en évidence les axes de transit attendus ?
Le Sunburst restitue-t-il correctement la hiérarchie analysée ?  
Ce retour oral, même informel, constitue une validation qualitative de la
pertinence pratique du prototype.

\paragraph{Mini-test utilisateur exploratoire.}
Enfin, un test utilisateur \emph{exploratoire} sera conduit avec un
utilisateur unique. Conformément aux recommandations de Lallemand et
Gronier, un tel protocole vise à observer les réactions et la
compréhension sans prétention de représentativité statistique
\parencite{LallemandUX2016}. Le participant sera invité à réaliser deux
actions : (1) identifier la zone la plus dense sur la carte de flux et
(2) interpréter la part d’une catégorie donnée dans le Sunburst. Les
observations et commentaires recueillis permettront de détecter d’éventuels
points d’incompréhension ou de friction.

\paragraph{Limites assumées.}
Cette combinaison revue-expert / démonstration / test exploratoire ne fournit
pas de preuve quantitative ; elle répond néanmoins aux exigences d’un PoC :
confirmer la faisabilité, vérifier l’utilité métier et recueillir des pistes
d’amélioration avant toute industrialisation future.
