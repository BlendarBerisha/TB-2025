%-----------------------------------------------------------
\section{Critères de réussite}
\label{sec:success-criteria}
%-----------------------------------------------------------

Les critères de réussite du prototype servent d’indicateurs de \textit{faisabilité
technique} et de \textit{rigueur méthodologique}, plutôt que d’objectifs
fonctionnels exhaustifs.
Ils visent à vérifier que la preuve de concept respecte un socle minimal en
matière de performance, d’ergonomie et de qualité.

\begin{enumerate}[noitemsep,topsep=0pt,label=\arabic*)]
    \item \textbf{Performance} : le temps de rendu et de mise à jour doit rester
          inférieur à 300~ms, garantissant une interaction fluide pour
          l’utilisateur.
    \item \textbf{Accessibilité (WCAG 2.2)} : le visuel doit se conformer aux
          recommandations WCAG 2.2 (contraste suffisant, navigation clavier,
          attributs \texttt{aria} pertinents) afin d’être utilisable par le
          plus grand nombre.
    \item \textbf{Couverture fonctionnelle} : toutes les fonctionnalités
          attendues (détails au survol, sélection et filtrage, zoom, etc.) sont
          implémentées, assurant la pertinence métier du prototype.
    \item \textbf{Intégration Power BI (cross-highlighting)} : le visuel prend
          en charge le filtrage et le surlignement croisés avec les autres
          visualisations du rapport, conformément au comportement natif de la
          plateforme.
    \item \textbf{Qualité du code} : le code source est modulaire, documenté et
          couvert par des tests unitaires (≥ 75 \%), facilitant la maintenance.
    \item \textbf{Sécurité (sandbox pbiviz)} : le prototype s’exécute dans
          l’\emph{iframe} sandbox sans appel réseau externe ni usage d’API non
          autorisées, passant l’audit
          \texttt{pbiviz package --certification-audit}.
    \item \textbf{Documentation} : un guide utilisateur et un guide technique
          détaillent l’installation, l’usage et l’architecture interne du
          visuel.
\end{enumerate}

\begin{table}[H]
    \centering
    \caption{Synthèse des critères de réussite}
    \label{tab:success-criteria}
    \begin{tabularx}{\textwidth}{@{}lX@{}}
        \toprule
        \textbf{Axe} & \textbf{Critère cible} \\
        \midrule
        Performance            & Temps de réponse du visuel < 300 ms \\
        Accessibilité          & Conformité complète aux WCAG 2.2 \\
        Couverture fonctionnelle & Fonctionnalités attendues implémentées \\
        Intégration Power BI   & Cross-highlighting pleinement opérationnel \\
        Qualité du code        & Code modulaire, tests ≥ 75 \%, commentaires \\
        Sécurité               & Exécution conforme aux règles du sandbox \\
        Documentation          & Guides utilisateur et développeur fournis \\
        \bottomrule
    \end{tabularx}
\end{table}

Chaque critère énuméré ci-dessus constituera un point de contrôle lors de
l’évaluation finale : leur satisfaction attestera de la faisabilité technique
et de la qualité du travail réalisé dans le cadre de ce
\textit{proof of concept}.
