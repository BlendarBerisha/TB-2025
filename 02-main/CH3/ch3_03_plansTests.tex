%-----------------------------------------------------------
\section{Plan de tests (datasets démo, scénarios tableau de bord)}
\label{sec:test-plan}
%-----------------------------------------------------------

La phase de tests s’appuie sur deux jeux de données de démonstration couvrant les cas d’usage clés. La définition exhaustive des champs, rôles Power BI et contraintes est centralisée en Annexe A1 (\autoref{ann:a1-dictionnaires}). Les artefacts du fond cartographique et des obstacles (schéma JSON abrégé et figure d’overlay) sont fournis en Annexe A2 (\autoref{ann:a2-fond}). La présente section se concentre sur les scénarios et critères de validation.

%-----------------------------------------------------------
\subsection{Carte de flux de passagers (Passenger-Flow Map)}
\label{subsec:test-flowmap}
%-----------------------------------------------------------

\textbf{Cas d’usage simulés.} Les essais portent sur le filtrage par tranches horaires, la distinction des catégories de flux (Départ/Arrivée) et l’activation d’un rendu de type carte de chaleur pour localiser les zones d’intensité. Les interactions croisées sont vérifiées : un filtrage temporel répercute immédiatement ses effets sur la carte des flux et inversement. Les info-bulles natives affichent les agrégations pertinentes conformément aux filtres actifs.

\textbf{Validation de la faisabilité fonctionnelle.} Les scénarios confirment la conformité du visuel aux comportements attendus de Power BI (filtrage, surbrillance croisée, info-bulles), ainsi que la bonne mise à jour des valeurs lorsque la plage horaire varie. L’activation de la carte de chaleur démontre la capacité à agréger spatialement les poids et à traduire l’intensité en densité visuelle.

%-----------------------------------------------------------
\subsection{Sunburst hiérarchique (Analyse multi-niveaux)}
\label{subsec:test-sunburst}
%-----------------------------------------------------------

\textbf{Cas d’usage simulés.} Les tests couvrent la sélection de niveaux, les opérations de drill-down/drill-up, la mise en évidence de branches et l’application de seuils sur la mesure. L’interaction croisée est contrôlée : sélectionner un segment filtre les autres visuels du rapport, et réciproquement.

\textbf{Validation de la faisabilité fonctionnelle.} Les essais montrent que la hiérarchie et ses agrégations sont correctement représentées, que la navigation entre niveaux conserve les proportions attendues et que les filtres (catégoriels ou numériques) modifient les segments de manière cohérente avec les règles de filtrage de Power BI.


