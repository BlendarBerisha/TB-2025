%-----------------------------------------------------------
\section{Origine du besoin}
\label{sec:need-origin}
%-----------------------------------------------------------

Dans le cadre de son activité de conseil, la société ECRINS~SA étudie la
\emph{faisabilité} de visuels Power BI dépassant les limites du catalogue
natif et des extensions AppSource.  
Deux scénarios métiers proposés par des clients—l’un marketing, l’autre
analytique—ont été retenus afin de démontrer, sous forme de
\textbf{preuve de concept (PoC)}, la viabilité technique de visuels
personnalisés susceptibles d’être industrialisés par la suite.

%-----------------------------------------------------------
\subsection{Besoin marketing : carte des flux passagers aéroportuaire}
\label{subsec:need-marketing-flow}
%-----------------------------------------------------------

Un client gestionnaire d’un aéroport international souhaite
\textbf{cartographier les flux de passagers sur l’ensemble du site}
(parkings, halls d’enregistrement, points de contrôle, boutiques,
portes d’embarquement) pour identifier les zones de plus forte
affluence et optimiser l’emplacement des surfaces publicitaires.  
Le visuel attendu adopte le paradigme de la \emph{flow map} : des arcs
reliant les points d’intérêt, dont l’épaisseur matérialise le volume de
voyageurs, révèlent immédiatement les axes de déplacement dominants
\parencite{GuoFlowMaps2011}.  
Une couche de \textit{heat map} complétera la lecture en soulignant les
« hot-spots » commerciaux.  
Les choix techniques (plan SVG, coordonnées, routage A*, calcul de densité)
seront détaillés au chapitre \ref{sec:implementation}.

%-----------------------------------------------------------
\subsection{Besoin d’analyse hiérarchique : Sunburst de décomposition}
\label{subsec:need-hierarchical-sunburst}
%-----------------------------------------------------------

Un autre client d’ECRINS~SA exige une visualisation capable
d’explorer \textit{n’importe quelle hiérarchie de données}
(structure organisationnelle, catalogue produits, processus métier, etc.).
Le \emph{Decomposition Tree} natif de Power BI, limité en personnalisation
et en nombre de mesures, ne répond pas à cette exigence.  
Un \textit{diagramme Sunburst} a donc été retenu : son organisation radiale
affiche plusieurs niveaux en une seule vue et compare visuellement les
contributions relatives de chaque branche
\parencite{StaskoSunburst2000}.  
L’objectif est de disposer d’un composant \textbf{personnalisable} et
\textbf{réutilisable} pour tout scénario d’analyse hiérarchique ; les
aspects d’implémentation (partition D3, drill, accessibilité) seront
présentés également au chapitre \ref{sec:implementation}.