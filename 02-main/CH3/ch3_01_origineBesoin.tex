%-----------------------------------------------------------
\section{Origine du besoin}
\label{sec:need-origin}
%-----------------------------------------------------------

Dans le cadre de son activité de conseil, la société ECRINS~SA étudie la
faisabilité de visuels Power BI dépassant les limites du catalogue
natif et des extensions AppSource.  
Deux scénarios métiers proposés par des clients—l’un marketing, l’autre
analytique—ont été retenus afin de démontrer, sous forme de
preuves de concept (PoC), la viabilité technique de visuels
personnalisés.  
Il ne s’agit pas de livrer des composants prêts à la production, mais de
valider un processus reproductible qui pourra servir de référence à de
futurs développements.

%-----------------------------------------------------------
\subsection{Besoin marketing : carte des flux passagers aéroportuaire}
\label{subsec:need-marketing-flow}
%-----------------------------------------------------------

Un client gestionnaire d’un aéroport international souhaite
cartographier les flux de passagers dans l'aéroport. 
(Zones d'entrées, halls d’enregistrement, points de contrôle, boutiques,
portes d’embarquement) pour identifier les zones de plus forte
affluence et optimiser l’emplacement des surfaces publicitaires.  
Le visuel attendu adopte le paradigme de la carte de flux : des arcs
reliant les points d’intérêt, dont l’épaisseur matérialise le volume de
voyageurs, révèlent immédiatement les axes de déplacement dominants
\parencite{GuoFlowMaps2011}.  
Une couche de carte de chaleur complétera la lecture en soulignant les
« hot-spots ».  


%-----------------------------------------------------------
\subsection{Besoin d’analyse hiérarchique : Sunburst de décomposition}
\label{subsec:need-hierarchical-sunburst}
%-----------------------------------------------------------

Le second besoin a été formulé au cours d’un entretien tenu le 13 juin 2025 entre l’auteur, la professeure C. Bioley et un consultant décisionnel d’ECRINS SA.
L’équipe souhaitait disposer d’un Decomposition Tree plus riche que le visuel natif de Power BI — celui-ci se révèle rapidement contraignant, tant par le nombre limité de mesures que par la faible marge de personnalisation offerte. 
Après analyse, le choix s’est porté sur un Sunburst, dont la structure radiale permet d’afficher plusieurs niveaux hiérarchiques en une seule vue et de comparer d’un coup d’œil les contributions relatives de chaque branche \parencite{StaskoSunburst2000}. 

