% ---------------------------------------------------------------------------
%  ch4_4B_07_synthese.tex
%  Section 4B.7 — Synthèse et positionnement pour ECRINS SA
% ---------------------------------------------------------------------------
\subsection{Synthèse et positionnement pour ECRINS SA}
\label{subsec:4B-synthese}

\selectlanguage{french}
\setlength{\parindent}{0pt}

Le Radial Sunburst Decomposition Tree  confirme la faisabilité d’une représentation hiérarchique riche au sein de Power BI tout en se différenciant nettement du Passenger-Flow Map. Là où le visuel A démontrait la gestion d’un maillage spatial et l’animation de flux, le Sunburst met l’accent sur la lisibilité analytique, la légèreté du bundle et la personnalisation par l’utilisateur final.

Grâce à d3.partition et à une allocation radiale paramétrique, le visuel restitue fidèlement les proportions numériques. 

La logique métier reste découplée du moteur de rendu : le ViewModel expose une structure à plat qui pourrait, si nécessaire, être raccordée à un autre moteur (Canvas ou WebGL) sans réécrire la chaîne amont. Les paramètres exposés dans capabilities.json — profondeur maximale, activation des animations, palettes de couleurs, KPI central — rendent le composant facilement reconfigurable par les équipes métiers.

La conformité visée (navigation clavier, contrastes, annonces ARIA) et l’intégration avec l’ISelectionManager en font un bon candidat pour un pilote interne et une base d’industrialisation. Contrairement au visuel A, aucune dépendance à un algorithme de path-finding n’est requise ; la transformation s’appuie sur le modèle hiérarchique déjà présent dans Power BI, ce qui simplifie la maintenance et le transfert de connaissances.

Pour ECRINS SA, le Sunburst peut jouer un double rôle : outil opérationnel pour l’analyse budgétaire et gabarit réutilisable pour d’autres cas d’usage hiérarchiques. Sa modularité ouvre la voie à des évolutions ciblées (agrégation automatique, règles de colorimétrie pilotées par des seuils métiers). Par rapport au Passenger-Flow Map, il illustre une stratégie complémentaire : un composant léger, paramétrable et rapidement déployable. 
% ---------------------------------------------------------------------------
