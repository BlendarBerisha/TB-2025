% -----------------------------------------------------------------
% CH7_01_mesures_techniques.tex
% Section 7.1 — Mesures techniques : performance, accessibilité, poids
% -----------------------------------------------------------------
\section{Mesures techniques : performance, accessibilité, poids}
\label{sec:mesures-techniques}

Les visuels ont été évalués selon les critères définis au chapitre 3 : temps de rendu, poids du bundle, accessibilité et respect des contraintes de sécurité (absence d’opérations interdites). Les mesures ci-dessous ont été réalisées sur le jeu de données de démonstration à l’aide du Performance Analyzer de Power BI Desktop. Les tableaux détaillés de mesures, distributions et contrôles complémentaires sont fournis en
annexe~A3.


\subsection{Temps de rendu et fluidité}

Le Passenger-Flow Map, plus algorithmique (grille de navigation, A*, simplification et rendu SVG), présente un temps de rendu moyen d'environ \SI{131}{\milli\second} sur le dataset de référence, avec des pointes restant sous le seuil de \SI{300}{\milli\second}. 

Le Sunburst atteint un temps de rendu moyen d’environ \SI{40}{\milli\second} sur le dataset de démonstration (hiérarchie multi-niveaux), conformément à l’objectif inférieur à \SI{100}{\milli\second}.

\subsection{Poids du bundle et optimisation}

Les artefacts \texttt{.pbiviz} restent largement sous le seuil interne fixé à \SI{1}{\mebi\byte} (contrôle CI sur la taille du paquet). En release minifiée, les mesures sont : \textit{Radial Sunburst Decomposition Tree} \(\approx\) 37~Ko (\(\approx\) \SI{36.1}{\kibi\byte}, \(\approx\) \SI{0.0353}{\mebi\byte}) ; \textit{Passenger-Flow Map} \(\approx\) 1006~Ko (\(\approx\) \SI{982.4}{\kibi\byte}, \(\approx\) \SI{0.959}{\mebi\byte}). Ces résultats sont obtenus grâce à un usage ciblé de D3, à l’élimination des dépendances inutiles et à une configuration stricte de build. La CI vérifie à chaque build que la taille du fichier \texttt{.pbiviz} demeure \(\leq\) \SI{1}{\mebi\byte}.

\subsection{Accessibilité et internationalisation}

Conformément au périmètre défini au chapitre 3 :
\begin{itemize}
  \item les deux visuels supportent la navigation clavier sur les éléments interactifs principaux (focus et activation Enter) ;
  \item les contrastes respectent le ratio AA (4,5:1) pour les textes et éléments essentiels ;
  \item l’internationalisation couvre les locales de test prévues (français fr-CH et anglais en-US). 
\end{itemize}


\subsection{Conformité aux contraintes de sécurité}

Les packages sont construits avec l’audit --certification-audit activé. Aucune opération interdite (appels réseau sortants, évaluation dynamique de code) n’a été détectée, et les limites de ressources imposées par l’hôte Power BI n’ont pas été dépassées lors des tests.
