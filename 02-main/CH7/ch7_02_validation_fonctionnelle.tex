% -----------------------------------------------------------------
% CH7_02_validation_fonctionnelle.tex
% Section 7.2 — Validation fonctionnelle et retours utilisateurs
% -----------------------------------------------------------------
\section{Validation fonctionnelle et retours utilisateurs}
\label{sec:validation-fonctionnelle}

Au-delà des mesures techniques, la valeur d’un visuel Power~BI se juge à son utilité perçue et à sa lisibilité dans des scénarios métier concrets. 
Dans le cadre de ce projet, la validation fonctionnelle a pris la forme d’une \textbf{revue experte formative} réalisée par la professeure référente, 
en conditions réelles dans un rapport Power~BI dédié.

\subsection{Méthodologie de la revue experte}

Conformément au protocole décrit au Chapitre~\ref{sec:validation-sources}, 
l’évaluation a consisté à examiner chaque visuel au travers de scénarios ciblés, 
alimentés par les jeux de données de démonstration présentés au Chapitre~3. 
Les cas d’usage testés étaient les suivants :
\begin{itemize}[nosep]
  \item \textbf{Carte de flux de passagers} : filtrage par tranches horaires, distinction du type de flux (départs/arrivées), activation de la couche de densité (carte de chaleur), vérification de la mise à jour des infobulles et des interactions croisées.
  \item \textbf{Sunburst hiérarchique} : navigation par drill-down et drill-up, utilisation du fil d’Ariane, sélection de branches, application de filtres numériques simples, vérification du cross-filtering avec les autres visuels du rapport.
\end{itemize}
Les observations ont été consignées sous forme de commentaires qualitatifs, 
sans prétention de représentativité statistique.

\subsection{Retours sur la carte de flux de passagers}

La lecture des grands flux est jugée immédiate et efficace. 
Il est suggéré qu’une \textit{variation d’épaisseur des traits par zone} pourrait renforcer la hiérarchie visuelle. 
La carte de chaleur est considérée comme pertinente pour identifier rapidement les zones d’affluence, 
mais l’échantillon de données disponible ne permet pas de tester pleinement son potentiel. 
L’ajout d’une \textbf{légende} explicite est recommandé, ainsi que la possibilité de 
\textit{définir manuellement les paliers de couleur} plutôt que de s’appuyer uniquement sur un calcul automatique.

\subsection{Retours sur le visuel Sunburst}

La hiérarchie et les proportions sont claires dès le premier coup d’œil. 
La navigation drill-down et le fil d’Ariane sont jugés naturels et faciles à prendre en main. 
Le KPI central n’est pas perçu comme indispensable, mais ne gêne pas la lecture ; 
il pourrait donc rester en option désactivable.

\subsection{Satisfaction générale et axes d’amélioration}

La revue experte conclut que le niveau de validation est suffisant pour un usage professionnel 
dans le cadre des périmètres visés. 
La carte de flux apparaît comme un composant spécialisé, pertinent pour l’analyse de flux spatiaux ciblés, 
avec un potentiel d’amélioration sur la personnalisation visuelle (épaisseurs, paliers de couleur, légende). 
Le Sunburst se distingue par sa polyvalence et sa facilité de réutilisation dans différents contextes hiérarchiques.
