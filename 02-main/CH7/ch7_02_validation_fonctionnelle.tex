% -----------------------------------------------------------------
% CH7_02_validation_fonctionnelle.tex
% Section 7.2 — Validation fonctionnelle et retours utilisateurs
% -----------------------------------------------------------------
\section{Validation fonctionnelle et retours utilisateurs}
\label{sec:validation-fonctionnelle}

Au-delà des mesures techniques, la valeur d’un visuel Power BI se juge à son utilité perçue par les utilisateurs finaux. Dans le cadre de ce projet, des démonstrations internes ont été organisées afin de recueillir des retours qualitatifs auprès de parties prenantes métiers concernées. 

\subsection{Méthodologie des tests}

Conformément au chapitre 3, l’évaluation fonctionnelle a combiné trois volets complémentaires : i) une revue experte basée sur des scénarios ciblés, ii) une courte démonstration interne, et iii) un mini-test utilisateur exploratoire (n=1) à visée qualitative. Chaque visuel a été présenté dans un rapport Power BI dédié, alimenté par les jeux de données de démonstration décrits au chapitre 3, avec les cas d’usage suivants :
\begin{itemize}[nosep]
  \item Carte de flux de passagers : filtrage par tranches horaires, distinction du type de flux (départs/arrivées), activation de la couche de densité (carte de chaleur), vérification de la mise à jour des infobulles et des interactions croisées.
  \item Sunburst hiérarchique : navigation par drill-down et drill-up, sélection de branches, application de filtres numériques simples, vérification du cross-filtering avec les autres visuels du rapport.
\end{itemize}
Les observations ont été consignées au moyen d’une grille d’observation et de questions ouvertes, sans prétention de représentativité statistique.

\subsection{Retours sur la carte de flux de passagers}

Les retours soulignent une lecture immédiate des axes dominants et des zones d’affluence. La bascule carte de chaleur est jugée utile pour repérer d’un coup d’œil les points chauds, tandis que les infobulles et les interactions croisées facilitent l’exploration. L’ergonomie du panneau de contrôle (filtres rapides) est appréciée.

\subsection{Retours sur le visuel Sunburst}

Le sunburst est perçu comme clair et efficace pour appréhender les hiérarchies et comparer les contributions relatives. La navigation par drill-down et drill-up, le fil d'Ariane et le KPI central améliorent la compréhension et le sens métier. La réactivité lors des interactions a été notée positivement.


\subsection{Satisfaction générale et axes d’amélioration}

Dans l’ensemble, le niveau de validation est jugé suffisant pour un usage professionnel sur les périmètres visés. La carte de flux est considérée comme un composant spécialisé, pertinent pour des besoins ciblés d’analyse spatiale. Le sunburst apparaît plus transversal et réutilisable dans plusieurs rapports hiérarchiques.

