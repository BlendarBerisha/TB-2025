\section{Architecture des visuels Power BI}
\label{sec:archi-powerbi}

Cette section présente l’architecture technique d’un visuel personnalisé (custom visual) dans Power BI. Comprendre ces fondations est essentiel pour aborder correctement le développement, l’intégration et la mise en production de ces composants.

\subsection{Isolation et exécution dans un environnement sandboxé}

Power BI charge chaque visuel dans une \texttt{iframe} sandboxée, isolée du reste du rapport. Ce conteneur est orchestré par le « Power BI host », qui injecte le SDK JavaScript, le bundle du visuel (\texttt{.js}, \texttt{.css}) et transmet les données et options de rendu via une interface standardisée.

Ce mécanisme garantit :
\begin{itemize}
  \item une isolation stricte, évitant les conflits de dépendances et les fuites mémoire~\parencite{MicrosoftSandbox2016};
  \item une compatibilité multi-visuels au sein d’un même rapport ;
  \item un canal de communication contrôlé (filtres croisés, sélection, thème).
\end{itemize}

Depuis la version~4.6 de l’API Power BI, le fichier \texttt{capabilities.json} doit déclarer explicitement les privilèges d’accès, comme les connexions réseau ou fichiers externes~\parencite{MSCapabilities2024}. La branche 5.x, publiée en mai 2024, étend ce modèle avec la prise en charge des Web Workers pour les traitements asynchrones~\parencite{PBIAPIV5_2025}.

\subsection{Structure d’un projet \texttt{pbiviz}}

L’outil en ligne de commande \texttt{pbiviz} permet de générer la structure d’un nouveau composant via :

\begin{verbatim}
pbiviz new <nom_du_visual>
\end{verbatim}

Il en résulte l’arborescence suivante :
\begin{itemize}
  \item \texttt{pbiviz.json} : métadonnées globales (nom, version, \texttt{apiVersion}) ;
  \item \texttt{capabilities.json} : déclaration des rôles de données et des options d’interactivité ;
  \item \texttt{src/visual.ts} : classe principale du visuel (implémente \texttt{IVisual}) ;
  \item \texttt{assets/} : ressources telles que les icônes ou les fichiers de traduction ;
  \item fichiers de build : \texttt{package.json}, \texttt{webpack.config.js}, etc.
\end{itemize}

L’ensemble de ces fichiers constitue le cycle de vie d’un visual Power BI, depuis sa construction jusqu’à son exécution dans le client.

\subsection{Cycle de vie du visuel : interface \texttt{IVisual}}

Le SDK Power BI impose l’implémentation d’une interface nommée \texttt{IVisual}. Celle-ci repose sur quatre méthodes principales~\parencite{PBIAPIV46_2023} :

\begin{description}
  \item[\texttt{constructor(options)}] : Initialise le visuel, reçoit le conteneur DOM et les paramètres de configuration.
  \item[\texttt{update(options)}] : Méthode clé appelée à chaque modification du rapport (données, filtre, taille, thème).
  \item[\texttt{enumerateObjectInstances()} (optionnelle)] : Permet de générer dynamiquement les options du volet « Format ».
  \item[\texttt{destroy()} (optionnelle)] : Nettoie les ressources (listeners, canvas, workers) lors du déchargement du visuel.
\end{description}

L’objet \texttt{VisualUpdateOptions} contient notamment :
\begin{itemize}
  \item la \texttt{dataView} (structure tabulaire de données) ;
  \item les dimensions du conteneur ;
  \item les options de thème et d’interactivité.
\end{itemize}

\subsection{Gestion du rendu et des interactions}

Le moteur Power BI invoque \texttt{update()} selon un modèle différentiel (\emph{diff rendering}) : seul le rendu des éléments affectés par une modification est nécessaire. Cette approche nécessite un code idempotent et optimisé.

Les interactions croisées (filtres, surbrillance) sont gérées via un \texttt{Selection Manager} fourni par la bibliothèque \texttt{powerbi-visuals-utils-interactivityutils}. Un cycle d’interaction typique est le suivant :

\begin{enumerate}
  \item L’utilisateur sélectionne un élément dans un autre visuel (ex. une barre d’un histogramme).
  \item Le host applique le filtre et met à jour la \texttt{dataView} du visuel custom.
  \item Le visuel adapte son affichage (ex. opacité réduite des éléments non sélectionnés).
\end{enumerate}

\subsection{Versions de l’API et implications sécuritaires}

Le champ \texttt{apiVersion} de \texttt{pbiviz.json} détermine l’ensemble des fonctionnalités disponibles.  
Les versions 5.x, introduites en 2024, apportent plusieurs nouveautés majeures~\parencite{PBIAPIV5_2025} :

\begin{itemize}
  \item chargement dynamique via \texttt{import('@powerbi/visuals-api')} ;
  \item support natif des Web Workers pour l’exécution parallèle ;
  \item support du \texttt{advancedEditModeSupport} (niveau 2) pour les options conditionnelles.
\end{itemize}

Le moteur sandbox interdit tout accès direct au DOM parent, au stockage local ou aux appels \texttt{fetch()} non autorisés, sauf si explicitement déclarés dans la clé \texttt{privileges}.

\subsection*{Résumé opérationnel}

Cette architecture impose une rigueur dès les premières étapes de développement :

\begin{itemize}
  \item tests ciblés par méthode : \texttt{constructor} vs \texttt{update} ;
  \item stratégie de rendu performante (DOM partiel, SVG, Canvas ou virtual DOM) ;
  \item déclaration précise des rôles de données dans \texttt{capabilities.json} ;
  \item compatibilité stricte avec les versions \texttt{apiVersion} dans le pipeline CI/CD.
\end{itemize}

Les prochaines sections examineront les autres approches disponibles (Python, R), pour justifier ensuite le choix du SDK TypeScript comme solution prioritaire.

