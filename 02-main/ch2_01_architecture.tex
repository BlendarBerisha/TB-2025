\section{Architecture des visuels Power BI}
\label{sec:archi-powerbi}

Power~BI est conçu autour d’une \textbf{architecture de visualisation ouverte
et extensible}.  
Chaque élément visuel (graphique, carte, jauge, etc.) est rendu côté client
à partir des données du modèle, via du code JavaScript/TypeScript exécuté
dans Power~BI Desktop ou dans le service web \parencite{MicrosoftOpenVis2015}.  
Depuis 2015, Microsoft propose non seulement une panoplie de visuels
« \emph{core} » (natifs), mais permet aussi l’importation de visuels
additionnels développés par la communauté ou des éditeurs tiers
\parencite{MicrosoftMarketplace2016}.  
Cette ouverture repose sur des standards web : \emph{« en s’appuyant sur des
standards ouverts d’Internet et des bibliothèques open-source comme D3.js »},
la création de visuels personnalisés a été grandement simplifiée
\parencite{MicrosoftD3Blog2017}.  
Microsoft publie d’ailleurs le code source de nombreux visuels natifs sur
GitHub, attestant de sa volonté d’encourager un écosystème ouvert
\parencite{GitHubPowerBISamples2024}.

%-----------------------------------------------------------
\subsection{Visual container et bac à sable}
\label{subsec:sandbox}
%-----------------------------------------------------------

Qu’il soit natif ou personnalisé, un visuel s’insère dans le \emph{canevas}
du rapport et interagit avec le modèle de données via des rôles prédéfinis.
Chaque visuel reçoit, du moteur Power BI, les données filtrées qui lui sont
attribuées (colonnes, mesures, hiérarchies), puis exécute son propre code de
rendu.  
Pour les visuels \emph{custom}, ce code est empaqueté dans un fichier
\verb|.pbiviz| contenant les scripts, les styles et le manifeste
\parencite{MicrosoftPbivizDocs2023}.  
Power BI exécute alors le visuel dans un \textbf{bac à sable sécurisé}
(\emph{sandbox}) : une \texttt{iframe} isolée du reste du rapport
\parencite{OkVizSandbox2022}.  
Le visuel n’accède ni aux autres visuels ni au modèle global ; il ne « voit »
que les champs que l’utilisateur lui a explicitement liés
\parencite{OkVizSandbox2022}.  
Cette mesure garantit qu’aucun code malveillant ne peut lire ou exfiltrer des
données sans autorisation \parencite{MediumSecurityPBI2023}.

%-----------------------------------------------------------
\subsection{Interactions et intégration}
\label{subsec:interactions}
%-----------------------------------------------------------

Malgré cet isolement technique, les visuels s’intègrent pleinement dans
l’expérience interactive globale.  
Un visuel personnalisé correctement développé se comporte \emph{exactement
comme un visuel natif} : il réagit aux filtres, autorise le
\textit{cross-highlight} (mise en surbrillance croisée) et expose des options
de mise en forme dans le panneau \emph{Format}
\parencite{MicrosoftCustomVisGuide2024}.  
Lorsqu’un utilisateur clique, par exemple, sur une barre d’histogramme, le
moteur Power BI propage l’événement de sélection aux autres visuels.
Si le développeur a implémenté l’API \verb|ISelectionManager|, son visuel
peut émettre et recevoir ces événements, assurant ainsi une \textbf{intégration
uniforme} des visuels natifs et ajoutés \parencite{MicrosoftSelectionAPI2024}.  

La différence fondamentale reste donc interne : les visuels natifs font
partie du produit et peuvent exploiter des API internes non exposées,
tandis que les visuels personnalisés s’appuient uniquement sur
l’API publique du SDK, avec les restrictions de sécurité détaillées en
section~\ref{sec:sdk}.