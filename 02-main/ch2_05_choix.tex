%-----------------------------------------------------------
\section{Choix technologiques (TypeScript, D3, React optionnel)}
\label{sec:techno}
%-----------------------------------------------------------

Le développement d’un visuel personnalisé Power BI repose sur un stack
web moderne articulé autour de trois briques : TypeScript pour le langage,
D3.js pour le rendu vectoriel et, à titre optionnel, React pour la
structuration de l’interface. Ce choix résulte d’une analyse des
alternatives en termes de maintenabilité, performance, sécurité et
accessibilité.

\subsection{TypeScript vs JavaScript.}
Le SDK Power BI est conçu nativement pour TypeScript, sur-ensemble typé de
JavaScript que Microsoft recommande pour les visuels personnalisés
\parencite{MicrosoftPBISDKTS2025}. Le typage statique détecte précocement
les incohérences et réduit les bogues en production : une étude
empirique portant sur plus de 400 projets GitHub montre une diminution
moyenne de 15 \% des défauts après migration vers TypeScript
\parencite{BeyerEtAl2023}. Les annotations rendent le code plus explicite,
facilitant lecture, revue et refactorisation. Par ailleurs, TypeScript
apporte des abstractions modernes — interfaces, classes, generics —
qui encouragent une architecture modulaire et extensible. Le code est
ensuite transcompilé en JavaScript ES 2019, sans impact mesurable sur les
performances d’exécution \parencite{EcmaBenchmark2024}. Ne pas passer par
cette couche (écrire directement en JavaScript ES6+) aurait simplifié la
phase de build, mais au prix d’une dette technique accrue et d’un risque de
régression plus élevé, notamment pour un composant destiné à évoluer avec
l’API Power BI.

\subsection{D3.js pour le rendu SVG}

D3 — «\,Data-Driven Documents\,» — est la librairie de référence pour manipuler le DOM SVG et créer des visualisations « sur mesure ». 
Elle établit un lien direct entre données et éléments graphiques, autorisant des transformations déclaratives efficientes \parencite{Bostock2019}. 
Cette approche bas niveau confère un contrôle complet sur chaque attribut visuel (couleur, position, animation), condition nécessaire à la réalisation de graphiques non standards répondant à des exigences métier spécifiques. 
D3 propose en outre un vaste ensemble de modules (générateurs de formes, projections cartographiques, échelles, layouts hiérarchiques) et s’appuie sur un écosystème mature d’exemples réutilisables. 
Les bibliothèques « haut niveau » telles que Chart.js ou Plotly raccourcissent le prototypage, mais leurs abstractions atteignent rapidement leurs limites pour des designs originaux ; de leur côté, les rendus basés sur canvas ou WebGL complexifient l’accessibilité et la netteté en cas de zoom. Le choix d’un SVG produit par D3 facilite enfin la mise à l’échelle et l’ajout d’attributs ARIA ou de balises \verb|<title>|, conformément aux recommandations WCAG 2.2 applicables aux rapports Power BI \parencite{W3CAccessibility2023}.


\subsection{React (optionnel) pour l’UI.}
React n’est pas requis par le SDK, mais devient pertinent dès lors que le
visuel embarque une interface utilisateur complexe : sélecteurs, menus
contextuels, légende cliquable. Sa philosophie component-based et
son virtual DOM optimisent les mises à jour d’interface en
réduisant les re-rendus coûteux \parencite{ReactDocs2024}. L’association
« React pilote la structure, D3 gère les calculs et applique les
transformations » est désormais un pattern reconnu ; plusieurs
visuels open-source l’utilisent déjà dans AppSource
\parencite{PowerBIReactD3Sample2024}. L’empreinte ajoutée (≈ 40 kB
minifiés) reste compatible avec la limite de 2,5 MiB du package .pbiviz.
Pour les projets très légers, on peut encore préférer Preact, clone
allégé compatible avec l’API React. Le coût cognitif — JSX, gestion d’état —
est maîtrisé par l’équipe et amorti par la facilité de test unitaire des
composants, réalisée ici sous Jest avec le preset officiel
jest-pbi-visuals-preset. Si le visuel n’exige qu’un rendu statique
ou des animations D3 simples, il est cohérent de se passer de React ; c’est
pourquoi l’usage reste qualifié d’« optionnel ».

\subsection{Synthèse.}
Le triptyque TypeScript + D3 (+ React) offre un compromis robuste :
rigueur logicielle, expressivité graphique, performances maîtrisées et
accessibilité native. Il s’inscrit dans les standards de l’écosystème
Power BI, maximise la réutilisabilité du savoir-faire front-end de l’équipe
et minimise la dette technique à long terme.
