% =============================================================
\chapter{État de l'art}
\label{chap:state-of-the-art}
\selectlanguage{french}
\setlength{\parindent}{0pt}



Le développement de visuels personnalisés dans Power BI s’appuie sur un écosystème technologique dense qui combine l’offre native de la plateforme et des extensions construites par code. 
Afin de situer précisément le cadre de ce travail, ce chapitre commence par décrire l'architecture interne des visuels Power BI et distingue les trois grandes familles actuellement accessibles aux concepteurs : les visuels livrés par Microsoft, les visuels générés par scripts Python/R et les visuels personnalisés réalisés au moyen du SDK officiel. 
Chaque famille est examinée tour à tour, en s’appuyant sur la documentation Microsoft (2025) et les travaux de référence en datavisualisation \parencite{Ware2019, MicrosoftPBISDKTS2025}. L’analyse porte autant sur les capacités offertes que sur les limitations fonctionnelles, de performance ou d’accessibilité que les praticiens doivent anticiper.

Dans un second temps, nous explicitons les choix technologiques retenus pour ce projet : le langage TypeScript, le moteur de rendu SVG de D3.js et, le cas échéant, l’emploi de React pour la gestion du DOM. Chaque choix est justifié au regard des exigences identifiées précédemment (maintenabilité, interactivité et conformité WCAG 2.2).

Enfin, la dernière section opère une analyse critique des écarts entre les besoins métier relevés chez ECRINS SA et l’état actuel de l’art. Ce bilan établit la légitimité d’un développement custom et prépare les décisions techniques détaillées au chapitre 5.


% -----------------------------------------------------------------
% 2.1 Concepts BI et datavisualisation
\section{Concepts BI et datavisualisation}
\label{sec:concepts-bi-dataviz}

La Business Intelligence peut se définir comme l’ensemble des méthodes et technologies visant à transformer des données brutes en connaissances utiles pour la prise de décision organisationnelle. Chen, Chiang et Storey~\parencite{Chen2012} rappellent que la valeur de la BI réside moins dans l’acquisition massive de données que dans la capacité à les modéliser, les analyser et les représenter de manière intelligible pour l’humain. Un rapport plus récent de Gartner~\parencite{Gartner2024} confirme cette évolution vers une BI self-service, dans laquelle la souplesse des visuels et leur rapidité de création constituent des facteurs différenciants.

L’étape de visualisation constitue ainsi le dernier maillon du pipeline \og ingestion, modélisation, analyse, présentation\fg{}, mais elle s’avère décisive pour convertir des métriques abstraites en informations actionnables. C’est précisément à ce niveau que se situent les visuels Power BI, natifs ou personnalisés, objets d’étude du présent travail.

Dans le domaine de la datavisualisation, les sciences cognitives ont montré que l’œil humain perçoit rapidement certains attributs dits préattentifs (position, longueur, orientation, couleur, etc.). Les travaux fondateurs de Cleveland et McGill établissent une hiérarchie de précision des encodages (la position sur une échelle commune étant la plus fiable)~\parencite{ClevelandMcGill1984}, complétés par une synthèse de référence sur les principes et processus perceptifs~\parencite{Munzner2014}. Ware~\parencite{Ware2019} démontre que l’exploitation adéquate de ces attributs maximise la vitesse et la justesse de la lecture visuelle. Tufte~\parencite{Tufte1983} a, pour sa part, popularisé l’idée de data–ink ratio, soulignant que le graphisme ne doit conserver que l’encre strictement indispensable au message ; tout élément décoratif superflu — le chart-junk — nuit à la clarté. Des travaux empiriques plus récents nuancent toutefois cette position, en montrant que la mémorabilité d’un visuel dépend aussi de facteurs perceptifs et sémantiques~\parencite{BorkinEtAl2013}. Few~\parencite{Few2009} prolonge cette perspective en montrant que la cohérence des encodages (axes, couleurs, échelles) constitue une condition essentielle pour comparer de façon fiable plusieurs séries de données.

L’unification théorique de ces principes a été proposée par Wilkinson~\parencite{Wilkinson2005} puis formalisée par Wickham sous le nom de Grammaire des graphiques. Le modèle décrit chaque graphique comme la combinaison déclarative de couches : données, transformations, géométries, échelles, systèmes de coordonnées et facettage. Ce cadre a influencé la plupart des bibliothèques modernes — notamment D3.js, Vega ou ggplot2 — et se retrouve implicitement dans l’API de Power BI ; chaque visuel y spécifie ses champs (data roles), ses encodages (capabilities) et son canevas de rendu. L’adoption de D3 pour les visuels personnalisés s’appuie par ailleurs sur une base scientifique et industrielle reconnue~\parencite{BostockOgievetskyHeer2011}.

Au-delà des principes, la BI professionnelle ajoute des impératifs tels que la performance d’affichage et l’accessibilité numérique, auxquels s’ajoute la qualité des interactions comme levier d’analyse~\parencite{HeerShneiderman2012}. Ces impératifs serviront de grille d’évaluation au chapitre~6. Les visuels standards de Power BI satisfont ces exigences pour des cas courants, mais ils se heurtent aux demandes spécifiques de certains métiers ; c’est autour de ces limites qu’émerge le besoin de visuels personnalisés. Comprendre les fondements de la BI et de la datavisualisation éclaire ainsi la double problématique du mémoire : confirmer la pertinence d’enrichir Power~BI par de nouveaux composants, puis garantir que ces composants respectent les bonnes pratiques cognitives tout en s’intégrant dans un environnement d’entreprise contrôlé.

% 2.2 Architecture des visuels Power BI
\section{Architecture des visuels Power BI}
\label{sec:archi-powerbi}

Power~BI est conçu autour d’une  {architecture de visualisation ouverte
et extensible}.  
Chaque élément visuel (graphique, carte, jauge, etc.) est rendu côté client
à partir des données du modèle, via du code JavaScript/TypeScript exécuté
dans Power~BI Desktop ou dans le service web \parencite{MicrosoftOpenVis2015}.  
Depuis 2015, Microsoft propose non seulement une panoplie de visuels
« \emph{core} » (natifs), mais permet aussi l’importation de visuels
additionnels développés par la communauté ou des éditeurs tiers
\parencite{MicrosoftMarketplace2016}.  
Cette ouverture repose sur des standards web : \emph{« en s’appuyant sur des
standards ouverts d’Internet et des bibliothèques open-source comme D3.js »},
la création de visuels personnalisés a été grandement simplifiée
\parencite{MicrosoftD3Blog2017}.  
Microsoft publie d’ailleurs le code source de nombreux visuels natifs sur
GitHub, attestant de sa volonté d’encourager un écosystème ouvert
\parencite{GitHubPowerBISamples2024}.  
 {L’API publique est passée en version 5.10, puis en préversion 6
(22 juillet 2024), introduisant un DOM sécurisé et la \textit{Rendering Events API}, ce qui
renforce la cohérence entre ouverture et sécurité} \parencite{MicrosoftApiChangelog2024}.

%-----------------------------------------------------------
\subsection{Visual container et bac à sable}
\label{subsec:sandbox}
%-----------------------------------------------------------

Qu’il soit natif ou personnalisé, un visuel s’insère dans le \emph{canevas}
du rapport et interagit avec le modèle de données via des rôles prédéfinis.
Chaque visuel reçoit, du moteur Power BI, les données filtrées qui lui sont
attribuées (colonnes, mesures, hiérarchies), puis exécute son propre code de
rendu.  
Pour les visuels \emph{custom}, ce code est empaqueté dans un fichier
\texttt{.pbiviz} contenant les scripts, les styles et le manifeste
\parencite{MicrosoftPbivizDocs2023}.  
Power BI exécute alors le visuel dans un  {bac à sable sécurisé}
(\emph{sandbox}) : une \texttt{iframe} isolée du reste du rapport
\parencite{OkVizSandbox2022}.  
Le visuel n’accède ni aux autres visuels ni au modèle global ; il ne « voit »
que les champs que l’utilisateur lui a explicitement liés.  
{Depuis la mise à jour 2.140 (février 2024), tout visuel — qu’il soit
privé (\textit{organizational visual}) ou destiné à AppSource — est audité par
l’option \texttt{pbiviz package --certification-audit} de
\textit{powerbi-visuals-tools} ≥ 6.1 : les appels réseau
(\texttt{fetch}, \texttt{XMLHttpRequest}, WebSockets) et l’évaluation dynamique
de code (\texttt{eval}, \texttt{Function}) sont bloqués, et l’exécution est
interrompue au-delà d’environ 120 s de CPU ou de 230 Mio de mémoire%
\parencite{MicrosoftCertificationGuide2025}.  
Ces garde-fous empêchent qu’un code malveillant puisse lire ou exfiltrer des données sans autorisation \parencite{MediumSecurityPBI2023}.

%-----------------------------------------------------------
\subsection{Interactions et intégration}
\label{subsec:interactions}
%-----------------------------------------------------------

Malgré cet isolement technique, les visuels s’intègrent pleinement dans
l’expérience interactive globale.  
Un visuel personnalisé correctement développé se comporte \emph{exactement
comme un visuel natif} : il réagit aux filtres, autorise le
\textit{cross-highlight} (mise en surbrillance croisée) et expose des options
de mise en forme dans le panneau \emph{Format}
\parencite{MicrosoftCustomVisGuide2024}.  
Lorsqu’un utilisateur clique, par exemple, sur une barre d’histogramme, le
moteur Power BI propage l’événement de sélection aux autres visuels.
Si le développeur a implémenté l’API \texttt{ISelectionManager}, son visuel
peut émettre et recevoir ces événements ;  {il peut également recourir
à \texttt{ITooltipService} pour les infobulles contextuelles ou à
\texttt{ILocalizationManager} pour l’internationalisation, de sorte que
l’intégration fonctionnelle et linguistique demeure uniforme}
\parencite{MicrosoftSelectionAPI2024,MicrosoftTooltipAPI2024}.  

La différence fondamentale reste donc interne : les visuels natifs font
partie du produit et peuvent exploiter des API internes non exposées,
tandis que les visuels personnalisés s’appuient uniquement sur
l’API publique du SDK, avec les restrictions de sécurité détaillées en
section~\ref{sec:sdk}.  
{La section suivante examinera d’abord les capacités et limites de ces
visuels natifs avant de traiter, en 2.3, l’approche Python/R, puis, en 2.4,
le développement complet via le SDK.}

% 2.3 Visuels scriptés Python / R
%-----------------------------------------------------------
\section{Visuels Python/R : usages, atouts, limites}
\label{sec:python-r-visuals}
%-----------------------------------------------------------

En plus des visuels préfabriqués, Power~BI permet d’incorporer des scripts Python ou R pour générer des visuels sur mesure.  
Concrètement, l’utilisateur peut ajouter un élément de type ``Python Visual'' ou ``R Visual'' dans un rapport, puis fournir un script dans l’éditeur associé.  
Le moteur Power~BI va alors exécuter ce script en coulisse, en lui transmettant les données du modèle (colonnes et mesures sélectionnées) sous forme de \textit{dataframe}, et récupérer en résultat un graphique statique produit par le code Python ou R.

Cette fonctionnalité vise à tirer parti de l’écosystème analytique de ces langages~: il devient possible d’utiliser des bibliothèques populaires comme \textit{Matplotlib}, \textit{Seaborn} ou \textit{Plotly} en Python, ou \textit{ggplot2}, \textit{plotly R} en R, afin de créer des visualisations avancées non disponibles nativement.  
Par exemple, un data scientist peut réaliser dans Power~BI un nuage de points avec une régression LOESS en R, ou un diagramme de réseau en Python, en quelques lignes de code utilisant des packages spécialisés — des visuels difficiles à obtenir autrement.

Les atouts de cette approche sont liés à la puissance et à la flexibilité des langages R et Python en matière de visualisation et de calculs statistiques.  
On bénéficie de l’énorme bibliothèque de packages open-source~: analyses statistiques poussées, machine learning, visualisations scientifiques (heatmaps, dendrogrammes, etc.), tout peut théoriquement être intégré dans un rapport.  
De plus, ces visuels scriptés permettent d’automatiser des traitements de données directement avant l’affichage.  
Par exemple, on peut programmer un script Python qui agrège des données, calcule des indicateurs personnalisés ou entraîne un modèle prédictif, puis affiche un graphique du résultat — le tout s’exécutant dynamiquement lors du rafraîchissement du visuel.  
Cela étend les capacités de Power~BI au-delà du seul langage DAX, en offrant la richesse de Python/R pour des besoins spécifiques (analyses de texte, séries temporelles avancées, etc.).

Cependant, les visuels Python/R présentent des limitations importantes dues à leur nature même de scripts externes.  
D’abord, le rendu produit est une image statique (format PNG) intégrée dans le rapport.  
Ainsi, «~les visualisations Python dans Power BI ne sont que des images statiques (résolution 72 DPI), sans aucune interactivité~»\parencite{RealPythonPowerBI2023}.  
Concrètement, l’utilisateur ne peut pas cliquer sur un élément du graphique Python/R pour filtrer d’autres visuels — toute la surface du visuel est une image plane.  
Il y a bien une interaction partielle~: si l’on applique un filtre ou sélectionne un élément sur un autre visuel du rapport, Power~BI ré-exécute le script Python/R avec les données filtrées, ce qui met à jour l’image correspondante\parencite{RealPythonPowerBI2023}.  
Mais cela reste plus lent et moins fluide que les visuels natifs, car il faut relancer le calcul du script à chaque changement.

Microsoft documente que ces visuels scriptés «~se rafraîchissent lors des mises à jour ou filtrages des données, mais l’image en elle-même n’est pas interactive~»\parencite{MicrosoftPythonRVisualsDocs2024}.  
Ils répondent aux \textit{highlightings} provenant d’autres visuels, «~mais on ne peut pas sélectionner des éléments du visuel Python/R pour croiser le filtre~»\parencite{MicrosoftPythonRVisualsDocs2024}.  
Cette absence d’interactivité locale est un frein dans un tableau de bord où l’on attend généralement de pouvoir explorer les données de façon interactive.

Une autre contrainte forte est la taille des données transmises aux scripts.  
Pour éviter des temps d’exécution excessifs, Power~BI limite à 150\,000 lignes le volume de données qu’un visuel Python ou R peut traiter.  
«~Si plus de 150\,000 lignes sont sélectionnées, seules les 150\,000 premières sont utilisées~», et un message d’avertissement s’affiche sur l’image\parencite{MicrosoftPythonRVisualsDocs2024}.  
De plus, l’ensemble des données en entrée du script ne doit pas excéder 250~MB en mémoire\parencite{MicrosoftPythonRVisualsDocs2024}.  
Ces seuils signifient que les visuels Python/R sont adaptés à des échantillons de données ou des agrégats, mais pas au traitement de données massives brutes.  
Au-delà, il faut compter sur le modèle tabulaire de Power~BI pour pré-agréger ou filtrer avant d’envoyer au script.

En termes de performance, il existe également un \textit{timeout} de 5~minutes~: si le script met plus de 5~minutes à s’exécuter, il sera interrompu et le visuel affichera une erreur\parencite{MicrosoftPythonRVisualsDocs2024}.  
Cela empêche l’utilisation de calculs trop longs.  
Par exemple, entraîner un modèle complexe de machine learning sur un jeu de données volumineux dépasserait ce temps imparti.

Il convient aussi de noter des limitations techniques supplémentaires~: les chaînes de caractères de plus de 32\,766 caractères sont tronquées lors de la transmission au \textit{dataframe}\parencite{MicrosoftPythonRVisualsDocs2024} ;  
certains appareils ou environnements cloud ne supportent pas ces visuels (par exemple, les rapports publiés via \textit{publish to web} ne peuvent pas afficher de visuels Python/R pour des raisons de sécurité)\parencite{MicrosoftPythonRVisualsDocs2024}.  
De plus, pour utiliser un visuel Python ou R en Power~BI Desktop, il faut que l’utilisateur ait installé localement l’interpréteur Python ou R, ainsi que les packages nécessaires.  
Sur le service Power~BI en ligne, l’exécution des scripts est prise en charge côté serveur, mais seulement pour les tenants disposant de capacités Premium ou pour les utilisateurs Pro — ce qui signifie qu’un utilisateur Free ne pourra pas voir un visuel Python/R dans un rapport à moins que celui-ci soit hébergé sur un \textit{workspace Premium}\parencite{MicrosoftPythonRVisualsDocs2024}.

Enfin, du point de vue sécurité, Power~BI traite les visuels Python/R comme du code potentiellement risqué.  
Lorsqu’on ajoute pour la première fois un tel visuel, un avertissement de sécurité apparaît, invitant l’utilisateur à n’exécuter que des scripts de source fiable\parencite{MicrosoftPythonRVisualsDocs2024}.  
En entreprise, l’utilisation de ces visuels peut soulever des enjeux de validation du code et de maintenance (le script étant incorporé dans le rapport, il doit être maintenu manuellement en cas de mise à jour de librairie, etc.).

\textbf{Bilan.}  
Les visuels Python et R offrent donc une solution d’appoint puissante pour réaliser des visualisations avancées ou des analyses spécifiques au sein de Power~BI, sans avoir à développer un visuel \textit{custom} complet.  
Ils comblent certaines lacunes des visuels natifs en ouvrant la porte à la riche panoplie des bibliothèques \textit{data science}.  
Cependant, ils doivent être utilisés en connaissance de leurs limites~: performances réduites, absence d’interactivité directe, et restrictions d’usage dans l’environnement de service Power~BI.  
Pour un besoin de visualisation récurrent, à destination d’un large public, ou nécessitant une expérience utilisateur interactive, il peut être préférable de développer un visuel personnalisé via le SDK (ou d’utiliser un visuel custom existant sur AppSource) plutôt que de s’appuyer sur un script Python/R intégré.  
C’est ce que nous examinons dans la section suivante.

% 2.4 SDK Power BI : structure, sécurité, pipeline
%-----------------------------------------------------------
\section{SDK Custom Visuals : principes, sécurité, pipeline}
\label{sec:sdk}
%-----------------------------------------------------------

Lorsque les visuels natifs ne suffisent plus et qu’un script Python ou R se révèle trop limité, la solution la plus aboutie consiste à créer un visuel complet à l’aide du Software Development Kit (SDK) de Power BI. Microsoft distribue ce SDK sous la forme du package npm powerbi-visuals-tools, dont la version stable 6.1.2 publiée le 26 mai 2025 sert de référence pour ce travail \parencite{MicrosoftSDKNpm2025}.  

\subsection{Principe général.} Un visuel Power BI est, en substance, une mini-application web encapsulée. Le développeur décrit, dans un fichier capabilities.json, les champs de données et les options de format qu’il souhaite exposer, puis implémente en TypeScript une classe qui respecte l’interface IVisual. Power BI invoque la méthode update(options) à chaque changement de filtre ou de données ; le code traduit alors les informations reçues en éléments DOM (SVG ou Canvas) à l’aide, par exemple, de D3.js.  

\subsection{Pipeline de développement}\label{sec:pipeline-ref}

Le Power BI Custom Visuals SDK prescrit un flux de travail de référence :  
pbiviz new <NomDuVisuel> crée le squelette (manifeste, capabilities.json, fichiers TypeScript),  
puis pbiviz package --certification-audit génère l’archive .pbiviz tout en exécutant les contrôles réseau et mémoire recommandés \parencite{MicrosoftAuditCLI2025}.  
Depuis la dépréciation du serveur de développement intégré (2024), les tests interactifs s’effectuent dans un workspace développeur Power BI Service, avant qu’un package conforme ne soit importé dans un rapport ou soumis à AppSource pour certification \parencite{MicrosoftCertificationGuide2025}.

\subsection{Cadre de sécurité.} Le code exécuté l’est toujours dans une iframe sandbox dont la politique de contenu interdit tout appel réseau non approuvé et toute évaluation dynamique de code (eval, new Function) ; le SDK force également le dessin à l’intérieur de la bounding box du visuel et n’expose au script que les champs explicitement liés par l’utilisateur \parencite{OkVizSandbox2022}. Les règles de certification 2025 précisent qu’un composant destiné à AppSource doit renoncer à tout flux sortant \parencite{MicrosoftAPIv6CSP2025}.  

\subsection{Certification et limitations associées}\label{sec:certification}

La certification « Power BI Certified », réservée aux visuels publiés sur AppSource, conditionne l’inclusion du composant dans les exports PDF/PowerPoint, son rendu dans les rapports distribués par courriel et l’accès aux privileged APIs telles que FileDownload ou Licensing \parencite{MicrosoftCustomVisualsCertified2025, MicrosoftFileDownloadAPI2024}. Un visuel non certifié est exclu de ces scénarios et peut être bloqué si l’administrateur active l’option locataire « Add and use certified visuals only » \parencite{MicrosoftTenantSettings2024}.

\subsection{Synthèse.} Le SDK ouvre la voie à des composants strictement sur mesure qui héritent de la totalité des interactions offertes par Power BI ; l’effort de développement et la discipline de gouvernance qu’il requiert constituent la contrepartie logique de cette flexibilité. La section \ref{sec:techno} précise les choix d’outillage — TypeScript, D3.js et React optionnel — retenus pour réaliser les preuves de concept développées dans ce mémoire.

% 2.5 Choix technologiques (TypeScript, D3, React optionnel)
%-----------------------------------------------------------
\section{Choix technologiques (TypeScript, D3, React optionnel)}
\label{sec:techno}
%-----------------------------------------------------------

Le développement d’un visuel personnalisé Power BI repose sur un stack
web moderne articulé autour de trois briques : TypeScript pour le langage,
D3.js pour le rendu vectoriel et, à titre optionnel, React pour la
structuration de l’interface. Ce choix résulte d’une analyse des
alternatives en termes de maintenabilité, performance, sécurité et
accessibilité.

\subsection{TypeScript vs JavaScript.}
Le SDK Power BI est conçu nativement pour TypeScript, sur-ensemble typé de
JavaScript que Microsoft recommande pour les visuels personnalisés
\parencite{MicrosoftPBISDKTS2025}. Le typage statique détecte précocement
les incohérences et réduit les bogues en production : une étude
empirique portant sur plus de 400 projets GitHub montre une diminution
moyenne de 15 \% des défauts après migration vers TypeScript
\parencite{BeyerEtAl2023}. Les annotations rendent le code plus explicite,
facilitant lecture, revue et refactorisation. Par ailleurs, TypeScript
apporte des abstractions modernes — interfaces, classes, generics —
qui encouragent une architecture modulaire et extensible. Le code est
ensuite transcompilé en JavaScript ES 2019, sans impact mesurable sur les
performances d’exécution \parencite{EcmaBenchmark2024}. Ne pas passer par
cette couche (écrire directement en JavaScript ES6+) aurait simplifié la
phase de build, mais au prix d’une dette technique accrue et d’un risque de
régression plus élevé, notamment pour un composant destiné à évoluer avec
l’API Power BI.

\subsection{D3.js pour le rendu SVG}

D3 — «\,Data-Driven Documents\,» — est la librairie de référence pour manipuler le DOM SVG et créer des visualisations « sur mesure ». 
Elle établit un lien direct entre données et éléments graphiques, autorisant des transformations déclaratives efficientes \parencite{Bostock2019}. 
Cette approche bas niveau confère un contrôle complet sur chaque attribut visuel (couleur, position, animation), condition nécessaire à la réalisation de graphiques non standards répondant à des exigences métier spécifiques. 
D3 propose en outre un vaste ensemble de modules (générateurs de formes, projections cartographiques, échelles, layouts hiérarchiques) et s’appuie sur un écosystème mature d’exemples réutilisables. 
Les bibliothèques « haut niveau » telles que Chart.js ou Plotly raccourcissent le prototypage, mais leurs abstractions atteignent rapidement leurs limites pour des designs originaux ; de leur côté, les rendus basés sur canvas ou WebGL complexifient l’accessibilité et la netteté en cas de zoom. Le choix d’un SVG produit par D3 facilite enfin la mise à l’échelle et l’ajout d’attributs ARIA ou de balises \verb|<title>|, conformément aux recommandations WCAG 2.2 applicables aux rapports Power BI \parencite{W3CAccessibility2023}.


\subsection{React (optionnel) pour l’UI.}
React n’est pas requis par le SDK, mais devient pertinent dès lors que le
visuel embarque une interface utilisateur complexe : sélecteurs, menus
contextuels, légende cliquable. Sa philosophie component-based et
son virtual DOM optimisent les mises à jour d’interface en
réduisant les re-rendus coûteux \parencite{ReactDocs2024}. L’association
« React pilote la structure, D3 gère les calculs et applique les
transformations » est désormais un pattern reconnu ; plusieurs
visuels open-source l’utilisent déjà dans AppSource
\parencite{PowerBIReactD3Sample2024}. L’empreinte ajoutée (≈ 40 kB
minifiés) reste compatible avec la limite de 2,5 MiB du package .pbiviz.
Pour les projets très légers, on peut encore préférer Preact, clone
allégé compatible avec l’API React. Le coût cognitif — JSX, gestion d’état —
est maîtrisé par l’équipe et amorti par la facilité de test unitaire des
composants, réalisée ici sous Jest avec le preset officiel
jest-pbi-visuals-preset. Si le visuel n’exige qu’un rendu statique
ou des animations D3 simples, il est cohérent de se passer de React ; c’est
pourquoi l’usage reste qualifié d’« optionnel ».

\subsection{Synthèse.}
Le triptyque TypeScript + D3 (+ React) offre un compromis robuste :
rigueur logicielle, expressivité graphique, performances maîtrisées et
accessibilité native. Il s’inscrit dans les standards de l’écosystème
Power BI, maximise la réutilisabilité du savoir-faire front-end de l’équipe
et minimise la dette technique à long terme.

% 2.6 Solutions concurrentes (Tableau, Qlik, Looker)
%-----------------------------------------------------------
\section{Solutions concurrentes (Tableau, Qlik, Looker)}
\label{sec:concurrence}
%-----------------------------------------------------------

Les principales plateformes BI concurrentes — Tableau, Qlik Sense et
Looker — proposent chacune des mécanismes de personnalisation de visuels
différents de ceux de Power BI, tant sur le plan technique que dans leurs
implications métier. Examiner ces modèles permet de situer les « visuels
custom » Power BI dans un paysage concurrentiel plus large.

\subsection{Tableau.}  
Tableau étend ses capacités à l’aide des Tableau Extensions, des
applications Web encapsulées dans le tableau de bord par l’Extension API
\parencite{TableauExtGuide2024}. Le composant, écrit en JavaScript et
logé dans une iframe, peut introduire un graphique inédit, du
write-back ou l’intégration d’un service tiers
\parencite{TableauBlogExt2024}. Depuis la version 2019.4, les extensions
s’exécutent par défaut en sandbox : aucun accès réseau n’est permis
sans liste blanche explicite dans l’interface d’administration
\parencite{TableauAdmin2025}. Tableau délègue le support de ces briques à
leurs éditeurs, si bien que chaque entreprise doit auditer la fiabilité de
la source avant déploiement. Sur le plan financier, le coût d’entrée reste
sensiblement supérieur à celui de Power BI : l’abonnement Creator est passé
à 75 USD par utilisateur et par mois en juillet 2025
\parencite{TableauPricing2025}. La personnalisation n’est donc rentable que
si l’organisation dispose déjà d’une base installée Tableau ou d’un budget
élargi.

\subsection{Qlik Sense.}  
Qlik propose les Visualization Extensions : objets écrits
en HTML/JavaScript/CSS et déclarés par un manifeste .qext
\parencite{QlikDevHub2024}. L’extension, quand elle respecte l’API Qlik,
s’intègre comme un objet natif, profite du moteur associatif et réagit aux
sélections \parencite{QlikExtAPI2024}. Qlik mise sur une dynamique
communautaire : de nombreux composants sont partagés via Qlik Branch
sans validation officielle, la gouvernance incombant à l’administrateur
qui doit installer manuellement l’archive sur le tenant SaaS.
Commercialement, la déclinaison « Qlik Cloud Starter » débute à
200 USD/mois pour dix utilisateurs, la formule Standard à 825 USD et la
Premium à 2 750 USD, chaque palier incluant un volume de données et un
nombre maximal de créateurs \parencite{QlikPricing2025}. Le modèle reste
avantageux quand l’entreprise possède une équipe front-end JavaScript
capable de maintenir ces extensions.

\subsection{Looker.}  
Looker, désormais composante de Google Cloud, autorise des visuels
spécifiques via le Looker Custom Visualization SDK. Le développeur
implémente la fonction updateAsync qui reçoit les données
d’une explore Looker et restitue un rendu SVG ou Canvas
\parencite{LookerVizSDK2025}. Pour diffusion publique, Google impose une
revue Marketplace : le code doit être hébergé sur GitHub et passer un
contrôle de conformité avant publication
\parencite{LookerMarketplace2024}. Il reste toutefois possible de limiter
l’usage à un tenant interne via l’interface Admin. La logique de Looker
demeure plus centralisée : la personnalisation est permise, mais soumise à
un contrôle étroit et, surtout, à une tarification négociée au cas par cas
qui dépasse largement les paliers Tableau ou Qlik ; les clients visent
donc des gains élevés sur un périmètre de visuels custom restreint.

\subsection{Lecture comparative.}  
Tableau et Qlik accordent à l’utilisateur avancé la liberté de charger
localement une extension, tout en transférant la gouvernance de sécurité à
l’entreprise ; Power BI et Looker exigent au contraire une validation
centralisée — certification AppSource ou revue Marketplace — avant tout
déploiement massif. Techniquement, les trois concurrents s’appuient, comme
Power BI, sur le triptyque HTML/JS/CSS, mais Power BI se démarque par un
SDK TypeScript/D3 structuré, une CLI d’audit intégrée et un vivier
AppSource de presque 600 visuels au 1\textsuperscript{er}
août 2025 \parencite{AppSourceCount2025}. Sur le plan économique, Power BI
reste l’option la plus abordable pour industrialiser des visuels custom à
grande échelle, tandis que Tableau, Qlik et surtout Looker réservent la
personnalisation intensive aux environnements dont le budget justifie
l’investissement initial et la maintenance continue.

% 2.7 Analyse des gaps et opportunités
%-----------------------------------------------------------
\section{Synthèse des écarts \& opportunités}
\label{sec:synthese}
%-----------------------------------------------------------

Après avoir étudié chacune des approches disponibles à l’intérieur même de Power BI, il est possible de dresser un panorama clair de leurs forces et de leurs limites, puis d’en déduire les cas d’usage qui conviennent le mieux à ECRINS SA. Quatre familles de solutions se distinguent : les visuels natifs, les scripts Python / R, les composants certifiés AppSource et les visuels développés avec le SDK.


\subsection{Visuels natifs Power BI.}  
Fournis d’office, ces graphiques constituent le socle de la majorité des rapports ; ils sont maintenus par Microsoft, optimisés pour la performance et immédiatement compatibles avec l’ensemble des interactions (sélections croisées, filtres, export PDF / PPT, affichage mobile). Leur fiabilité et leur sécurité sont maximales, aucun code externe n’étant exécuté. Leur rigidité demeure cependant le principal frein : dès qu’un scénario exige un diagramme alluvial, une cartographie indoor ou un bullet chart spécifique, l’utilisateur doit composer avec les limites de paramétrage ou se tourner vers une autre voie.


\subsection{Visuels Python / R.}  
L’exécution d’un script Python ou R ouvre la porte à l’immense écosystème de ces deux langages ; tout graphique réalisable dans Matplotlib, Seaborn, ggplot2 ou Plotly peut, en théorie, être intégré. La contrepartie tient dans la nature strictement statique du rendu : Power BI génère un PNG 72 DPI dépourvu d’interactivité et relance le script à chaque rafraîchissement, allongeant sensiblement le temps de calcul. Au-delà de 150 000 lignes transmises, la plateforme tronque les données, tandis que le service Cloud limite l’image à 2 Mio et impose un runtime R 4.3.3 / Python 3.11 contrôlé par Microsoft. La démarche convient donc surtout au prototypage analytique rapide — rarement à un déploiement massif.


\subsection{Visuels AppSource (certifiés).}  
AppSource, la galerie officielle de Microsoft, répertorie un peu moins de 600 visuels au 1\textsuperscript{er} août 2025 \parencite{AppSourceCount2025}. Chaque composant apparaît après une revue de sécurité et de performance ; il se comporte, pour l’utilisateur final, comme un visuel natif tout en couvrant des besoins de niche (Gantt, Waterfall amélioré, radar avancé). L’éditeur demeure toutefois responsable des mises à jour ; certains modules reposent sur un abonnement SaaS payant. L’approche constitue donc un juste milieu entre « prêt à l’emploi » et flexibilité.


\subsection{Visuels SDK Power BI.}  
Le SDK — TypeScript, D3 et, au besoin, React — autorise la création d’un composant exactement aligné sur les spécifications métier : intégration complète avec les filtres, compatibilité mobile, export, diffusion via AppSource ou magasin organisationnel. Le revers est un effort de développement plus élevé et l’obligation d’une gouvernance logicielle continue : veille des versions, correctifs de sécurité, tests unitaires Jest et gestion du certificat X.509.

\subsection{Vue d’ensemble comparative.}

\begin{sidewaystable}[p]
\footnotesize
\centering
\caption{Comparaison factuelle des voies de personnalisation Power BI (état : août 2025)}
\label{tab:comparaison-approches}
\begin{tabularx}{\linewidth}{>{\raggedright\arraybackslash\bfseries}p{2.9cm}XXXX}
\toprule
\textbf{Critère} &
\textbf{Visuels natifs} &
\textbf{Scripts Python / R} &
\textbf{Visuels AppSource (certifiés)} &
\textbf{Visuels SDK internes (.pbiviz)} \\
\midrule
Cas d’usage typique &
Tableaux de bord courants ; graphiques standards interactifs. &
Analyses statistiques ou graphiques scientifiques spécifiques issus de code R/Python. &
Graphiques spécialisés prêts à l’emploi (financiers, Gantt, KPI, etc.). &
Visuels sur-mesure répondant à des exigences métiers uniques ou branding interne.\\[0.4em]

Limitations principales &
Borné à la bibliothèque Microsoft ; pas de nouveau type sans extension. &
Image statique ; 150 k lignes; pas de sélection croisée ; timeout 5 min. &
Pas d’appel réseau externe ; mises à jour soumises à re-certification ; version freemium possible. &
Pas d’export PDF/PPT ni d’e-mailing ; optimisation et sécurité 100 \% à charge du dev.\\[0.4em]

Situation d’utilisation recommandée &
Choix par défaut tant que le visuel existe en natif. &
POC data-science, prototypes analytiques ponctuels. &
Quand un visuel AppSource couvre exactement le besoin et qu’une diffusion large est visée. &
Quand aucune solution existante ne convient et qu’on dispose des ressources de développement.\\[0.4em]

Accessibilité (WCAG 2.2) &
Conformité assurée par Microsoft. &
Faible : rendu bitmap non lisible par lecteur d’écran. &
Variable : dépend de chaque éditeur ; à tester avant adoption. &
Dépend entièrement du développeur ; nécessite implémentation ARIA, focus clavier, contraste.\\[0.4em]

Difficulté d’utilisation &
Très faible ; glisser-déposer. &
Élevée ; compétences R/Python requises. &
Moyenne ; configuration similaire aux visuels natifs. &
Très élevée ; maîtrise TypeScript + D3 et API Power BI.\\[0.4em]

Performance (rendu / volumétrie) &
Optimisée ; jusqu’à ~30 M points selon le type. &
Plus lente ; recalcul complet du PNG à chaque interaction. &
Dépend du visuel ; généralement satisfaisant < 30 k points. &
Variable ; plafonné à ~30 k points, dépend de l’optimisation réalisée.\\[0.4em]

Coût (licence / dev) &
Inclus dans Power BI. &
Inclus ; demande temps de codage et maintien packages. &
Souvent gratuit ; licence possible pour fonctionnalités avancées. &
Frais de développement internes importants ; maintenance continue.\\[0.4em]

Réutilisabilité / industrialisation &
Automatique ; présent partout. &
Faible ; script encapsulé dans chaque rapport. &
Élevée ; déploiement via AppSource ou store orga ; mises à jour automatiques. &
Bonne en interne via store orga ; mises à jour manuelles ; partage externe impossible sans certification.\\
\bottomrule
\end{tabularx}
\end{sidewaystable}


% Ensure the sidewaystable environment is properly closed before the document ends.


\subsection{Conclusion.}  
Chaque approche comble un manque laissé par les autres : les visuels natifs offrent robustesse et simplicité, les scripts Python / R autorisent un prototypage statistique rapide au prix d’un rendu non interactif, AppSource fournit une solution plug-and-play pour des besoins spécialisés courants, et le SDK ouvre la voie aux composants entièrement sur mesure, moyennant un investissement en développement et en gouvernance. Pour ECRINS SA, la maîtrise du SDK constitue la clé méthodologique afin de répondre, à terme, aux demandes hors norme — un enjeu approfondi dans le chapitre 3.





% =============================================================
