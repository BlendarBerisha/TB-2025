% =============================================================
\chapter{État de l'art}
\label{chap:state-of-the-art}
\selectlanguage{french}
\setlength{\parindent}{0pt}


Le développement de visuels personnalisés dans Power~BI s’appuie sur un
écosystème technologique riche, combinant les fonctionnalités \emph{natives}
de la plateforme et des extensions via code.  
Ce chapitre présente d’abord l’\textbf{architecture} des visuels Power~BI et
distingue les différentes catégories de visuels disponibles.  
Ensuite, il examine successivement les \textbf{visuels natifs}
(ceux fournis par défaut par Microsoft), les \textbf{visuels basés sur
Python/R} (générés à partir de scripts), et enfin les \textbf{visuels
personnalisés via le SDK} officiel.  
Chaque section discute des capacités offertes et des limites inhérentes.  
Enfin, nous abordons les \textbf{choix technologiques} pour la création de
nouveaux visuels (langage TypeScript, bibliothèque~D3.js, usage éventuel de
React), en justifiant ces choix dans le contexte actuel.  
L’objectif est d’établir l’état de l’art des technologies de visualisation de
données dans Power~BI, tout en adoptant un regard critique sur leurs forces et
faiblesses respectives.

% -----------------------------------------------------------------
% 2.1 Architecture des visuels Power BI
\section{Architecture des visuels Power BI}
\label{sec:archi-powerbi}

Cette section présente l’architecture technique d’un visuel personnalisé (custom visual) dans Power BI. Comprendre ces fondations est essentiel pour aborder correctement le développement, l’intégration et la mise en production de ces composants.

\subsection{Isolation et exécution dans un environnement sandboxé}

Power BI charge chaque visuel dans une \texttt{iframe} sandboxée, isolée du reste du rapport. Ce conteneur est orchestré par le « Power BI host », qui injecte le SDK JavaScript, le bundle du visuel (\texttt{.js}, \texttt{.css}) et transmet les données et options de rendu via une interface standardisée.

Ce mécanisme garantit :
\begin{itemize}
  \item une isolation stricte, évitant les conflits de dépendances et les fuites mémoire~\parencite{MicrosoftSandbox2016};
  \item une compatibilité multi-visuels au sein d’un même rapport ;
  \item un canal de communication contrôlé (filtres croisés, sélection, thème).
\end{itemize}

Depuis la version~4.6 de l’API Power BI, le fichier \texttt{capabilities.json} doit déclarer explicitement les privilèges d’accès, comme les connexions réseau ou fichiers externes~\parencite{MSCapabilities2024}. La branche 5.x, publiée en mai 2024, étend ce modèle avec la prise en charge des Web Workers pour les traitements asynchrones~\parencite{PBIAPIV5_2025}.

\subsection{Structure d’un projet \texttt{pbiviz}}

L’outil en ligne de commande \texttt{pbiviz} permet de générer la structure d’un nouveau composant via :

\begin{verbatim}
pbiviz new <nom_du_visual>
\end{verbatim}

Il en résulte l’arborescence suivante :
\begin{itemize}
  \item \texttt{pbiviz.json} : métadonnées globales (nom, version, \texttt{apiVersion}) ;
  \item \texttt{capabilities.json} : déclaration des rôles de données et des options d’interactivité ;
  \item \texttt{src/visual.ts} : classe principale du visuel (implémente \texttt{IVisual}) ;
  \item \texttt{assets/} : ressources telles que les icônes ou les fichiers de traduction ;
  \item fichiers de build : \texttt{package.json}, \texttt{webpack.config.js}, etc.
\end{itemize}

L’ensemble de ces fichiers constitue le cycle de vie d’un visual Power BI, depuis sa construction jusqu’à son exécution dans le client.

\subsection{Cycle de vie du visuel : interface \texttt{IVisual}}

Le SDK Power BI impose l’implémentation d’une interface nommée \texttt{IVisual}. Celle-ci repose sur quatre méthodes principales~\parencite{PBIAPIV46_2023} :

\begin{description}
  \item[\texttt{constructor(options)}] : Initialise le visuel, reçoit le conteneur DOM et les paramètres de configuration.
  \item[\texttt{update(options)}] : Méthode clé appelée à chaque modification du rapport (données, filtre, taille, thème).
  \item[\texttt{enumerateObjectInstances()} (optionnelle)] : Permet de générer dynamiquement les options du volet « Format ».
  \item[\texttt{destroy()} (optionnelle)] : Nettoie les ressources (listeners, canvas, workers) lors du déchargement du visuel.
\end{description}

L’objet \texttt{VisualUpdateOptions} contient notamment :
\begin{itemize}
  \item la \texttt{dataView} (structure tabulaire de données) ;
  \item les dimensions du conteneur ;
  \item les options de thème et d’interactivité.
\end{itemize}

\subsection{Gestion du rendu et des interactions}

Le moteur Power BI invoque \texttt{update()} selon un modèle différentiel (\emph{diff rendering}) : seul le rendu des éléments affectés par une modification est nécessaire. Cette approche nécessite un code idempotent et optimisé.

Les interactions croisées (filtres, surbrillance) sont gérées via un \texttt{Selection Manager} fourni par la bibliothèque \texttt{powerbi-visuals-utils-interactivityutils}. Un cycle d’interaction typique est le suivant :

\begin{enumerate}
  \item L’utilisateur sélectionne un élément dans un autre visuel (ex. une barre d’un histogramme).
  \item Le host applique le filtre et met à jour la \texttt{dataView} du visuel custom.
  \item Le visuel adapte son affichage (ex. opacité réduite des éléments non sélectionnés).
\end{enumerate}

\subsection{Versions de l’API et implications sécuritaires}

Le champ \texttt{apiVersion} de \texttt{pbiviz.json} détermine l’ensemble des fonctionnalités disponibles.  
Les versions 5.x, introduites en 2024, apportent plusieurs nouveautés majeures~\parencite{PBIAPIV5_2025} :

\begin{itemize}
  \item chargement dynamique via \texttt{import('@powerbi/visuals-api')} ;
  \item support natif des Web Workers pour l’exécution parallèle ;
  \item support du \texttt{advancedEditModeSupport} (niveau 2) pour les options conditionnelles.
\end{itemize}

Le moteur sandbox interdit tout accès direct au DOM parent, au stockage local ou aux appels \texttt{fetch()} non autorisés, sauf si explicitement déclarés dans la clé \texttt{privileges}.

\subsection*{Résumé opérationnel}

Cette architecture impose une rigueur dès les premières étapes de développement :

\begin{itemize}
  \item tests ciblés par méthode : \texttt{constructor} vs \texttt{update} ;
  \item stratégie de rendu performante (DOM partiel, SVG, Canvas ou virtual DOM) ;
  \item déclaration précise des rôles de données dans \texttt{capabilities.json} ;
  \item compatibilité stricte avec les versions \texttt{apiVersion} dans le pipeline CI/CD.
\end{itemize}

Les prochaines sections examineront les autres approches disponibles (Python, R), pour justifier ensuite le choix du SDK TypeScript comme solution prioritaire.



% 2.2 Visuels natifs : analyse fonctionnelle
\section{Visuels natifs : capacités et limites}
\label{sec:natifs-powerbi}

Avant d'envisager la création de visuels personnalisés via Python, R ou le SDK TypeScript, il convient de dresser un panorama complet des visuels dits \emph{natifs} disponibles dans Power BI. Ceux-ci sont préinstallés et activement maintenus par Microsoft, couvrant la majorité des besoins courants en analyse et restitution de données. Cette section propose non seulement une typologie complète, mais aussi une lecture critique de leur rôle et de leurs limites dans un contexte professionnel.

\subsection{Typologie fonctionnelle des visuels standards}

\subsubsection{Graphiques de comparaison}
Barres et colonnes (groupées, empilées, 100~\%) permettent de visualiser des agrégations comparées entre catégories. Elles acceptent le tri dynamique et le forage hiérarchique (drill-down) jusqu'au niveau de granularité le plus fin. Leur usage est optimal pour des tableaux de bord orientés suivi de performance, mais reste limité par l'absence de sous-totaux intermédiaires et la rigidité des axes (un seul axe secondaire).

\subsubsection{Graphiques de tendance}
Les courbes et aires (simples, empilées, 100~\%) modélisent des séries temporelles ou continues. Idéales pour l'analyse temporelle de KPI, les courbes supportent l'affichage d'un axe secondaire, et les aires mettent en valeur les volumes cumulés. Leur capacité de drill-down et de filtrage croisé les rend adaptées à des rapports interactifs.

\subsubsection{Graphiques combinés}
Les visuels de type ``combo'' (barres + lignes) permettent de superposer par exemple un chiffre d'affaires mensuel et une marge en pourcentage. Ils sont recommandés lorsque deux échelles de mesure doivent coexister dans un même visuel, sans multiplier les graphiques.

\subsubsection{Graphique de ruban}
Spécifique à la représentation du \emph{rang}, le graphique de ruban est adapté à l'analyse de parts de marché dynamiques. Il est utile pour suivre les changements de position relative de catégories dans le temps (produits, régions, marques).

\subsubsection{Graphiques de processus}
\emph{Cascade} (waterfall) permet de visualiser des variations successives contribuant à un total, tandis que \emph{entonnoir} (funnel) illustre les déperditions d'un processus (par exemple ventes ou parcours client). Le waterfall accepte les sous-totaux intermédiaires, contrairement à l'entonnoir.

\subsubsection{Nuages de points et bulles}
Le scatter plot est utile pour explorer les corrélations entre deux variables quantitatives. L'ajout d'une dimension via la taille de la bulle permet des lectures croisées. Ces visuels sont pertinents pour des analyses exploratoires mais se heurtent à une limite de volume : au-delà de 30\,000 points, l’agrégation est imposée.

\subsubsection{Graphiques circulaires}
Secteurs (pie) et anneaux (donut) représentent les parts d'un total. Bien que fréquemment utilisés, ils sont à réserver à un nombre réduit de segments (idéalement $<6$) sous peine de perte de lisibilité. Leur usage reste plus esthétique que fonctionnel.

\subsubsection{Treemaps et arborescences}
Les treemaps permettent d'encapsuler plusieurs niveaux hiérarchiques dans une surface limitée. Leur lecture est efficace pour les structures catégorielles denses (portefeuilles produits, répartitions géographiques). Un clic fore dans la hiérarchie, avec mise en forme conditionnelle possible.

\subsubsection{Cartographie}
\begin{itemize}
  \item \textbf{Carte Bing} : affiche des bulles proportionnelles sur fond routier ; recommandée pour des localisations simples.
  \item \textbf{Carte remplie (choroplèthe)} : colore les régions selon une valeur agrégée ; utile pour les données administratives.
  \item \textbf{Azure Maps} : propose une cartographie vectorielle moderne avec gestion de clusters et carte thermique.
  \item \textbf{ArcGIS for Power BI} : permet des fonctions spatiales avancées (zones isochrones, couches socio-démographiques), avec certaines restrictions sans compte Esri.
\end{itemize}

\subsubsection{Cartes de KPI et jauges}
\begin{itemize}
  \item \textbf{Carte (single)} : affiche une métrique unique, très lisible sur petits écrans.
  \item \textbf{Carte multi-lignes} : combine plusieurs KPIs dans un seul composant.
  \item \textbf{KPI} : présente une valeur, une cible et une tendance ; bien adapté au suivi d’objectifs.
  \item \textbf{Jauge} : visualisation qualitative d’une performance vs objectif ; limitée à un usage mono-métrique.
\end{itemize}

\subsubsection{Tableaux et matrices}
Les tableaux affichent les données ligne à ligne ; les matrices pivotent les dimensions et permettent les totaux intermédiaires. Pour des rapports très formatés, Power BI propose \emph{le rapport paginé} (RDL), utile dans un contexte administratif.

\subsubsection{Filtres interactifs}
\textbf{Segments (slicers)} : composants filtrants par catégorie, plage ou date. Le segment bouton améliore l’ergonomie sur mobile et permet une navigation par clic visuel.

\subsubsection{Visuels d'intelligence artificielle}
\begin{itemize}
  \item \textbf{Influenceurs clés} : détecte les dimensions explicatives d’une mesure cible.
  \item \textbf{Arborescence de décomposition} : propose des détails successifs de manière semi-automatisée.
  \item \textbf{Q\&A} : traduit une requête en langage naturel en visuel dynamique.
  \item \textbf{Narratif intelligent} : génère du texte automatisé en fonction des filtres actifs.
\end{itemize}

\subsubsection{Intégrations actionnables}
\begin{itemize}
  \item \textbf{Power Apps visual} : intègre une app canvas interactive pour actions en base (formulaires, validation).
  \item \textbf{Power Automate visual} : ajoute un bouton qui lance un processus automatisé (email, mise à jour CRM).
\end{itemize}

\subsection*{Analyse critique}

\paragraph{Points forts}
\begin{itemize}
  \item Performance optimale : chargement rapide, compatibilité multi-plateforme.
  \item Maintenance assurée : mise à jour automatique et fiabilité.
  \item Accessibilité native : compatibilité avec les lecteurs d'écran et thèmes d'entreprise.
  \item Intégration simple : aucun développement requis.
\end{itemize}

\paragraph{Limitations clés}
\begin{itemize}
  \item Faible personnalisation : interface de configuration figée.
  \item Interactivité limitée à ce que prévoit Microsoft ; pas de logique conditionnelle.
  \item Structure tabulaire obligatoire : peu adapté aux modèles de graphes, répétitions ou réseaux.
  \item Pas d’accès au DOM, au code ou aux \texttt{event listeners}.
\end{itemize}

\subsection*{Conclusion intermédiaire}
Les visuels natifs forment un socle de démarrage fiable, rapide et adapté à 80~\% des besoins analytiques en entreprise. Cependant, leur manque de souplesse les rend insuffisants pour des besoins de personnalisation graphique avancée, d’interactivité dynamique ou d’intégration à des architectures complexes. C’est précisément dans ces cas que les visuels personnalisés (Python, R, SDK) prennent toute leur valeur.


% 2.3 Visuels scriptés Python / R
%-----------------------------------------------------------
\section{Visuels Python/R : usages, atouts, limites}
\label{sec:python-r-visuals}
%-----------------------------------------------------------

En plus des visuels préfabriqués, Power~BI permet d’incorporer des scripts Python ou R pour générer des visuels sur mesure.  
Concrètement, l’utilisateur peut ajouter un élément de type ``Python Visual'' ou ``R Visual'' dans un rapport, puis fournir un script dans l’éditeur associé.  
Le moteur Power~BI va alors exécuter ce script en coulisse, en lui transmettant les données du modèle (colonnes et mesures sélectionnées) sous forme de \textit{dataframe}, et récupérer en résultat un graphique statique produit par le code Python ou R.

Cette fonctionnalité vise à tirer parti de l’écosystème analytique de ces langages~: il devient possible d’utiliser des bibliothèques populaires comme \textit{Matplotlib}, \textit{Seaborn} ou \textit{Plotly} en Python, ou \textit{ggplot2}, \textit{plotly R} en R, afin de créer des visualisations avancées non disponibles nativement.  
Par exemple, un data scientist peut réaliser dans Power~BI un nuage de points avec une régression LOESS en R, ou un diagramme de réseau en Python, en quelques lignes de code utilisant des packages spécialisés — des visuels difficiles à obtenir autrement.

Les atouts de cette approche sont liés à la puissance et à la flexibilité des langages R et Python en matière de visualisation et de calculs statistiques.  
On bénéficie de l’énorme bibliothèque de packages open-source~: analyses statistiques poussées, machine learning, visualisations scientifiques (heatmaps, dendrogrammes, etc.), tout peut théoriquement être intégré dans un rapport.  
De plus, ces visuels scriptés permettent d’automatiser des traitements de données directement avant l’affichage.  
Par exemple, on peut programmer un script Python qui agrège des données, calcule des indicateurs personnalisés ou entraîne un modèle prédictif, puis affiche un graphique du résultat — le tout s’exécutant dynamiquement lors du rafraîchissement du visuel.  
Cela étend les capacités de Power~BI au-delà du seul langage DAX, en offrant la richesse de Python/R pour des besoins spécifiques (analyses de texte, séries temporelles avancées, etc.).

Cependant, les visuels Python/R présentent des limitations importantes dues à leur nature même de scripts externes.  
D’abord, le rendu produit est une image statique (format PNG) intégrée dans le rapport.  
Ainsi, «~les visualisations Python dans Power BI ne sont que des images statiques (résolution 72 DPI), sans aucune interactivité~»\parencite{RealPythonPowerBI2023}.  
Concrètement, l’utilisateur ne peut pas cliquer sur un élément du graphique Python/R pour filtrer d’autres visuels — toute la surface du visuel est une image plane.  
Il y a bien une interaction partielle~: si l’on applique un filtre ou sélectionne un élément sur un autre visuel du rapport, Power~BI ré-exécute le script Python/R avec les données filtrées, ce qui met à jour l’image correspondante\parencite{RealPythonPowerBI2023}.  
Mais cela reste plus lent et moins fluide que les visuels natifs, car il faut relancer le calcul du script à chaque changement.

Microsoft documente que ces visuels scriptés «~se rafraîchissent lors des mises à jour ou filtrages des données, mais l’image en elle-même n’est pas interactive~»\parencite{MicrosoftPythonRVisualsDocs2024}.  
Ils répondent aux \textit{highlightings} provenant d’autres visuels, «~mais on ne peut pas sélectionner des éléments du visuel Python/R pour croiser le filtre~»\parencite{MicrosoftPythonRVisualsDocs2024}.  
Cette absence d’interactivité locale est un frein dans un tableau de bord où l’on attend généralement de pouvoir explorer les données de façon interactive.

Une autre contrainte forte est la taille des données transmises aux scripts.  
Pour éviter des temps d’exécution excessifs, Power~BI limite à 150\,000 lignes le volume de données qu’un visuel Python ou R peut traiter.  
«~Si plus de 150\,000 lignes sont sélectionnées, seules les 150\,000 premières sont utilisées~», et un message d’avertissement s’affiche sur l’image\parencite{MicrosoftPythonRVisualsDocs2024}.  
De plus, l’ensemble des données en entrée du script ne doit pas excéder 250~MB en mémoire\parencite{MicrosoftPythonRVisualsDocs2024}.  
Ces seuils signifient que les visuels Python/R sont adaptés à des échantillons de données ou des agrégats, mais pas au traitement de données massives brutes.  
Au-delà, il faut compter sur le modèle tabulaire de Power~BI pour pré-agréger ou filtrer avant d’envoyer au script.

En termes de performance, il existe également un \textit{timeout} de 5~minutes~: si le script met plus de 5~minutes à s’exécuter, il sera interrompu et le visuel affichera une erreur\parencite{MicrosoftPythonRVisualsDocs2024}.  
Cela empêche l’utilisation de calculs trop longs.  
Par exemple, entraîner un modèle complexe de machine learning sur un jeu de données volumineux dépasserait ce temps imparti.

Il convient aussi de noter des limitations techniques supplémentaires~: les chaînes de caractères de plus de 32\,766 caractères sont tronquées lors de la transmission au \textit{dataframe}\parencite{MicrosoftPythonRVisualsDocs2024} ;  
certains appareils ou environnements cloud ne supportent pas ces visuels (par exemple, les rapports publiés via \textit{publish to web} ne peuvent pas afficher de visuels Python/R pour des raisons de sécurité)\parencite{MicrosoftPythonRVisualsDocs2024}.  
De plus, pour utiliser un visuel Python ou R en Power~BI Desktop, il faut que l’utilisateur ait installé localement l’interpréteur Python ou R, ainsi que les packages nécessaires.  
Sur le service Power~BI en ligne, l’exécution des scripts est prise en charge côté serveur, mais seulement pour les tenants disposant de capacités Premium ou pour les utilisateurs Pro — ce qui signifie qu’un utilisateur Free ne pourra pas voir un visuel Python/R dans un rapport à moins que celui-ci soit hébergé sur un \textit{workspace Premium}\parencite{MicrosoftPythonRVisualsDocs2024}.

Enfin, du point de vue sécurité, Power~BI traite les visuels Python/R comme du code potentiellement risqué.  
Lorsqu’on ajoute pour la première fois un tel visuel, un avertissement de sécurité apparaît, invitant l’utilisateur à n’exécuter que des scripts de source fiable\parencite{MicrosoftPythonRVisualsDocs2024}.  
En entreprise, l’utilisation de ces visuels peut soulever des enjeux de validation du code et de maintenance (le script étant incorporé dans le rapport, il doit être maintenu manuellement en cas de mise à jour de librairie, etc.).

\textbf{Bilan.}  
Les visuels Python et R offrent donc une solution d’appoint puissante pour réaliser des visualisations avancées ou des analyses spécifiques au sein de Power~BI, sans avoir à développer un visuel \textit{custom} complet.  
Ils comblent certaines lacunes des visuels natifs en ouvrant la porte à la riche panoplie des bibliothèques \textit{data science}.  
Cependant, ils doivent être utilisés en connaissance de leurs limites~: performances réduites, absence d’interactivité directe, et restrictions d’usage dans l’environnement de service Power~BI.  
Pour un besoin de visualisation récurrent, à destination d’un large public, ou nécessitant une expérience utilisateur interactive, il peut être préférable de développer un visuel personnalisé via le SDK (ou d’utiliser un visuel custom existant sur AppSource) plutôt que de s’appuyer sur un script Python/R intégré.  
C’est ce que nous examinons dans la section suivante.


% 2.4 SDK Power BI : structure, sécurité, pipeline
%-----------------------------------------------------------
\section{SDK Custom Visuals : principes, sécurité, pipeline}
\label{sec:sdk}
%-----------------------------------------------------------

Lorsque les visuels natifs ne suffisent plus et qu’un script Python ou R se révèle trop limité, la solution la plus aboutie consiste à créer un visuel complet à l’aide du Software Development Kit (SDK) de Power BI. Microsoft distribue ce SDK sous la forme du package npm powerbi-visuals-tools, dont la version stable 6.1.2 publiée le 26 mai 2025 sert de référence pour ce travail \parencite{MicrosoftSDKNpm2025}.  

\subsection{Principe général.} Un visuel Power BI est, en substance, une mini-application web encapsulée. Le développeur décrit, dans un fichier capabilities.json, les champs de données et les options de format qu’il souhaite exposer, puis implémente en TypeScript une classe qui respecte l’interface IVisual. Power BI invoque la méthode update(options) à chaque changement de filtre ou de données ; le code traduit alors les informations reçues en éléments DOM (SVG ou Canvas) à l’aide, par exemple, de D3.js.  

\subsection{Pipeline de développement}\label{sec:pipeline-ref}

Le Power BI Custom Visuals SDK prescrit un flux de travail de référence :  
pbiviz new <NomDuVisuel> crée le squelette (manifeste, capabilities.json, fichiers TypeScript),  
puis pbiviz package --certification-audit génère l’archive .pbiviz tout en exécutant les contrôles réseau et mémoire recommandés \parencite{MicrosoftAuditCLI2025}.  
Depuis la dépréciation du serveur de développement intégré (2024), les tests interactifs s’effectuent dans un workspace développeur Power BI Service, avant qu’un package conforme ne soit importé dans un rapport ou soumis à AppSource pour certification \parencite{MicrosoftCertificationGuide2025}.

\subsection{Cadre de sécurité.} Le code exécuté l’est toujours dans une iframe sandbox dont la politique de contenu interdit tout appel réseau non approuvé et toute évaluation dynamique de code (eval, new Function) ; le SDK force également le dessin à l’intérieur de la bounding box du visuel et n’expose au script que les champs explicitement liés par l’utilisateur \parencite{OkVizSandbox2022}. Les règles de certification 2025 précisent qu’un composant destiné à AppSource doit renoncer à tout flux sortant \parencite{MicrosoftAPIv6CSP2025}.  

\subsection{Certification et limitations associées}\label{sec:certification}

La certification « Power BI Certified », réservée aux visuels publiés sur AppSource, conditionne l’inclusion du composant dans les exports PDF/PowerPoint, son rendu dans les rapports distribués par courriel et l’accès aux privileged APIs telles que FileDownload ou Licensing \parencite{MicrosoftCustomVisualsCertified2025, MicrosoftFileDownloadAPI2024}. Un visuel non certifié est exclu de ces scénarios et peut être bloqué si l’administrateur active l’option locataire « Add and use certified visuals only » \parencite{MicrosoftTenantSettings2024}.

\subsection{Synthèse.} Le SDK ouvre la voie à des composants strictement sur mesure qui héritent de la totalité des interactions offertes par Power BI ; l’effort de développement et la discipline de gouvernance qu’il requiert constituent la contrepartie logique de cette flexibilité. La section \ref{sec:techno} précise les choix d’outillage — TypeScript, D3.js et React optionnel — retenus pour réaliser les preuves de concept développées dans ce mémoire.


% 2.5 Choix technologiques (TypeScript, D3, React ?)
\section{Choix technologiques : TypeScript, D3, (option React)}
\label{sec:choix-technologiques}

Créer un visuel Power BI custom revient essentiellement à développer une application web monopage embarquée. À ce titre, le choix des technologies de développement est déterminant pour réussir à la fois l’implémentation technique et la maintenabilité du code.

À l’heure actuelle (2023–2025), un consensus s’est formé autour du trio technologique suivant pour les visuels Power BI : \textbf{TypeScript} comme langage de développement, \textbf{D3.js} comme bibliothèque de visualisation de bas niveau, et éventuellement \textbf{React} (ou un autre framework UI) pour structurer l’interface si nécessaire. Ces choix ne sont pas exclusifs – en principe, le SDK permet l’usage de n’importe quel framework JavaScript du moment qu’il peut s’intégrer dans le bundle en module ES6\footnote{\url{https://www.reddit.com/r/PowerBI/comments/12ggh93/is_it_possible_to_use_react_with_powerbi_custom/}} – mais ils reflètent les recommandations et pratiques courantes dans la communauté.

\subsection{TypeScript comme langage par défaut}

Le SDK Power BI est conçu pour TypeScript, un sur-ensemble typé de JavaScript. Lorsqu’on initialise un projet de visuel, les fichiers générés (ex. \texttt{visual.ts}, \texttt{settings.ts}) sont en TypeScript\footnote{\url{https://learn.microsoft.com/en-us/power-bi/developer/visuals/custom-visual-develop-tutorial}}.

L’adoption de TypeScript présente plusieurs avantages :
\begin{itemize}
  \item (a) Typage statique permettant d’attraper des erreurs à la compilation et de mieux documenter les intentions du code, ce qui est précieux pour des visuels complexes ;
  \item (b) Meilleure productivité grâce à l’autocomplétion et la documentation intégrée dans les IDE (Microsoft fournit les définitions d’interfaces du Power BI Visuals API en TypeScript) ;
  \item (c) Interopérabilité avec JavaScript – TypeScript compile en JavaScript standard, ce qui permet d’utiliser toute librairie JS existante.
\end{itemize}

En somme, TypeScript est devenu le langage de facto pour ce genre de développement front-end structuré. D’après les experts, « pour bien faire les choses, il faut utiliser le SDK TypeScript, qui donne accès à toutes les APIs fournies par Microsoft »\footnote{\url{https://www.reddit.com/r/PowerBI/comments/12ggh93/is_it_possible_to_use_react_with_powerbi_custom/}}.

Un développeur purement JavaScript pourrait en théorie coder un visuel, mais il perdrait les bénéfices du typage et devrait écrire des bindings à la main pour l’API – d’où l’intérêt de se conformer à l’outil TypeScript officiel.

\subsection{D3.js pour le rendu visuel}

D3.js (Data-Driven Documents) est une bibliothèque JavaScript largement utilisée pour manipuler le DOM en fonction des données, en particulier via SVG, pour créer des visualisations personnalisées. Microsoft elle-même a misé sur D3.js dès l’origine des custom visuals : « en s’appuyant sur des librairies open-source comme D3.js, nous avons rendu la création de nouveaux visuels incroyablement simple »\footnote{\url{https://powerbi.microsoft.com/fr-fr/blog/announcing-the-power-bi-custom-visuals-contest/}}.

La majorité des visuels personnalisés (et même plusieurs visuels natifs sous le capot) utilisent D3 pour générer les éléments graphiques (formes, échelles, axes...).

D3 offre un contrôle très fin sur le rendu : on peut construire quasiment n’importe quel type de visualisation en partant de primitives de base (lignes, cercles, chemins SVG) liées aux données par des bindings data$\rightarrow$DOM. Par exemple, pour un graphique en barres custom, D3 facilite la création des \texttt{<rect>} dimensionnés selon les valeurs des données et applique des échelles et axes avec quelques fonctions.

Un avantage clé est que D3 est agnostique : on peut l’utiliser dans le sandbox sans souci, et il n’a pas de dépendances externes. En fait, Power BI intègre une prise en charge particulière de D3 : il a été noté que « Power BI attache D3 au contexte global de la sandbox, ce qui le rend immédiatement disponible »\footnote{\url{https://community.fabric.microsoft.com/t5/Developer/React-js-not-working-properly-in-custom-visual/td-p/3111857}}.

Cela signifie que D3 s’exécute aisément dans le cadre isolé, alors que d’autres librairies nécessitent parfois des ajustements (voir paragraphe sur React). D3 est donc hautement recommandé pour quiconque crée un visuel custom du fait de sa puissance et de sa souplesse.

Bien sûr, son utilisation requiert une certaine expertise (ce n’est pas une librairie de chart “clé en main” mais un outil bas niveau). Alternativement, certains développeurs optent pour des librairies plus haut niveau (par ex. Chart.js, Plotly.js, ou Vega-Lite).

Celles-ci peuvent être utilisées via le SDK, mais elles offrent moins de contrôle et parfois posent des problèmes de taille de bundle ou de compatibilité. D3 reste le standard de facto, au point que même des solutions no-code comme Deneb (un visuel custom permettant de configurer des graphiques via Vega-Lite) utilisent D3 en interne pour le rendu.

\subsection{React (optionnel) pour la structure UI}

L’utilisation de React – ou d’un autre framework comme Angular, Vue – dans un visuel Power BI est possible, bien que non indispensable dans tous les cas. Un visuel Power BI est par nature assez ciblé (il rend une visualisation spécifique dans un espace contraint).

Beaucoup de visuels custom n’emploient donc pas de framework UI sophistiqué : le code TypeScript avec D3 suffit à créer le SVG ou canvas voulu.

Néanmoins, React peut s’avérer utile dans plusieurs scénarios :
\begin{itemize}
  \item (a) Visuels complexes avec état interne – par exemple un visuel qui propose une petite interface utilisateur (boutons, menus) en plus du graphique. React aide à organiser le code en composants et à gérer les interactions utilisateur de manière déclarative.
  \item (b) Réutilisation de composants existants – un développeur peut intégrer des librairies React de composants (par ex. un sélecteur de couleur, une timeline interactive) au lieu de tout coder from scratch.
  \item (c) Faciliter le développement – pour des développeurs déjà familiers avec React, il peut être plus confortable de raisonner en composant React et virtual DOM qu’en manipulations directes du DOM.
\end{itemize}

Microsoft a d’ailleurs publié un tutoriel officiel montrant pas à pas la création d’un visuel custom en React (un “circle card” affichant un KPI formaté)\footnote{\url{https://learn.microsoft.com/en-us/power-bi/developer/visuals/custom-visual-develop-react-circle-card}}.

Cela dit, intégrer React dans un visuel custom introduit de la complexité. D’une part, cela alourdit le package car il faut inclure React (et possiblement ReactDOM) dans le bundle du visuel – sauf si l’on s’appuie sur la version déjà incluse par Power BI (non garantie dans le sandbox).

D’autre part, du fait du sandboxing, le \texttt{window} global utilisé par React n’est pas le même contexte que celui du host Power BI. Des développeurs ont constaté qu’il fallait ruser pour faire fonctionner React, par exemple en sauvegardant une référence du \texttt{window} isolé pour l’utiliser lors du rendu, car « lorsque React est chargé, il est attaché à un objet \texttt{window} qui est ensuite modifié par le sandbox, le rendant indisponible lors de l’appel \texttt{update} »\footnote{\url{https://community.fabric.microsoft.com/t5/Developer/React-js-not-working-properly-in-custom-visual/td-p/3111857}}.

Un ingénieur Microsoft a expliqué à ce sujet que « Power BI crée une copie de l’objet \texttt{window} pour isoler les librairies du visuel de celles de l’application ; D3 est relié correctement au contexte global, mais React ne l’était pas, nécessitant ce contournement »\footnote{\url{https://community.fabric.microsoft.com/t5/Developer/React-js-not-working-properly-in-custom-visual/td-p/3111857}}.

Il a suggéré qu’à terme l’isolation pourrait être assouplie. Il s’agit de détails techniques, mais qui illustrent que toute librairie externe doit être testée pour fonctionner dans l’environnement particulier de Power BI.

\subsection{Autres choix et écosystème}

À noter que TypeScript/D3/React ne sont pas les seules technologies possibles, mais représentent un socle éprouvé. Certains visuels spécifiques intègrent par exemple Mapbox (pour la cartographie), Three.js (pour de la 3D), ou d’autres libs selon le besoin – ce qui est faisable tant que cela respecte les contraintes du sandbox et du bundle.

Par ailleurs, pour les utilisateurs non développeurs qui veulent tout de même des visuels sur mesure, des outils comme Charticulator (de Microsoft Research) ou Deneb (basé sur Vega-Lite) permettent de créer des visuels custom sans écrire du code, en restant dans les limites du sandbox et avec un résultat intégrable comme un visuel custom.

Ces alternatives confirment qu’au cœur, le moteur de rendu reste du JS/TS manipulant le DOM SVG/HTML.

\subsection{Choix technologique pour ECRINS SA}

Dans le contexte du projet de bachelor « Création de composants BI custom dans Power BI » pour ECRINS SA, le choix de TypeScript et D3.js apparaît donc incontournable pour développer de nouveaux visuels performants et maintenables.

TypeScript assurera la robustesse du code et la compatibilité avec l’API Power BI, tandis que D3.js fournira la boîte à outils nécessaire pour représenter visuellement les données de la manière la plus flexible possible.

L’adoption de React dépendra de la complexité des visuels à créer : si les cas d’usage exigent des interactions UI poussées ou si l’équipe de développement maîtrise déjà React, il pourrait être judicieux de l’utiliser pour structurer l’application visuelle. Sinon, on pourra s’en passer afin de minimiser la complexité.

Quoiqu’il en soit, le respect des bonnes pratiques de sécurité (pas de dépendances externes non contrôlées, respect du sandbox) et d’optimisation (volume de données raisonnable, rendu efficace) guidera le développement. En s’appuyant sur ce trio technologique moderne, l’entreprise disposera d’une base solide pour étendre les capacités de Power BI avec des visuels sur mesure adaptés à ses besoins, tout en s’alignant sur l’état de l’art actuel en matière de visualisation de données interactive.


% =============================================================
