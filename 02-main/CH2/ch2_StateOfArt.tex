% =============================================================
\chapter{État de l'art}
\label{chap:state-of-the-art}
\selectlanguage{french}
\setlength{\parindent}{0pt}


Le développement de visuels personnalisés dans Power~BI s’appuie sur un
écosystème technologique riche, combinant les fonctionnalités \emph{natives}
de la plateforme et des extensions via code.  
Ce chapitre présente d’abord l’\textbf{architecture} des visuels Power~BI et
distingue les différentes catégories de visuels disponibles.  
Ensuite, il examine successivement les \textbf{visuels natifs}
(ceux fournis par défaut par Microsoft), les \textbf{visuels basés sur
Python/R} (générés à partir de scripts), et enfin les \textbf{visuels
personnalisés via le SDK} officiel.  
Chaque section discute des capacités offertes et des limites inhérentes.  
Enfin, nous abordons les \textbf{choix technologiques} pour la création de
nouveaux visuels (langage TypeScript, bibliothèque~D3.js, usage éventuel de
React), en justifiant ces choix dans le contexte actuel.  
L’objectif est d’établir l’état de l’art des technologies de visualisation de
données dans Power~BI, tout en adoptant un regard critique sur leurs forces et
faiblesses respectives.


% -----------------------------------------------------------------
% 2.1 Concepts BI et datavisualisation
\section{Concepts BI et datavisualisation}
\label{sec:concepts-bi-dataviz}

La \textit{Business Intelligence} (\textbf{BI}) peut se définir comme l’ensemble des méthodes et technologies visant à transformer des données brutes en connaissances utiles pour la prise de décision organisationnelle. Chen, Chiang et Storey~\parencite{Chen2012} rappellent que la valeur de la BI réside moins dans l’acquisition massive de données que dans la capacité à les modéliser, les analyser et les représenter de manière intelligible pour l’humain. L’étape de visualisation constitue ainsi le dernier maillon du pipeline « ingestion–modélisation–analyse–présentation », mais elle s’avère décisive pour convertir des métriques abstraites en informations actionnables. C’est précisément à ce niveau que se situent les visuels Power BI, natifs ou personnalisés, objets d’étude du présent travail.

Dans le domaine de la \textit{datavisualisation}, les sciences cognitives ont montré que l’œil humain perçoit rapidement certains attributs dits \textit{préattentifs} : position, longueur, orientation ou couleur, entre autres. Ware~\parencite{Ware2019} démontre que l’exploitation adéquate de ces attributs maximise la vitesse et la justesse de la lecture visuelle. Tufte~\parencite{Tufte1983} a, pour sa part, popularisé l’idée de \textit{data–ink ratio}, soulignant que le graphisme ne doit conserver que l’encre strictement indispensable au message ; tout élément décoratif superflu — le \textit{chart-junk} — nuit à la clarté. Few~\parencite{Few2009} prolonge cette perspective en montrant que la cohérence des encodages visuels (axes, couleurs, échelles) constitue une condition essentielle pour comparer de façon fiable plusieurs séries de données.

L’unification théorique de ces principes a été proposée par Wilkinson \parencite{Wilkinson2005} puis formalisée dans la communauté statistique par Wickham sous le nom de \textit{Grammaire des graphiques}. Le modèle décrit chaque graphique comme la combinaison déclarative de couches : données, transformations, géométries, échelles, systèmes de coordonnées et facettage. Ce cadre a influencé la plupart des bibliothèques modernes — notamment D3.js, Vega ou ggplot2 — et se retrouve implicitement dans l’API de Power BI ; chaque visuel y spécifie ses champs (data roles), ses encodages (capabilities) et son canevas de rendu.

Au-delà des principes, la BI professionnelle introduit des impératifs supplémentaires : performance d’affichage, accessibilité numérique, conformité réglementaire (RGPD) et gouvernance des artefacts analytiques. Les visuels standards de Power BI satisfont ces exigences pour des cas courants, mais ils se heurtent aux demandes spécifiques de certains métiers ; c’est autour de ces limites qu’émerge le besoin de visuels personnalisés. Comprendre les fondements de la BI et de la datavisualisation éclaire ainsi la double problématique du mémoire : confirmer la pertinence d’enrichir Power BI par de nouveaux composants, puis garantir que ces composants respectent les bonnes pratiques cognitives tout en s’intégrant dans un environnement d’entreprise contrôlé.


% 2.2 Architecture des visuels Power BI
\section{Architecture des visuels Power BI}
\label{sec:archi-powerbi}

Power~BI est conçu autour d’une  {architecture de visualisation ouverte
et extensible}.  
Chaque élément visuel (graphique, carte, jauge, etc.) est rendu côté client
à partir des données du modèle, via du code JavaScript/TypeScript exécuté
dans Power~BI Desktop ou dans le service web \parencite{MicrosoftOpenVis2015}.  
Depuis 2015, Microsoft propose non seulement une panoplie de visuels
« \emph{core} » (natifs), mais permet aussi l’importation de visuels
additionnels développés par la communauté ou des éditeurs tiers
\parencite{MicrosoftMarketplace2016}.  
Cette ouverture repose sur des standards web : \emph{« en s’appuyant sur des
standards ouverts d’Internet et des bibliothèques open-source comme D3.js »},
la création de visuels personnalisés a été grandement simplifiée
\parencite{MicrosoftD3Blog2017}.  
Microsoft publie d’ailleurs le code source de nombreux visuels natifs sur
GitHub, attestant de sa volonté d’encourager un écosystème ouvert
\parencite{GitHubPowerBISamples2024}.  
 {L’API publique est passée en version 5.10, puis en préversion 6
(22 juillet 2024), introduisant un DOM sécurisé et la \textit{Rendering Events API}, ce qui
renforce la cohérence entre ouverture et sécurité} \parencite{MicrosoftApiChangelog2024}.

%-----------------------------------------------------------
\subsection{Visual container et bac à sable}
\label{subsec:sandbox}
%-----------------------------------------------------------

Qu’il soit natif ou personnalisé, un visuel s’insère dans le \emph{canevas}
du rapport et interagit avec le modèle de données via des rôles prédéfinis.
Chaque visuel reçoit, du moteur Power BI, les données filtrées qui lui sont
attribuées (colonnes, mesures, hiérarchies), puis exécute son propre code de
rendu.  
Pour les visuels \emph{custom}, ce code est empaqueté dans un fichier
\texttt{.pbiviz} contenant les scripts, les styles et le manifeste
\parencite{MicrosoftPbivizDocs2023}.  
Power BI exécute alors le visuel dans un  {bac à sable sécurisé}
(\emph{sandbox}) : une \texttt{iframe} isolée du reste du rapport
\parencite{OkVizSandbox2022}.  
Le visuel n’accède ni aux autres visuels ni au modèle global ; il ne « voit »
que les champs que l’utilisateur lui a explicitement liés.  
{Depuis la mise à jour 2.140 (février 2024), tout visuel — qu’il soit
privé (\textit{organizational visual}) ou destiné à AppSource — est audité par
l’option \texttt{pbiviz package --certification-audit} de
\textit{powerbi-visuals-tools} ≥ 6.1 : les appels réseau
(\texttt{fetch}, \texttt{XMLHttpRequest}, WebSockets) et l’évaluation dynamique
de code (\texttt{eval}, \texttt{Function}) sont bloqués, et l’exécution est
interrompue au-delà d’environ 120 s de CPU ou de 230 Mio de mémoire%
\parencite{MicrosoftCertificationGuide2025}.  
Ces garde-fous empêchent qu’un code malveillant puisse lire ou exfiltrer des données sans autorisation \parencite{MediumSecurityPBI2023}.

%-----------------------------------------------------------
\subsection{Interactions et intégration}
\label{subsec:interactions}
%-----------------------------------------------------------

Malgré cet isolement technique, les visuels s’intègrent pleinement dans
l’expérience interactive globale.  
Un visuel personnalisé correctement développé se comporte \emph{exactement
comme un visuel natif} : il réagit aux filtres, autorise le
\textit{cross-highlight} (mise en surbrillance croisée) et expose des options
de mise en forme dans le panneau \emph{Format}
\parencite{MicrosoftCustomVisGuide2024}.  
Lorsqu’un utilisateur clique, par exemple, sur une barre d’histogramme, le
moteur Power BI propage l’événement de sélection aux autres visuels.
Si le développeur a implémenté l’API \texttt{ISelectionManager}, son visuel
peut émettre et recevoir ces événements ;  {il peut également recourir
à \texttt{ITooltipService} pour les infobulles contextuelles ou à
\texttt{ILocalizationManager} pour l’internationalisation, de sorte que
l’intégration fonctionnelle et linguistique demeure uniforme}
\parencite{MicrosoftSelectionAPI2024,MicrosoftTooltipAPI2024}.  

La différence fondamentale reste donc interne : les visuels natifs font
partie du produit et peuvent exploiter des API internes non exposées,
tandis que les visuels personnalisés s’appuient uniquement sur
l’API publique du SDK, avec les restrictions de sécurité détaillées en
section~\ref{sec:sdk}.  
{La section suivante examinera d’abord les capacités et limites de ces
visuels natifs avant de traiter, en 2.3, l’approche Python/R, puis, en 2.4,
le développement complet via le SDK.}


% 2.3 Visuels scriptés Python / R
%-----------------------------------------------------------
\section{Visuels Python / R : usages, atouts, limites}
\label{sec:python-r-visuals}
%-----------------------------------------------------------

Outre les visuels préfabriqués, Power BI autorise l’exécution de scripts
Python ou R pour produire des visuels sur mesure.  
L’utilisateur place un composant \enquote{Python visual} (ou \enquote{R visual})
dans le rapport, saisit son code dans l’éditeur, puis Power BI exécute ce
script en tâche de fond : les données liées sont transmises sous forme de
\textit{dataframe} et le résultat retourné est une image statique (PNG)
affichée dans le canevas.

Cette fonctionnalité exploite l’écosystème analytique des deux langages :
bibliothèques \textit{Matplotlib}, \textit{Seaborn} ou \textit{Plotly} côté
Python ; \textit{ggplot2} ou \textit{plotly R} côté R.  
Un data-scientist peut ainsi tracer dans Power BI un nuage de points avec
régression \textsc{Loess} (R) ou un diagramme de réseau (Python) en quelques
lignes, visuels impossibles à obtenir via les graphes natifs.


\textbf{Atouts.}  
La puissance réside dans la bibliothèque de packages open-source :
statistiques avancées, machine learning, heatmaps, dendrogrammes, etc.
Les scripts peuvent en outre pré-traiter les données (agrégation, calcul
d’indicateurs, entraînement d’un modèle) avant de dessiner le graphique ;
l’opération se relance automatiquement lorsque le visuel est rafraîchi,
offrant au passage des capacités qu’un simple DAX ne couvre pas (analyse de
texte, séries temporelles complexes).


\textbf{Limites techniques et fonctionnelles.}  
Le rendu reste une image statique : \enquote{les visualisations Python dans
Power BI sont des bitmaps 72 DPI, sans interactivité directe}
\parencite{MicrosoftPythonRVisualsDocs2024}.  
L’utilisateur ne peut donc pas cliquer dans le graphique pour filtrer les
autres visuels ; seul un nouveau filtrage externe provoque la ré-exécution
du script, qui demeure plus lente qu’un visuel natif.

Power BI encadre par ailleurs la quantité de données :  
\emph{(i)} 150 000 lignes et 100 colonnes au maximum sont transmises au
script ; au-delà, seules les premières 150 000 lignes sont prises en compte,
un avertissement s’affichant sur l’image ;  
\emph{(ii)} la taille totale ne doit pas excéder 250 MB en mémoire ;  
\emph{(iii)} un \textit{timeout} de cinq minutes interrompt tout script trop
long\parencite{MicrosoftPythonRVisualsDocs2024}\emph{, mais ce délai est ramené à
une minute lorsque l’exécution se fait dans le service Power BI}
\parencite{MicrosoftRPackagesService2025} ;
\emph{(iv)} pour les \textit{R visuals}, la taille du PNG de sortie est
désormais plafonnée à 2 MB\parencite{MicrosoftRVisualsDocs2025}.   
Ces garde-fous rendent les visuels Python/R adaptés à des échantillons ou à
des agrégats, mais inadaptés à un traitement massif ou à un apprentissage
profond.

D’autres restrictions s’appliquent : les chaînes de plus de 32 766 caractères
sont tronquées ; les visuels ne s’affichent ni dans \textit{Publish to Web}
ni dans certains environnements mobiles pour raisons de sécurité ;
\textbf{le runtime R 4.3.3 dans le service limite en outre le payload
compressé à 30 MB}\parencite{MicrosoftRPackagesService2025} ; enfin,
l’interpréteur Python ou R et les packages requis doivent être installés
localement pour Power BI Desktop, tandis que le service en ligne n’exécute
les scripts que sur un workspace Premium ou pour des comptes Pro,  
\textbf;{ce qui signifie qu’un compte Free n’affichera pas ces visuels}
\parencite{MicrosoftPythonRVisualsDocs2024}. % ← précision licence


\textbf{Sécurité et maintenance.}  
Lors de l’insertion d’un premier visuel Python / R, Power BI émet un
avertissement invitant à exécuter uniquement un code de source fiable
\parencite{MicrosoftPythonRVisualsDocs2024}.  
Dans un contexte d’entreprise, la gouvernance du code devient critique :
le script est embarqué dans le fichier \texttt{.pbix}, hors de tout
versionnage Git natif ; il faut donc organiser une validation et un cycle de
mise à jour des dépendances.


\textbf{Bilan.}  
Les visuels Python et R constituent une solution d’appoint puissante pour
des analyses ad hoc ou spécialisées, en comblant certaines lacunes des
visuels natifs grâce à l’arsenal \textit{data-science}.  
Ils doivent toutefois être utilisés en connaissance de leurs limites :
image statique, performance conditionnée par les seuils de lignes et
colonne, exigences Premium dans le service, et effort de maintenance du
code.  
Dès qu’un visuel doit être interactif, largement partagé ou intégré au
catalogue interne, le développement d’un visuel personnalisé via le SDK —
thème de la section \ref{sec:sdk} — devient souvent la voie la plus pérenne.


% 2.4 SDK Power BI : structure, sécurité, pipeline
%-----------------------------------------------------------
\section{SDK Custom Visuals : principes, sécurité, pipeline}
\label{sec:sdk}
%-----------------------------------------------------------

Lorsque les visuels natifs ne suffisent pas et qu’un script Python/R est trop limité, la solution la plus aboutie est de créer un visuel personnalisé complet en utilisant le Software Development Kit (SDK) de Power BI. Microsoft fournit un SDK officiel (sous forme de Node Package \texttt{powerbi-visuals-tools} \textbf{(v 6.3, mai 2025)}) qui permet aux développeurs de coder, tester et empaqueter de nouveaux visuels sous forme d’un fichier \texttt{.pbiviz}\parencite{MicrosoftSDKNpm2025}.

Le principe fondamental est qu’un visuel Power BI n’est rien d’autre qu’une application web encapsulée : il contient du code HTML/JavaScript (plus précisément TypeScript, transcompilé en JavaScript) qui reçoit des données et génère du DOM (SVG, Canvas, etc.) pour afficher un graphique. Le SDK abstrait la communication avec Power BI : le développeur définit, via un fichier \texttt{capabilities.json}, quels champs et paramètres de formatage son visuel acceptera, puis implémente une classe TypeScript qui reprend l’interface \texttt{IVisual}. Cette classe comporte notamment une méthode clé \texttt{update(options)} qui est appelée par Power BI à chaque rafraîchissement des données ou interaction, fournissant les données filtrées à représenter. Le code du développeur doit alors traduire ces données en éléments visuels, en utilisant éventuellement des bibliothèques de visualisation comme D3.js (voir section~\ref{sec:techno}).

\subsection*{Cycle de développement (pipeline)}
Le développement d’un custom visual suit généralement ce pipeline :
\begin{enumerate}
  \item Initialisation du projet via l’outil en ligne de commande (CLI) du SDK – par ex. \texttt{pbiviz new MyVisual} génère une structure de projet avec les fichiers de base (manifest, capabilities, code TS, etc.)\parencite{LearnMicrosoftSDK2}.
  \item Développement et test en local : le SDK permet de lancer un serveur local et d’attacher le visuel en développement à Power BI Desktop (en mode Developer). On peut ainsi tester le visuel dans un rapport en temps réel pendant qu’on code.
  \item Compilation et packaging : une fois le développement achevé, la commande \texttt{pbiviz package --certification-audit} produit le fichier \texttt{.pbiviz} final et exécute un contrôle automatique des appels réseau\parencite{MicrosoftAuditCLI2025}.
  \item Distribution : ce package peut être importé manuellement dans n’importe quel rapport Power BI (option \emph{Importer un visuel à partir d’un fichier}), ou déployé de façon plus gérée. Pour une utilisation interne à une organisation, l’administrateur tenant peut le publier dans le magasin organisationnel de Power BI\parencite{MicrosoftCertGuide2025}.
\end{enumerate}

Pour une distribution publique à l’écosystème Power BI, le développeur peut soumettre son visuel à AppSource, la place de marché Microsoft. Une étape de certification est alors nécessaire (voir plus loin). Microsoft indique que « n’importe quel développeur web peut créer un visuel personnalisé et le packager en un fichier .pbiviz unique à importer dans Power BI »\parencite{LearnMicrosoftSDK3}, et propose même un concours et une galerie communautaire pour encourager ces contributions\parencite{PowerBIMicrosoftCommunity}.

Depuis l’introduction des custom visuals, le catalogue AppSource s’est ainsi énormément étoffé : on y recense aujourd’hui plusieurs centaines de visuels disponibles (gratuits ou commerciaux) couvrant des usages variés (graphes, chronologies, infographies, etc.).

\subsection*{Capacités et restrictions}
Un visuel personnalisé bien conçu permet d’aller au-delà des limitations des visuels natifs. On peut par exemple introduire de nouveaux types de graphiques, des interactions innovantes, ou des designs sur mesure. Cependant, il est important de souligner que les visuels custom opèrent dans un cadre réglementé par Microsoft pour assurer la sécurité et la performance. D’une part, comme mentionné, ils tournent dans un \textit{sandbox} isolé. Cela a plusieurs implications :
\begin{itemize}
  \item[(a)] le visuel ne peut pas accéder aux données du modèle qui ne lui ont pas été explicitement fournies via les champs liés par l’utilisateur\parencite{OkVizSandbox};
  \item[(b)] il ne peut pas non plus lire ou modifier l’état d’un autre visuel ou élément du rapport (pas d’accès au DOM global ou aux variables globales de l’hôte);
  \item[(c)] le visuel ne peut dessiner qu’à l’intérieur de sa surface allouée : il lui est interdit, par exemple, de faire apparaître une fenêtre pop-up ou un élément HTML en dehors de son cadre \textit{bounding box}\parencite{OkVizDropdown}.
\end{itemize}

\textbf{Pas de communication externe non approuvée.}
Par mesure de sécurité, les visuels custom n’ont pas le droit d’envoyer des données vers un service externe ou d’en charger, sauf approbation explicite. Les règles de certification 2025 interdisent toute requête sortante pour un visuel certifié\parencite{MicrosoftCertGuide2025}. Même pour les visuels non certifiés, le sandbox applique une Content Security Policy très stricte (\texttt{default-src 'none'}) documentée dans les notes de version API v 6\parencite{MicrosoftAPIv6CSP2025}.

\subsection*{Certification et confiance}
Microsoft a mis en place un programme de certification des visuels. Un visuel certifié est un visuel custom qui a passé avec succès un processus de validation par Microsoft, incluant des vérifications de sécurité (absence de code malveillant ou de fuite de données), de performance et de conformité aux bonnes pratiques\parencite{MicrosoftCertGuide2025}. Les visuels certifiés sont signalés par une icône spéciale et certaines fonctionnalités de Power BI (export PDF/PPT, abonnements email) ne rendent que ces visuels\parencite{OkVizExport}.

\subsection*{Synthèse}
En résumé, le développement de visuels custom via le SDK offre une flexibilité maximale pour répondre à des besoins spécifiques, au prix d’un effort de développement et du respect de contraintes de sécurité. Dans la section suivante, nous discutons des choix technologiques concrets (TypeScript, D3.js et éventuellement React) pour implémenter un visuel custom.


% 2.5 Choix technologiques (TypeScript, D3, React optionnel)
%-----------------------------------------------------------
\section{Choix technologiques (TypeScript, D3, React optionnel)}
\label{sec:techno}
%-----------------------------------------------------------

Le développement d’un visuel personnalisé Power BI repose sur un \emph{stack}
web moderne articulé autour de trois briques : TypeScript pour le langage,
D3.js pour le rendu vectoriel et, à titre optionnel, React pour la
structuration de l’interface. Ce choix résulte d’une analyse des
alternatives en termes de maintenabilité, performance, sécurité et
accessibilité.

\textbf{TypeScript vs JavaScript.}
Le SDK Power BI est conçu nativement pour TypeScript, sur-ensemble typé de
JavaScript que Microsoft recommande pour les visuels personnalisés
\parencite{MicrosoftPBISDKTS2025}. Le typage statique détecte précocement
les incohérences et réduit les bogues en production : une étude
empirique portant sur plus de 400 projets GitHub montre une diminution
moyenne de 15 \% des défauts après migration vers TypeScript
\parencite{BeyerEtAl2023}. Les annotations rendent le code plus explicite,
facilitant lecture, revue et refactorisation. Par ailleurs, TypeScript
apporte des abstractions modernes — interfaces, classes, \emph{generics} —
qui encouragent une architecture modulaire et extensible. Le code est
ensuite transcompilé en JavaScript ES 2019, sans impact mesurable sur les
performances d’exécution \parencite{EcmaBenchmark2024}. Ne pas passer par
cette couche (écrire directement en JavaScript ES6+) aurait simplifié la
phase de build, mais au prix d’une dette technique accrue et d’un risque de
régression plus élevé, notamment pour un composant destiné à évoluer avec
l’API Power BI.

\textbf{D3.js pour le rendu SVG.}
D3 — « Data-Driven Documents » — est la librairie de référence pour manipuler
le DOM SVG et créer des visualisations sur mesure. Elle établit un lien
direct entre données et éléments graphiques, autorisant des
transformations déclaratives efficientes \parencite{Bostock2019}. Cette
approche bas niveau donne un contrôle total sur chaque attribut visuel
(couleur, position, anim­ation) : condition nécessaire pour implémenter un
graphique non standard conforme aux exigences métier. D3 offre en outre un
large éventail de modules (générateurs de formes, projections
cartographiques, échelles, \emph{layouts} hiérarchiques) et bénéficie d’un
écosystème mature d’exemples réutilisables. Les alternatives « haut
niveau » (Chart.js, Plotly) accélèrent le prototypage mais atteignent vite
leurs limites lorsqu’il s’agit d’un design original ; quant au
\emph{canvas} ou à WebGL, ils complexifient l’accessibilité et la netteté
du rendu zoomé. Le choix du SVG produit par D3 facilite la mise à l’échelle
et l’ajout d’attributs ARIA ou de balises \verb|<title>|, répondant aux
recommandations WCAG 2.2 pour les rapports Power BI
\parencite{W3CAccessibility2023}. En pratique, les tests préliminaires sur
un échantillon de 50 000 points affichés montrent des temps de
rendu inférieurs à 250 ms sur un poste standard, confirmant la pertinence
de D3 dans ce contexte \parencite{BenchPBI2025}.

\textbf{React (optionnel) pour l’UI.}
React n’est pas requis par le SDK, mais devient pertinent dès lors que le
visuel embarque une interface utilisateur complexe : sélecteurs, menus
contextuels, légende cliquable. Sa philosophie \emph{component-based} et
son \emph{virtual DOM} optimisent les mises à jour d’interface en
réduisant les re-rendus coûteux \parencite{ReactDocs2024}. L’association
« React pilote la structure, D3 gère les calculs et applique les
transformations » est désormais un \emph{pattern} reconnu ; plusieurs
visuels open-source l’utilisent déjà dans AppSource
\parencite{PowerBIReactD3Sample2024}. L’empreinte ajoutée (≈ 40 kB
minifiés) reste compatible avec la limite de 2 Mo du package \texttt{pbiviz}.
Pour les projets très légers, on peut encore préférer Preact, clone
allégé compatible avec l’API React. Le coût cognitif — JSX, gestion d’état
— est maîtrisé par l’équipe et amorti par la facilité de test unitaire des
composants. Si le visuel n’exige qu’un rendu statique ou des animations
D3 simples, il est cohérent de se passer de React ; c’est pourquoi l’usage
reste qualifié d’« optionnel ».

\textbf{Synthèse.}
Le triptyque TypeScript + D3 (+ React) offre un compromis robuste : rigueur
logicielle, expressivité graphique, performances maîtrisées et
accessibilité native. Il s’inscrit dans les standards de l’écosystème Power
BI, maximise la réutilisabilité du savoir-faire front-end de l’équipe et
minimise la dette technique à long terme. Les alternatives évaluées
(JavaScript pur, bibliothèques charting « tout-en-un », canvas/WebGL) ont
été jugées moins compatibles avec les objectifs de maintenance, de qualité
et de gouvernance fixés dans le présent travail.


% 2.6 Solutions concurrentes (Tableau, Qlik, Looker)
%-----------------------------------------------------------
\section{Solutions concurrentes (Tableau, Qlik, Looker)}
\label{sec:concurrence}
%-----------------------------------------------------------

Les principales plateformes BI concurrentes — Tableau, Qlik Sense et
Looker — proposent chacune des mécanismes de personnalisation de visuels
qui diffèrent de ceux de Power BI, tant sur le plan technique que sur les
implications métier. Comprendre ces approches permet de situer les
\textit{visuels custom} Power BI dans un paysage concurrentiel plus large.

\textbf{Tableau.}
Tableau étend ses capacités via les \emph{Tableau Extensions}, des
applications Web encapsulées dans les tableaux de bord grâce à l’Extension
API\parencite{TableauExtGuide2024}. Un développeur crée, en JavaScript,
un composant (souvent dans une \texttt{iframe}) qui ajoute un graphique
personnalisé ou une interaction spécifique, par exemple du \emph{write-back}
ou l’intégration d’un service tiers\parencite{TableauBlogExt2024}. Depuis
la v2019.4, les extensions sont exécutées par défaut en \emph{sandbox},
sans accès réseau, et toute extension nécessitant un accès externe doit
être explicitement inscrite sur liste blanche par l’administrateur serveur
ou site\parencite{TableauAdmin2025}. Tableau n’assure pas le support
technique des extensions ; cette responsabilité incombe à leur éditeur, si
bien que les entreprises doivent évaluer la fiabilité de la source avant
déploiement. Sur le plan financier, la licence Creator avoisine
70 USD utilisateur/mois\parencite{TableauPricing2025}, ce qui fait de la
personnalisation une option coûteuse pour les organisations sensibles au
budget, à la différence d’un environnement Power BI où le coût d’entrée est
plus faible.

\textbf{Qlik Sense.}
L’extensibilité de Qlik repose sur les \emph{Visualization Extensions} :
objets personnalisés écrits en HTML 5 / JavaScript / CSS et déclarés par un
manifeste \texttt{.qext}\parencite{QlikDevHub2024}. Une extension Qlik,
lorsqu’elle respecte l’API Qlik, s’intègre comme un objet natif : on la
fait glisser dans la feuille, on ajuste ses propriétés, et elle réagit
aux sélections grâce au moteur associatif\parencite{QlikExtAPI2024}. Le
développeur peut embarquer n’importe quelle bibliothèque (D3, three.js,
etc.) pour façonner un rendu spécifique. Qlik privilégie ici une approche
communautaire : de nombreuses extensions sont partagées via
\emph{Qlik Branch} sans validation officielle. L’administrateur doit donc
examiner puis installer manuellement l’extension sur le serveur ou le
tenant SaaS avant qu’elle soit disponible aux créateurs de contenu.
Commercialement, Qlik Sense Business se situe autour de 30 USD par
utilisateur/mois, et l’édition Enterprise encore au-dessus
\parencite{QlikPricing2025}. Les clients attendent donc qu’une capacité de
personnalisation avancée soit incluse, mais ils doivent disposer en interne
de compétences front-end JS pour en profiter pleinement.

\textbf{Looker.}
Looker (aujourd’hui composant de Google Cloud) autorise des visuels
personnalisés via le \emph{Looker Custom Visualization SDK}. Le
développeur code en JavaScript un composant qui implémente la fonction
\texttt{updateAsync} afin de recevoir les données d’une
\emph{explore} Looker et de restituer un rendu SVG ou Canvas
\parencite{LookerVizSDK2025}. Pour diffusion publique, Google impose un
processus Marketplace : le code doit être hébergé sur GitHub et passer une
revue de conformité avant publication\parencite{LookerMarketplace2024}. Il
est toutefois possible de limiter le visuel à un usage interne en l’ajoutant
directement via l’interface Admin. La philosophie lookerienne est donc plus
centralisée : la personnalisation est permise, mais sous contrôle
étroit. La tarification, négociée au cas par cas (souvent nettement au-delà
des niveaux Power BI/Tableau), signifie que les clients Looker visent
généralement une valeur élevée tirée soit du modèle LookML, soit d’un
nombre réduit, mais hautement spécifique, de visuels custom.

\textbf{Lecture comparative.}
Tableau et Qlik offrent une liberté initiale plus large — l’utilisateur
avancé peut charger une extension localement — mais délèguent la
gouvernance au client ; Power BI et Looker exigent un contrôle centralisé
(certification ou validation administrateur) avant déploiement massif.
Techniquement, toutes s’appuient sur le triptyque HTML/JS/CSS, mais Power BI
se distingue par son SDK TypeScript/D3 structuré et sa marketplace AppSource
très fournie, facteur d’adoption rapide. Sur le plan économique, Power BI
reste l’option la plus abordable pour déployer des visuels custom à grande
échelle, alors que Tableau, Qlik Sense et surtout Looker réservent la
personnalisation intensive à des environnements dont le budget justifie
l’investissement.


% 2.7 Analyse des gaps et opportunités
%-----------------------------------------------------------
\section{Synthèse des écarts \& opportunités}
\label{sec:synthese}
%-----------------------------------------------------------

À l’issue de cette analyse, il est possible de synthétiser les forces et
limites des différentes approches de visualisation de données dans Power BI
et ses alternatives, puis d’en déduire les cas d’usage privilégiés. Quatre
grandes catégories de solutions ont été examinées :

\begin{enumerate}
  \item les visuels natifs de Power BI ;
  \item les visuels basés sur des scripts R / Python intégrés à Power BI ;
  \item les solutions concurrentes (extensions Tableau, Qlik Sense,
        visuels Looker) ;
  \item les visuels personnalisés développés via le SDK Power BI.
\end{enumerate}

Chaque option présente des atouts spécifiques mais aussi des lacunes
fonctionnelles, dont la compréhension permet d’identifier des opportunités
d’amélioration ou d’utilisation optimale.

\medskip
\textbf{Visuels natifs Power BI.}  
Les visuels fournis d’office constituent le socle de la plupart des
rapports. Ils sont clefs en main, supportés par Microsoft, optimisés pour
des performances élevées et parfaitement intégrés (sélections croisées,
filtres, export PDF/PPT, affichage mobile). Leur fiabilité et leur sécurité
sont maximales puisqu’aucun code utilisateur n’est exécuté. Leur
limitation : une relative rigidité. Dès que l’on vise une représentation
moins courante — diagramme de Sankey, bullet chart spécifique, réseau,
alluvial, cartographie indoor — l’utilisateur reste tributaire des
fonctionnalités prévues. Certains visuels natifs souffrent
également de restrictions (impossibilité d’ajouter un second axe dans
certaines combinaisons, personnalisation limitée des étiquettes, etc.).
Ainsi, les visuels natifs excellent dans les usages courants avec un effort
zéro, mais offrent peu de latitude pour répondre aux demandes hors norme.

\medskip
\textbf{Visuels Python / R.}  
L’exécution d’un script Python ou R à l’intérieur de Power BI ouvre la
porte à l’immense écosystème de visualisation de ces deux langages
(matplotlib, seaborn, ggplot2, plotly offline, folium, networkx, etc.). Cet
atout est la flexibilité : tout graphique réalisable dans Python ou R peut,
en théorie, être intégré dans un rapport Power BI, ce qui est particulièrement
utile pour des analyses statistiques avancées ou des bibliothèques très
spécialisées.  
Limites :  
– rendu \emph{statique} (image PNG 72 DPI) : aucune interaction directe ;  
– plafond de 150 000 lignes transmises au script ;  
– liste de packages autorisés restreinte dans Power BI Service ;  
– exécution plus lente qu’un visuel natif (script relancé à chaque
rafraîchissement) ;  
– public restreint aux profils sachant coder.  
Les visuels R / Python sont donc idéaux pour un prototypage rapide ou
l’exploration ad hoc, mais peu adaptés à un déploiement massif et
interactif : ils servent souvent de tremplin vers un futur visuel SDK.

\medskip
\textbf{Extensions Tableau / Qlik / Looker.}  
Tableau se distingue par une bibliothèque native déjà riche ; néanmoins,
les \emph{Tableau Extensions} permettent d’incorporer des applications
web JS (dans une \texttt{iframe}) pour ajouter, par exemple, du
\emph{write-back} ou un composant interactif inédit. Depuis 2019.4, ces
extensions tournent par défaut en sandbox sans accès réseau ; un
administrateur doit inscrire sur liste blanche les extensions sortantes.  
Qlik Sense propose des \emph{Visualization Extensions} qui se comportent
comme des objets natifs ; elles participent aux sélections associatives et
peuvent être partagées via la communauté Qlik Branch. L’entreprise doit
toutefois valider le code avant déploiement sur le serveur.  
Looker, plus récent, autorise des visuels JS via le \emph{Custom
Visualization SDK}. Pour une diffusion Marketplace, Google exige une revue
GitHub, ce qui garantit qualité et sécurité mais freine la spontanéité.  
Toutes ces plateformes prouvent que l’extensibilité par code est devenue un
incontournable ; cependant, elles entraînent un effort de développement et
des coûts de licence souvent supérieurs à Power BI.

\medskip
\textbf{Visuels SDK Power BI.}  
La solution la plus puissante et flexible dans l’écosystème Power BI repose
sur le SDK (TypeScript, D3, éventuellement React). Elle permet de créer un
composant répondant exactement aux spécifications métier, avec une
intégration complète (filtres, mobile, export) et une possible diffusion
AppSource ou Organizational Visuals. Contreparties : effort de
développement élevé (semaines), maintenance continue (veille SDK,
correctifs, adaptation thèmes) et besoin de gouvernance du code. La
certification Microsoft atténue les risques en cas de diffusion publique.

\bigskip
\textbf{Vue d’ensemble comparative.}

\begin{table}[h]
\footnotesize
\centering
\begin{tabularx}{\textwidth}{lXXXXXX}
\toprule
\textbf{Critère} &
\textbf{Natifs PBI} &
\textbf{Python/R} &
\textbf{SDK PBI} &
\textbf{Ext.~Tableau} &
\textbf{Ext.~Qlik} &
\textbf{Visuels Looker} \\
\midrule
Interactivité &
Excellente, intégrée &
Statique, rafraîchie &
Équivalente aux natifs &
Bonne, API Tableau &
Native, moteur associatif &
Bonne, limitée au visuel \\
\addlinespace
Extensibilité &
Type figé &
Très large graphique &
Totale (TS/D3) &
Large (widget web) &
Large (JS libre) &
Moyenne (API Looker) \\
\addlinespace
Effort dev. &
Nul &
Faible–modéré &
Élevé &
Élevé &
Élevé &
Élevé \\
\addlinespace
Maintenance &
Microsoft &
Auteur script &
Équipe dev. &
Auteur ext. &
Auteur / communauté &
Auteur / Google \\
\addlinespace
Sécurité &
Maximale &
Sandbox packages &
Sandbox + signature &
Sandbox réseau &
Contrôle admin &
Marketplace / admin \\
\bottomrule
\end{tabularx}
\caption{Comparaison synthétique des approches de personnalisation}
\label{tab:comparaison-approches}
\end{table}

\medskip
\textbf{Conclusion.}  
Chaque approche comble des écarts laissés par les autres. Les visuels
natifs offrent robustesse et simplicité, mais limitent la créativité. Les
scripts R / Python ouvrent la voie à l’innovation visuelle rapide au prix
de l’interactivité. Les solutions concurrentes montrent que l’ensemble du
marché reconnaît l’importance de la personnalisation. Le SDK Power BI,
soutenu par un écosystème open-source foisonnant, transforme ces écarts en
opportunités : répondre à des besoins métier spécifiques, enrichir
l’offre AppSource, ou différencier les rapports par un design inédit.  
Pour Ecrins SA, investir dans la maîtrise des visuels SDK apparaît donc
comme une réponse stratégique aux limites identifiées, à condition de
mettre en place une gouvernance technique et sécuritaire adaptée — ce à
quoi le chapitre~\ref{sec:methodo} est consacré.




% =============================================================
