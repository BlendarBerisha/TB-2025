%-----------------------------------------------------------
\section{Synthèse des écarts \& opportunités}
\label{sec:synthese}
%-----------------------------------------------------------

À l’issue de cette analyse, il est possible de synthétiser les forces et
limites des différentes approches de visualisation de données dans Power BI
et ses alternatives, puis d’en déduire les cas d’usage privilégiés. Quatre
grandes catégories de solutions ont été examinées :

\begin{enumerate}
  \item les visuels natifs de Power BI ;
  \item les visuels basés sur des scripts R / Python intégrés à Power BI ;
  \item les solutions concurrentes (extensions Tableau, Qlik Sense,
        visuels Looker) ;
  \item les visuels personnalisés développés via le SDK Power BI.
\end{enumerate}

Chaque option présente des atouts spécifiques mais aussi des lacunes
fonctionnelles, dont la compréhension permet d’identifier des opportunités
d’amélioration ou d’utilisation optimale.

\medskip
\textbf{Visuels natifs Power BI.}  
Les visuels fournis d’office constituent le socle de la plupart des
rapports. Ils sont clefs en main, supportés par Microsoft, optimisés pour
des performances élevées et parfaitement intégrés (sélections croisées,
filtres, export PDF/PPT, affichage mobile). Leur fiabilité et leur sécurité
sont maximales puisqu’aucun code utilisateur n’est exécuté. Leur
limitation : une relative rigidité. Dès que l’on vise une représentation
moins courante — diagramme de Sankey, bullet chart spécifique, réseau,
alluvial, cartographie indoor — l’utilisateur reste tributaire des
fonctionnalités prévues. Certains visuels natifs souffrent
également de restrictions (impossibilité d’ajouter un second axe dans
certaines combinaisons, personnalisation limitée des étiquettes, etc.).
Ainsi, les visuels natifs excellent dans les usages courants avec un effort
zéro, mais offrent peu de latitude pour répondre aux demandes hors norme.

\medskip
\textbf{Visuels Python / R.}  
L’exécution d’un script Python ou R à l’intérieur de Power BI ouvre la
porte à l’immense écosystème de visualisation de ces deux langages
(matplotlib, seaborn, ggplot2, plotly offline, folium, networkx, etc.). Cet
atout est la flexibilité : tout graphique réalisable dans Python ou R peut,
en théorie, être intégré dans un rapport Power BI, ce qui est particulièrement
utile pour des analyses statistiques avancées ou des bibliothèques très
spécialisées.  
Limites :  
– rendu \emph{statique} (image PNG 72 DPI) : aucune interaction directe ;  
– plafond de 150 000 lignes transmises au script ;  
– liste de packages autorisés restreinte dans Power BI Service ;  
– exécution plus lente qu’un visuel natif (script relancé à chaque
rafraîchissement) ;  
– public restreint aux profils sachant coder.  
Les visuels R / Python sont donc idéaux pour un prototypage rapide ou
l’exploration ad hoc, mais peu adaptés à un déploiement massif et
interactif : ils servent souvent de tremplin vers un futur visuel SDK.

\medskip
\textbf{Extensions Tableau / Qlik / Looker.}  
Tableau se distingue par une bibliothèque native déjà riche ; néanmoins,
les \emph{Tableau Extensions} permettent d’incorporer des applications
web JS (dans une \texttt{iframe}) pour ajouter, par exemple, du
\emph{write-back} ou un composant interactif inédit. Depuis 2019.4, ces
extensions tournent par défaut en sandbox sans accès réseau ; un
administrateur doit inscrire sur liste blanche les extensions sortantes.  
Qlik Sense propose des \emph{Visualization Extensions} qui se comportent
comme des objets natifs ; elles participent aux sélections associatives et
peuvent être partagées via la communauté Qlik Branch. L’entreprise doit
toutefois valider le code avant déploiement sur le serveur.  
Looker, plus récent, autorise des visuels JS via le \emph{Custom
Visualization SDK}. Pour une diffusion Marketplace, Google exige une revue
GitHub, ce qui garantit qualité et sécurité mais freine la spontanéité.  
Toutes ces plateformes prouvent que l’extensibilité par code est devenue un
incontournable ; cependant, elles entraînent un effort de développement et
des coûts de licence souvent supérieurs à Power BI.

\medskip
\textbf{Visuels SDK Power BI.}  
La solution la plus puissante et flexible dans l’écosystème Power BI repose
sur le SDK (TypeScript, D3, éventuellement React). Elle permet de créer un
composant répondant exactement aux spécifications métier, avec une
intégration complète (filtres, mobile, export) et une possible diffusion
AppSource ou Organizational Visuals. Contreparties : effort de
développement élevé (semaines), maintenance continue (veille SDK,
correctifs, adaptation thèmes) et besoin de gouvernance du code. La
certification Microsoft atténue les risques en cas de diffusion publique.

\bigskip
\textbf{Vue d’ensemble comparative.}

\begin{table}[h]
\footnotesize
\centering
\begin{tabularx}{\textwidth}{lXXXXXX}
\toprule
\textbf{Critère} &
\textbf{Natifs PBI} &
\textbf{Python/R} &
\textbf{SDK PBI} &
\textbf{Ext.~Tableau} &
\textbf{Ext.~Qlik} &
\textbf{Visuels Looker} \\
\midrule
Interactivité &
Excellente, intégrée &
Statique, rafraîchie &
Équivalente aux natifs &
Bonne, API Tableau &
Native, moteur associatif &
Bonne, limitée au visuel \\
\addlinespace
Extensibilité &
Type figé &
Très large graphique &
Totale (TS/D3) &
Large (widget web) &
Large (JS libre) &
Moyenne (API Looker) \\
\addlinespace
Effort dev. &
Nul &
Faible–modéré &
Élevé &
Élevé &
Élevé &
Élevé \\
\addlinespace
Maintenance &
Microsoft &
Auteur script &
Équipe dev. &
Auteur ext. &
Auteur / communauté &
Auteur / Google \\
\addlinespace
Sécurité &
Maximale &
Sandbox packages &
Sandbox + signature &
Sandbox réseau &
Contrôle admin &
Marketplace / admin \\
\bottomrule
\end{tabularx}
\caption{Comparaison synthétique des approches de personnalisation}
\label{tab:comparaison-approches}
\end{table}

\medskip
\textbf{Conclusion.}  
Chaque approche comble des écarts laissés par les autres. Les visuels
natifs offrent robustesse et simplicité, mais limitent la créativité. Les
scripts R / Python ouvrent la voie à l’innovation visuelle rapide au prix
de l’interactivité. Les solutions concurrentes montrent que l’ensemble du
marché reconnaît l’importance de la personnalisation. Le SDK Power BI,
soutenu par un écosystème open-source foisonnant, transforme ces écarts en
opportunités : répondre à des besoins métier spécifiques, enrichir
l’offre AppSource, ou différencier les rapports par un design inédit.  
Pour Ecrins SA, investir dans la maîtrise des visuels SDK apparaît donc
comme une réponse stratégique aux limites identifiées, à condition de
mettre en place une gouvernance technique et sécuritaire adaptée — ce à
quoi le chapitre~\ref{sec:methodo} est consacré.
