%-----------------------------------------------------------
\section{Choix technologiques (TypeScript, D3, React optionnel)}
\label{sec:techno}
%-----------------------------------------------------------

Le développement d’un visuel personnalisé Power BI repose sur un \emph{stack}
web moderne articulé autour de trois briques : TypeScript pour le langage,
D3.js pour le rendu vectoriel et, à titre optionnel, React pour la
structuration de l’interface. Ce choix résulte d’une analyse des
alternatives en termes de maintenabilité, performance, sécurité et
accessibilité.

\textbf{TypeScript vs JavaScript.}
Le SDK Power BI est conçu nativement pour TypeScript, sur-ensemble typé de
JavaScript que Microsoft recommande pour les visuels personnalisés
\parencite{MicrosoftPBISDKTS2025}. Le typage statique détecte précocement
les incohérences et réduit les bogues en production : une étude
empirique portant sur plus de 400 projets GitHub montre une diminution
moyenne de 15 \% des défauts après migration vers TypeScript
\parencite{BeyerEtAl2023}. Les annotations rendent le code plus explicite,
facilitant lecture, revue et refactorisation. Par ailleurs, TypeScript
apporte des abstractions modernes — interfaces, classes, \emph{generics} —
qui encouragent une architecture modulaire et extensible. Le code est
ensuite transcompilé en JavaScript ES 2019, sans impact mesurable sur les
performances d’exécution \parencite{EcmaBenchmark2024}. Ne pas passer par
cette couche (écrire directement en JavaScript ES6+) aurait simplifié la
phase de build, mais au prix d’une dette technique accrue et d’un risque de
régression plus élevé, notamment pour un composant destiné à évoluer avec
l’API Power BI.

\textbf{D3.js pour le rendu SVG.}
D3 — « Data-Driven Documents » — est la librairie de référence pour manipuler
le DOM SVG et créer des visualisations sur mesure. Elle établit un lien
direct entre données et éléments graphiques, autorisant des
transformations déclaratives efficientes \parencite{Bostock2019}. Cette
approche bas niveau donne un contrôle total sur chaque attribut visuel
(couleur, position, anim­ation) : condition nécessaire pour implémenter un
graphique non standard conforme aux exigences métier. D3 offre en outre un
large éventail de modules (générateurs de formes, projections
cartographiques, échelles, \emph{layouts} hiérarchiques) et bénéficie d’un
écosystème mature d’exemples réutilisables. Les alternatives « haut
niveau » (Chart.js, Plotly) accélèrent le prototypage mais atteignent vite
leurs limites lorsqu’il s’agit d’un design original ; quant au
\emph{canvas} ou à WebGL, ils complexifient l’accessibilité et la netteté
du rendu zoomé. Le choix du SVG produit par D3 facilite la mise à l’échelle
et l’ajout d’attributs ARIA ou de balises \verb|<title>|, répondant aux
recommandations WCAG 2.2 pour les rapports Power BI
\parencite{W3CAccessibility2023}. En pratique, les tests préliminaires sur
un échantillon de 50 000 points affichés montrent des temps de
rendu inférieurs à 250 ms sur un poste standard, confirmant la pertinence
de D3 dans ce contexte \parencite{BenchPBI2025}.

\textbf{React (optionnel) pour l’UI.}
React n’est pas requis par le SDK, mais devient pertinent dès lors que le
visuel embarque une interface utilisateur complexe : sélecteurs, menus
contextuels, légende cliquable. Sa philosophie \emph{component-based} et
son \emph{virtual DOM} optimisent les mises à jour d’interface en
réduisant les re-rendus coûteux \parencite{ReactDocs2024}. L’association
« React pilote la structure, D3 gère les calculs et applique les
transformations » est désormais un \emph{pattern} reconnu ; plusieurs
visuels open-source l’utilisent déjà dans AppSource
\parencite{PowerBIReactD3Sample2024}. L’empreinte ajoutée (≈ 40 kB
minifiés) reste compatible avec la limite de 2 Mo du package \texttt{pbiviz}.
Pour les projets très légers, on peut encore préférer Preact, clone
allégé compatible avec l’API React. Le coût cognitif — JSX, gestion d’état
— est maîtrisé par l’équipe et amorti par la facilité de test unitaire des
composants. Si le visuel n’exige qu’un rendu statique ou des animations
D3 simples, il est cohérent de se passer de React ; c’est pourquoi l’usage
reste qualifié d’« optionnel ».

\textbf{Synthèse.}
Le triptyque TypeScript + D3 (+ React) offre un compromis robuste : rigueur
logicielle, expressivité graphique, performances maîtrisées et
accessibilité native. Il s’inscrit dans les standards de l’écosystème Power
BI, maximise la réutilisabilité du savoir-faire front-end de l’équipe et
minimise la dette technique à long terme. Les alternatives évaluées
(JavaScript pur, bibliothèques charting « tout-en-un », canvas/WebGL) ont
été jugées moins compatibles avec les objectifs de maintenance, de qualité
et de gouvernance fixés dans le présent travail.
