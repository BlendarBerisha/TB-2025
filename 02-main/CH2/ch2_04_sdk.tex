%-----------------------------------------------------------
\section{SDK Custom Visuals : principes, sécurité, pipeline}
\label{sec:sdk}
%-----------------------------------------------------------

Lorsque les visuels natifs ne suffisent pas et qu’un script Python/R est trop limité, la solution la plus aboutie est de créer un visuel personnalisé complet en utilisant le Software Development Kit (SDK) de Power BI. Microsoft fournit un SDK officiel (sous forme de Node Package \texttt{powerbi-visuals-tools} \textbf{(v 6.3, mai 2025)}) qui permet aux développeurs de coder, tester et empaqueter de nouveaux visuels sous forme d’un fichier \texttt{.pbiviz}\parencite{MicrosoftSDKNpm2025}.

Le principe fondamental est qu’un visuel Power BI n’est rien d’autre qu’une application web encapsulée : il contient du code HTML/JavaScript (plus précisément TypeScript, transcompilé en JavaScript) qui reçoit des données et génère du DOM (SVG, Canvas, etc.) pour afficher un graphique. Le SDK abstrait la communication avec Power BI : le développeur définit, via un fichier \texttt{capabilities.json}, quels champs et paramètres de formatage son visuel acceptera, puis implémente une classe TypeScript qui reprend l’interface \texttt{IVisual}. Cette classe comporte notamment une méthode clé \texttt{update(options)} qui est appelée par Power BI à chaque rafraîchissement des données ou interaction, fournissant les données filtrées à représenter. Le code du développeur doit alors traduire ces données en éléments visuels, en utilisant éventuellement des bibliothèques de visualisation comme D3.js (voir section~\ref{sec:techno}).

\subsection*{Cycle de développement (pipeline)}
Le développement d’un custom visual suit généralement ce pipeline :
\begin{enumerate}
  \item Initialisation du projet via l’outil en ligne de commande (CLI) du SDK – par ex. \texttt{pbiviz new MyVisual} génère une structure de projet avec les fichiers de base (manifest, capabilities, code TS, etc.)\parencite{LearnMicrosoftSDK2}.
  \item Développement et test en local : le SDK permet de lancer un serveur local et d’attacher le visuel en développement à Power BI Desktop (en mode Developer). On peut ainsi tester le visuel dans un rapport en temps réel pendant qu’on code.
  \item Compilation et packaging : une fois le développement achevé, la commande \texttt{pbiviz package --certification-audit} produit le fichier \texttt{.pbiviz} final et exécute un contrôle automatique des appels réseau\parencite{MicrosoftAuditCLI2025}.
  \item Distribution : ce package peut être importé manuellement dans n’importe quel rapport Power BI (option \emph{Importer un visuel à partir d’un fichier}), ou déployé de façon plus gérée. Pour une utilisation interne à une organisation, l’administrateur tenant peut le publier dans le magasin organisationnel de Power BI\parencite{MicrosoftCertGuide2025}.
\end{enumerate}

Pour une distribution publique à l’écosystème Power BI, le développeur peut soumettre son visuel à AppSource, la place de marché Microsoft. Une étape de certification est alors nécessaire (voir plus loin). Microsoft indique que « n’importe quel développeur web peut créer un visuel personnalisé et le packager en un fichier .pbiviz unique à importer dans Power BI »\parencite{LearnMicrosoftSDK3}, et propose même un concours et une galerie communautaire pour encourager ces contributions\parencite{PowerBIMicrosoftCommunity}.

Depuis l’introduction des custom visuals, le catalogue AppSource s’est ainsi énormément étoffé : on y recense aujourd’hui plusieurs centaines de visuels disponibles (gratuits ou commerciaux) couvrant des usages variés (graphes, chronologies, infographies, etc.).

\subsection*{Capacités et restrictions}
Un visuel personnalisé bien conçu permet d’aller au-delà des limitations des visuels natifs. On peut par exemple introduire de nouveaux types de graphiques, des interactions innovantes, ou des designs sur mesure. Cependant, il est important de souligner que les visuels custom opèrent dans un cadre réglementé par Microsoft pour assurer la sécurité et la performance. D’une part, comme mentionné, ils tournent dans un \textit{sandbox} isolé. Cela a plusieurs implications :
\begin{itemize}
  \item[(a)] le visuel ne peut pas accéder aux données du modèle qui ne lui ont pas été explicitement fournies via les champs liés par l’utilisateur\parencite{OkVizSandbox};
  \item[(b)] il ne peut pas non plus lire ou modifier l’état d’un autre visuel ou élément du rapport (pas d’accès au DOM global ou aux variables globales de l’hôte);
  \item[(c)] le visuel ne peut dessiner qu’à l’intérieur de sa surface allouée : il lui est interdit, par exemple, de faire apparaître une fenêtre pop-up ou un élément HTML en dehors de son cadre \textit{bounding box}\parencite{OkVizDropdown}.
\end{itemize}

\textbf{Pas de communication externe non approuvée.}
Par mesure de sécurité, les visuels custom n’ont pas le droit d’envoyer des données vers un service externe ou d’en charger, sauf approbation explicite. Les règles de certification 2025 interdisent toute requête sortante pour un visuel certifié\parencite{MicrosoftCertGuide2025}. Même pour les visuels non certifiés, le sandbox applique une Content Security Policy très stricte (\texttt{default-src 'none'}) documentée dans les notes de version API v 6\parencite{MicrosoftAPIv6CSP2025}.

\subsection*{Certification et confiance}
Microsoft a mis en place un programme de certification des visuels. Un visuel certifié est un visuel custom qui a passé avec succès un processus de validation par Microsoft, incluant des vérifications de sécurité (absence de code malveillant ou de fuite de données), de performance et de conformité aux bonnes pratiques\parencite{MicrosoftCertGuide2025}. Les visuels certifiés sont signalés par une icône spéciale et certaines fonctionnalités de Power BI (export PDF/PPT, abonnements email) ne rendent que ces visuels\parencite{OkVizExport}.

\subsection*{Synthèse}
En résumé, le développement de visuels custom via le SDK offre une flexibilité maximale pour répondre à des besoins spécifiques, au prix d’un effort de développement et du respect de contraintes de sécurité. Dans la section suivante, nous discutons des choix technologiques concrets (TypeScript, D3.js et éventuellement React) pour implémenter un visuel custom.
