%-----------------------------------------------------------
\section{Visuels natifs : capacités et limites}
\label{sec:visuels-natifs}
%-----------------------------------------------------------

Power~BI inclut en standard une collection d’une trentaine de visuels natifs couvrant les besoins courants de visualisation : histogrammes, courbes, secteurs, tables, matrices, cartes géographiques, jauges, etc., ainsi que des visuels analytiques plus avancés comme l’arborescence de décomposition (\textit{Decomposition Tree}) ou l’analyse d’influence (\textit{Key Influencers}).  
Ces visuels ont l’avantage d’être prêts à l’emploi, optimisés par Microsoft et intégrés de façon transparente à l’interface.  
Ils supportent nativement des fonctionnalités utiles telles que le \textit{tooltip} (info-bulle) survolé, le \textit{drill-down} (exploration hiérarchique), ou encore le paramétrage riche via le volet de format (couleurs, étiquettes, axes, etc.).  
De plus, les visuels natifs bénéficient de mises à jour régulières de Microsoft pour améliorer leurs fonctionnalités et performances.  
Par exemple, la fonctionnalité \textit{small multiples} (mini-graphiques multiples) a été ajoutée à plusieurs visuels natifs suite aux retours des utilisateurs, comblant ainsi certaines lacunes de représentation dans les versions antérieures de Power~BI.

Cependant, les visuels intégrés montrent des limitations de flexibilité et de personnalisation.  
Dans la mesure où ils sont génériques, ils ne permettent pas de répondre à tous les cas d’utilisation ou chartes graphiques spécifiques.  
Une critique fréquemment formulée concerne la faible latitude de personnalisation avancée comparé à des outils concurrents comme Tableau\parencite{FyndAcademyPBIvsTableau2024}.  
Par exemple, la mise en forme conditionnelle complexe, l’ajustement fin de la disposition des éléments, ou la combinaison de plusieurs types de visualisations en un seul objet sont souvent impossibles avec les visuels standards de Power~BI.  
Il est rapporté que «~Power~BI est moins flexible en termes de design et d’interactivité… Créer des rapports hautement sur-mesure peut se révéler contraignant~»\parencite{FyndAcademyPBIvsTableau2024}.  
Concrètement, si un besoin de visualisation s’écarte des modèles prévus (par exemple un diagramme spécifique à un domaine métier, ou une variante de graphique non proposée), l’utilisateur devra recourir soit à un visuel custom tiers, soit à des astuces (DAX pour produire une image, etc.), illustrant la portée limitée des visuels par défaut\parencite{FyndAcademyPBIvsTableau2024}.

Un autre point de comparaison défavorable est l’uniformité visuelle des rapports Power~BI utilisant exclusivement les visuels natifs.  
Même s’il est possible de personnaliser les couleurs, polices et certains styles, les éléments graphiques conservent globalement la même apparence et disposition standardisées.  
Certaines entreprises, pour des raisons de communication et d’ergonomie, souhaitent des graphiques sortant du cadre habituel (par ex. un design graphique particulier, des infographies, des animations spécifiques).  
Power~BI natif n’offre pas cette liberté créative, là où des frameworks de visualisation personnalisée (ou des outils comme Tableau avec son API) le permettent plus aisément\parencite{FyndAcademyPBIvsTableau2024}.

\textbf{Performance et contraintes techniques.}  
Par ailleurs, les visuels natifs sont soumis à des contraintes de performance lorsqu’ils manipulent de très larges volumes de données ou de nombreuses catégories.  
Microsoft recommande de limiter à \textasciitilde10–20 le nombre de champs différents assignés à un visuel pour des raisons de lisibilité et de performance, avec un maximum technique de 100 champs par visuel\parencite{MicrosoftVisualLimits2024}.  
De plus, l’exportation de données depuis un visuel est limitée (ex.~150\,000 lignes maximum exportables pour un visuel standard en mode Pro, au-delà il faut passer en mode Premium)\parencite{PowerBICommunityExportLimit2023}.

Ces restrictions poussent parfois les utilisateurs avancés à chercher des alternatives.  
Une analyse critique note ainsi que «~Power~BI, contrairement à Tableau, est assez rigide sur la personnalisation détaillée des visuels~», ce qui peut freiner des besoins spécifiques ou l’exploration interactive avancée des données\parencite{FyndAcademyPBIvsTableau2024}.

En somme, les visuels natifs offrent une base solide et fiable pour la plupart des tableaux de bord, mais ils montrent leurs limites dès qu’il s’agit de s’écarter des formats classiques ou de dépasser certaines capacités intégrées.  
Ces limites ont motivé l’apparition de solutions complémentaires, notamment les visuels \textit{custom} disponibles sur le marketplace AppSource, ou l’utilisation de scripts \textit{Python}/\textit{R} intégrés, que nous abordons dans les sections suivantes.
